\documentclass[8pt]{beamer}
%\usetheme{CambridgeUS}
%\logo{\includegraphics[scale=0.10]{../imagenes/logoa}}
\usepackage[spanish]{babel}
%\usecolortheme{seahorse}
%\usepackage{beamerthemeblackboard}
%\usepackage{graphics}
%\usecolortheme[RGB={6,138,200}]{structure}
%\usepackage[orientation=landscape, size=custom,
 % width=16, height=9, scale=0.5]{beamerposter}
%\usepackage[utf8]{inputenc}
%\usetheme{metropolis}
%\metroset{titleformat=smallcaps,block=fill}
\setbeamercolor{frametitle}{bg=titleColor,fg= white}
\newcommand{\imagen}[1]{\titlegraphic{\includegraphics[height=\paperheight]{../imagenes/#1}}}
\usetheme[titleformat=smallcaps,block=fill,sectionstyle=style2]{trigon}

% Define logos to use (comment if no logo)
\biglogo{../imagenes/logoa.jpg} % Used on titlepage only
%\smalllogo{../imagenes/logoaloratxoa.jpg} % Used on top right corner of regular frames
\usepackage{booktabs}
\usepackage[scale=2]{ccicons}

\usepackage{amsmath}
\usepackage{amsfonts}
\usepackage{amssymb}
\usepackage{graphicx}
\usepackage{colortbl}
\usepackage{tikz} 
\usetikzlibrary{matrix}
\usetikzlibrary{calendar,decorations.markings} 
\usetikzlibrary{shapes,positioning}

\usepackage{tkz-tab,tkz-euclide,tkz-fct}
\usetkzobj{all}  
\usepackage{tcolorbox} 
%\usepackage{enumitem} 
\usepackage{tasks} 
\usepackage{asymptote}  
\usepackage{cancel}
\usepackage{xfrac}

\newcommand{\sen}{\mathop{\rm sen}\nolimits}

\newcommand{\tg}{\mathop{\rm tg}\nolimits}
\newcommand{\arcsen}{\mathop{\rm arcsen}\nolimits}
\newcommand{\arctg}{\mathop{\rm arctg}\nolimits}
\newcommand{\g}{{}^\circ}     
        
\newcommand{\R}{\mathbb{R}}
\newcommand{\Z}{\mathbb{Z}}
\newcommand{\N}{\mathbb{N}}
\newcommand{\Q}{\mathbb{Q}}
\newcommand{\I}{\mathbb{I}}
\newcommand{\limite}[2]{\displaystyle \lim_{x \rightarrow #1}{#2}}
\renewcommand{\vector}[1]{\overrightarrow{#1}}

\newtcbox{\resultado}[1][center]{#1,colback=red!5!white,
colframe=red!75!black}

\newtcbox{\resaltado}[1][center]{#1,colback=blue!5!white,
colframe=blue!75!black}
\definecolor{titleColor}{rgb}{0.0, 0.42, 0.24}
\imagen{../imagenes/sarrus}
\title{Integrales}
\subtitle{Matemáticas CC.SS. II}
\author{Departamento de Matemáticas}
\date[UHEI-IVED]{ UHEI - IVED}
%\date{Matemáticas II}
\begin{document}
%\ECFJD
\titleframe
\begin{frame}
\tableofcontents
\end{frame}

\begin{frame}[t]{Primitiva de una función}
\begin{alertblock}{Primitiva de una función}
Una función $F(x)$ es primitiva de otra función $f(x)$ si y sólo si $F'(x)=f(x)$
\end{alertblock}

\pause
La función $F(x)=x^5$ es una primitiva de $f(x)=5x^4$ ya que $F'(x)=5x^4=f(x)$

\pause
La función $G(x)=x^5+4$ también es una primitiva de $f(x)=5x^4$ ya que $G'(x)=5x^4=f(x)$


\pause
Lo mismo ocurrirá para cualquier función del tipo $H(x)=x^5+C$, siendo $C$ una constante, ya que la derivada de una constante es cero.

\pause

\begin{alertblock}{Primitivas de una función}
Si la función $F(x)$ es una primitiva de otra función $f(x)$, cualquier otra función de la forma $F(x)+C$, donde $C \in \R$, es primitiva de $f(x)$.

\pause
Si la función $F(x)$ es una primitiva de otra función $f(x)$, cualquier otra función primitiva de $f(x)$ es de la forma $F(x)+C$, donde $C \in \R$
\end{alertblock}

\end{frame}

\begin{frame}[t]{Integral de una función}
\begin{alertblock}{Integral de una función}
La integral de una función $f(x)$ es el conjunto de todas sus primitivas y se representa como $\int f(x) \, dx$.

Esta expresión se lee de la siguiente forma: <<Integral de $f(x)$ diferencial de $x$>>

\pause
Por lo tanto, si $F(x)$ es una primitiva de $f(x)$ entonces:
\[ \int f(x)\,dx= F(x)+C \]
donde $C$ es la constante de integración.
\end{alertblock}

\textbf{Propiedades de la integral}

\pause
\begin{enumerate}
\item $\displaystyle \int \left[f(x) \pm g(x) \right]\,  dx = \int f(x)\, dx \pm \int g(x)\, dx $
\pause
\item $\displaystyle \int \left[k \cdot f(x) \right]\,  dx = k\cdot \int f(x)\, dx $
\end{enumerate}
\end{frame}

\begin{frame}[t]{Integrales}
\begin{exampleblock}{Ejemplo}
Resuelve las siguientes integrales
\begin{tasks}[label=\alph*)](1)
\task $\displaystyle \int \left( 5x^4 + 2x \right)\, dx$
\task $\displaystyle \int 3e^x \, dx$
\end{tasks}
\end{exampleblock}

\end{frame}

\begin{frame}[t]{Integrales de funciones elementales}
\begin{alertblock}{Integral de la función constante}
\[ \int k \, dx = kx+C \]
\end{alertblock}
\pause

\begin{alertblock}{Integral funciones potenciales}
\[ \int x^n \, dx = \dfrac{x^{n+1}}{n+1}+C \qquad \int f'(x)\cdot \left[f(x)\right]^n \, dx = \dfrac{\left[f(x)\right]^{n+1}}{n+1}+C \qquad n \neq -1  \]
\end{alertblock}
\pause
\begin{alertblock}{Integral del tipo logarítmico}
\[ \int \dfrac{1}{x} \, dx = \ln |x|+C \qquad \int \dfrac{f'(x)}{f(x)}\, dx = \ln | f(x) | +C   \]
\end{alertblock}
\pause
\begin{alertblock}{Integral de las funciones exponenciales}
\[ \int a^x \, dx = \dfrac{a^x}{\ln a} +C \qquad \int f'(x)\cdot a^{f(x)}\, dx = \dfrac{a^{f(x)}}{\ln a} +C   \]
\[ \int e^{x} \, dx = e^x+C \qquad \int f'(x)\cdot e^{f(x)}\, dx = e^{f(x)} +C   \]
\end{alertblock}
\end{frame}
\begin{frame}[t]{Integrales de funciones elementales}
\begin{alertblock}{Integral de las funciones trigonométricas}
\[ \int \sen x \, dx = -\cos x + C \qquad \int f'(x) \cdot \sen f(x) \, dx = -\cos f(x) +C \]
\pause
\[ \int \cos x \, dx = \sen x + C \qquad \int f'(x) \cdot \cos f(x) \, dx = \sen f(x) +C \]
\pause
\[ \int (1+\tg^2) x \, dx = \tg x + C \qquad \int f'(x) \cdot (1+\tg^2 f(x)) \, dx = \tg f(x) +C \]
\pause
\[ \int \dfrac{1}{\cos^2 x} \, dx = \tg x + C \qquad \int \dfrac{f'(x)}{\cos^2 f(x)} \, dx = \tg f(x) +C \]
\end{alertblock}
\end{frame}

\begin{frame}[t]{Integrales}
\begin{exampleblock}{Ejemplo}
Calcular las siguientes integrales:
\begin{tasks}[label=\alph*)](2)
\task $\displaystyle \int x^5 \, dx$
\task $\displaystyle \int \sqrt[4]{x}\, dx$
\task $\displaystyle \int \dfrac{1}{x^3}\, dx$
\task $\displaystyle \int \left(x^2- \dfrac{1}{2x^2}\right)\, dx$
\task $\displaystyle \int 6x\cdot (3x^2-5)^4 \, dx$
\task $\displaystyle \int x^3 \cdot (3x^4-2)^3\, dx$
\task $\displaystyle \int \dfrac{5x}{1-2x^2}\, dx$
\task $\displaystyle \int \dfrac{5x}{(1-2x^2)^3}\, dx$
\task $\displaystyle \int 2^x \, dx$
\task $\displaystyle \int e^{2x} \, dx$
\task $\displaystyle \int x^2 \cdot e^{x^3+2} \, dx$
\task $\displaystyle \int 2^{\frac{x}{3}}\, dx$
\task $\displaystyle \int \cos(5x+1) \, dx$
\task $\displaystyle \int x^2\cdot \sen (2x^3+1) \, dx$
\task $\displaystyle \int x\cdot (1+tg^2(1-x^2)) \, dx$
\task $\displaystyle \int \sen^2x \cdot \cos x \, dx$
\end{tasks}
\end{exampleblock}
\end{frame}

\begin{frame}[t]{Integral definida}
\begin{alertblock}{Regla de Barrow}
Si $f(x)$ es continua en el intervalo $[a,b]$, y $F(x)$ es una primitiva de $f(x)$, entonces:
\[ \int_a^b f(x) \, dx = F(b)-F(a) \]
\end{alertblock}

\pause
Para calcular $\displaystyle \int_a^b f(x) \, dx$, siendo $f(x)$ continua en el intervalo $[a,b]$ procederemos de la siguiente forma:

\begin{enumerate}[<+-| alert@+>]
\item Resolveremos la integral como una integral definida para calcular $F(x)$ que es una primitiva de $f(x)$.
\item Calcularemos los valores de esta función en $a$ y $b$: $F(a)$ y $F(b)$.
\item Hallaremos la diferencia entre estos dos valores que será el valor de la integral definida.
\[ \int_a^b f(x) \, dx = \left[ F(x) \right]_a^b =  F(b)-F(a) \]
\end{enumerate}
\end{frame}

\begin{frame}[t]{Integral definida}
\begin{exampleblock}{Ejemplo}
Calcular las siguiente integrales definidas:
\begin{tasks}[label=\alph*)](1)
\task  $ \displaystyle \int_1^5 (-2x^2+x-1) \, dx $.
\task  $ \displaystyle \int_{-2}^2 (2x^3-4x+3) \, dx $.
\task  $ \displaystyle \int_0^e \dfrac{3x}{x^2+1} \, dx $.
\end{tasks}

\end{exampleblock}
\end{frame}
\end{document}