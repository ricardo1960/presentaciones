\documentclass[8pt]{beamer}
%\usetheme{CambridgeUS}
\logo{\includegraphics[scale=0.10]{../imagenes/logoa}}
\usepackage[spanish]{babel}
%\usecolortheme{seahorse}
%\usepackage{beamerthemeblackboard}
%\usepackage{graphics}
%\usecolortheme[RGB={6,138,200}]{structure}
%\usepackage[orientation=landscape, size=custom,
 % width=16, height=9, scale=0.5]{beamerposter}
%\usepackage[utf8]{inputenc}
\usetheme{metropolis}
\metroset{titleformat=smallcaps,block=fill}
\usepackage{booktabs}
\usepackage[scale=2]{ccicons}

\usepackage{amsmath}
\usepackage{amsfonts}
\usepackage{amssymb}
\usepackage{graphicx}
\usepackage{colortbl}
\usepackage{tikz} 
\usetikzlibrary{matrix}
\usetikzlibrary{calendar,decorations.markings} 
\usetikzlibrary{shapes,positioning}

\usepackage{tkz-tab,tkz-euclide,tkz-fct}
\usetkzobj{all}  
\usepackage{tcolorbox} 
%\usepackage{enumitem} 
\usepackage{tasks} 
\usepackage{asymptote}  
\newcommand{\sen}{\mathop{\rm sen}\nolimits}

\newcommand{\tg}{\mathop{\rm tg}\nolimits}
\newcommand{\arcsen}{\mathop{\rm arcsen}\nolimits}
\newcommand{\arctg}{\mathop{\rm arctg}\nolimits}
\newcommand{\g}{{}^\circ}     
        
\newcommand{\R}{\mathbb{R}}
\newcommand{\Z}{\mathbb{Z}}
\newcommand{\N}{\mathbb{N}}
\newcommand{\Q}{\mathbb{Q}}
\newcommand{\I}{\mathbb{I}}
\newcommand{\limite}[2]{\displaystyle \lim_{x \rightarrow #1}{#2}}
\renewcommand{\vector}[1]{\overrightarrow{#1}}

\newtcbox{\resultado}[1][center]{#1,colback=red!5!white,
colframe=red!75!black}

\newtcbox{\resaltado}[1][center]{#1,colback=blue!5!white,
colframe=blue!75!black}
\definecolor{titleColor}{rgb}{0.0, 0.42, 0.24}

\title{Aplicaciones de las derivadas 1}
\author{Ricardo Mateos}
\institute[UHEI-IVED]{Departamento de Matemáticas \\ UHEI - IVED}
\date{Matemáticas Aplicadas a las CC.SS. II}
\begin{document}
%\ECFJD
\begin{frame}
\maketitle
\end{frame}

\begin{frame}
\tableofcontents
\end{frame}

\begin{frame}[t]{Recta tangente}
\begin{exampleblock}{Ejemplo}
Hallar la recta tangente a la gráfica de la función $f(x)=\dfrac{2x}{x^2+1}$ en el punto $x=0$.
\end{exampleblock}
\end{frame}

\begin{frame}[t]{Recta tangente}
\begin{exampleblock}{Ejemplo}
Dada la función real de variable real definida por:
\[ f(x)= \begin{cases} x^2-x-1 & x \leq 3 \\ \\ \dfrac{3a}{x} \end{cases} \]
\begin{tasks}[label=\alph*)](1)
\task Determine el valor del parámetro real $a$ para que la función $f(x)$ sea continua en todo su dominio. ¿Para ese valor de $a$ es $f(x)$ derivable?
\task Para $a = 1$, calcule la ecuación de la recta tangente a la gráfica de la función en el punto de abscisa $x = 1$.
\end{tasks}

\end{exampleblock}
\end{frame}

\begin{frame}[t]{Monotonia y extremos relativos}
\begin{exampleblock}{Ejemplo}
Hallar los intervalos de crecimiento y los extremos relativos de la función: $f(x)=4x^3-24x^2+36x+100$
\end{exampleblock}
\end{frame}

\begin{frame}[t]{Monotonia y extremos relativos}
\begin{exampleblock}{Ejemplo}
Hallar los intervalos de crecimiento y los extremos relativos de la función: $f(x)=\dfrac{1-x^2}{x^2-4}$
\end{exampleblock}
\end{frame}


\begin{frame}[t]{Monotonía}
\begin{exampleblock}{Ejemplo}
Hallar los intervalos de crecimiento y decrecimiento y los extremos relativos de la función: $f(x)=x^2 \cdot e^{-x}$.
\end{exampleblock}
\end{frame}


\begin{frame}[t]{Curvatura y puntos de inflexión}
\begin{exampleblock}{Ejemplo}
Hallar la curvatura y los puntos de inflexión de la función: $f(x)=x^4-6x^2$
\end{exampleblock}
\end{frame}

\begin{frame}[t]{Curvatura y puntos de inflexión}
\begin{exampleblock}{Ejemplo}
Hallar la curvatura y los puntos de inflexión de la función: $f(x)=\dfrac{6}{x^2+3}$
\end{exampleblock}
\end{frame}

\end{document}