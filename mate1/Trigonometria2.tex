\documentclass[8pt]{beamer}
%\usetheme{CambridgeUS}
\logo{\includegraphics[scale=0.10]{../imagenes/logoa}}
\usepackage[spanish]{babel}
%\usecolortheme{seahorse}
%\usepackage{beamerthemeblackboard}
%\usepackage{graphics}
%\usecolortheme[RGB={6,138,200}]{structure}
%\usepackage[orientation=landscape, size=custom,
 % width=16, height=9, scale=0.5]{beamerposter}
%\usepackage[utf8]{inputenc}
\usetheme{metropolis}
\metroset{titleformat=smallcaps,block=fill}
\usepackage{booktabs}
\usepackage[scale=2]{ccicons}

\usepackage{amsmath}
\usepackage{amsfonts}
\usepackage{amssymb}
\usepackage{graphicx}
\usepackage{colortbl}
\usepackage{tikz} 
\usetikzlibrary{matrix}
\usetikzlibrary{calendar,decorations.markings} 
\usetikzlibrary{shapes,positioning}

\usepackage{tkz-tab,tkz-euclide,tkz-fct}
\usetkzobj{all}  
\usepackage{tcolorbox} 
%\usepackage{enumitem} 
\usepackage{tasks} 
\usepackage{asymptote}  

\newcommand{\sen}{\mathop{\rm sen}\nolimits}
\newcommand{\arcsen}{\mathop{\rm arcsen}\nolimits}
\newcommand{\tg}{\mathop{\rm tg}\nolimits}
\newcommand{\arctg}{\mathop{\rm arctg}\nolimits}
\newcommand{\cotg}{\mathop{\rm cotg}\nolimits}
\newcommand{\arccotg}{\mathop{\rm arccotg}\nolimits}
        
\newcommand{\R}{\mathbb{R}}
\newcommand{\Z}{\mathbb{Z}}
\newcommand{\N}{\mathbb{N}}
\newcommand{\Q}{\mathbb{Q}}
\newcommand{\I}{\mathbb{I}}
\newcommand{\limite}[2]{\displaystyle \lim_{x \rightarrow #1}{#2}}
\renewcommand{\vector}[1]{\overrightarrow{#1}}
\newcommand{\g}{{}^\circ}
\newtcbox{\resultado}[1][center]{#1,colback=red!5!white,
colframe=red!75!black}

\newtcbox{\resaltado}[1][center]{#1,colback=blue!5!white,
colframe=blue!75!black}
\definecolor{titleColor}{rgb}{0.0, 0.42, 0.24}

\title{}
\author{Ricardo Mateos}
\institute[UHEI-IVED]{Departamento de Matemáticas \\ UHEI - IVED}
\date{Matemáticas I}
\begin{document}
%\ECFJD
\begin{frame}
\maketitle
\end{frame}

\begin{frame}
\tableofcontents
\end{frame}

\section{Fórmulas trigonométricas}

\begin{frame}{Fórmulas trigonométricas}
\begin{alertblock}{Suma y diferencia de ángulos}
$\begin{array}{ll}
\sen(\alpha+\beta)=\sen\alpha\cos\beta+\cos\alpha\sen\beta & 
\sen(\alpha-\beta)=\sen\alpha\cos\beta-\cos\alpha\sen\beta \\
\cos(\alpha+\beta)=\cos\alpha\cos\beta-\sen\alpha\sen\beta & 
\cos(\alpha-\beta)=\cos\alpha\cos\beta+\sen\alpha\sen\beta \\
\tg(\alpha+\beta)=\dfrac{\tg\alpha+\tg\beta}{1-\tg\alpha\tg\beta} & 
\tg(\alpha-\beta)=\dfrac{\tg\alpha-\tg\beta}{1+\tg\alpha\tg\beta}
\end{array}$
\end{alertblock}

\begin{columns}
\begin{column}{0.45\textwidth}
\begin{alertblock}{Ángulo doble}
$\begin{array}{l}
\sen 2\alpha=2\sen\alpha\cos\beta  \\
\cos 2\alpha=\cos^2\alpha-\sen^2\alpha \\
\tg 2\alpha =\dfrac{2\tg\alpha}{1-\tg^2\alpha} 
\end{array}$
\vspace{2.3em}

\end{alertblock}
\end{column}
\begin{column}{0.45\textwidth}
\begin{alertblock}{Ángulo mitad}
$\begin{array}{l}
\sen \dfrac{\alpha}{2}=\pm \sqrt{\dfrac{1-\cos\alpha}{2}} \\
\cos \dfrac{\alpha}{2}=\pm \sqrt{\dfrac{1+\cos\alpha}{2}} \\
\tg \dfrac{\alpha}{2}=\pm \sqrt{\dfrac{1-\cos\alpha}{1+\cos\alpha}} 
\end{array}$
\end{alertblock}
\end{column}
\end{columns}
\end{frame}

\begin{frame}{Fórmulas trigonométricas}
\begin{alertblock}{Transformaciones de sumas en productos}
$\begin{array}{ll}
\sen A + \sen B =2 \sen\left( \dfrac{A+B}{2}\right) \cos\left( \dfrac{A-B}{2}\right)  \\ \\ 
\sen A - \sen B =2 \cos\left( \dfrac{A+B}{2}\right) \sen\left( \dfrac{A-B}{2}\right)  \\ \\ 
\cos A + \cos B =2 \cos\left(  \dfrac{A+B}{2}\right) \cos\left( \dfrac{A-B}{2}\right)  \\ \\
\cos A - \cos B =-2 \sen\left( \dfrac{A+B}{2}\right) \sen\left( \dfrac{A-B}{2}\right)  \\
\end{array}$
\end{alertblock}
\end{frame}
\section{Identidades trigonométricas}
\begin{frame}{Identidades trigonométricas}
\begin{alertblock}{Ejemplo}
Comprobar que $\dfrac{\cos x +\cosec x}{\sen x +\sec x}= \cotg x$
\end{alertblock}

\[\dfrac{\cos x +\cosec x}{\sen x +\sec x}= \cotg x \Rightarrow \dfrac{\cos x + \dfrac{1}{\sen x}}{\sen x +\dfrac{1}{\cos x}}= \cotg x \Rightarrow\]
\[\Rightarrow \dfrac{\dfrac{\sen x \cos x +1}{\sen x}}{\dfrac{\cos x \sen x +1}{\cos x}}= cotg x \Rightarrow \dfrac{(\sen x \cos x +1)\cos x }{(\sen x \cos x +1)\sen x}= \cotg x \Rightarrow\]
\[\Rightarrow \dfrac{\cos x }{\sen x}= \cotg x \Rightarrow \cotg x = \cotg x\]

\end{frame}

\begin{frame}{Identidades trigonométricas}
\begin{alertblock}{Ejemplo}
Comprobar que $\dfrac{\sen (\alpha + \beta)}{\cos \alpha \cos \beta}= \tg \alpha +\tg \beta$
\end{alertblock}

\[\dfrac{\sen (\alpha + \beta)}{\cos \alpha \cos \beta}= \tg \alpha +\tg \beta \Rightarrow \dfrac{\sen \alpha \cos \beta +\cos \alpha \sen \beta}{\cos \alpha \cos \beta}= \tg \alpha +\tg \beta \Rightarrow\]

\[\dfrac{\sen \alpha  \cos \beta}{\cos \alpha \cos \beta}+\dfrac{\cos \alpha  \sen \beta}{\cos \alpha \cos \beta}= \tg \alpha +\tg \beta \Rightarrow \dfrac{\sen \alpha}{\cos \alpha}+\dfrac{\sen \beta}{\cos \beta}= \tg \alpha +\tg \beta \Rightarrow\]

\[ \Rightarrow \tg \alpha +\tg \beta = \tg \alpha +\tg \beta \]
\end{frame}
\section{Ecuaciones trigonométricas}

\begin{frame}{Ecuaciones trigonométricas}
\begin{alertblock}{Ejemplo}
Resolver la ecuación $2\sen (2x+60)= \sqrt{3}$
\end{alertblock}
Primero aislamos la razón trigonométrica 

$\sen(2x+60)=\dfrac{\sqrt{3}}{2}$

Buscamos los ángulos donde el seno vale $\dfrac{\sqrt{3}}{2}$. Los menores de $360\g$ son $60\g$ y $120\g$. Sumamos a estos ángulos vueltas completas $k\cdot 360\g$, para hallar todas las soluciones.

$2x+60 = \begin{cases} 60\g+ k\cdot 360\g \\ 120\g+ k\cdot 360\g \end{cases} \Rightarrow 
2x = \begin{cases} 0\g+ k\cdot 360\g \\ 60\g+ k\cdot 360\g \end{cases} \Rightarrow$

\resultado{$x = \begin{cases} 0\g+ k\cdot 180\g \\ 30\g+ k\cdot 180\g \end{cases}\quad (k \in \Z)$}

\end{frame}

\begin{frame}{Ecuaciones trigonométricas}
\begin{alertblock}{Ejemplo}
Resolver la ecuación $\sen 2x - \cos x= 0$
\end{alertblock}
Primero desarrollamos $\sen 2x$.

$\sen 2x - \cos x= 0 \Rightarrow 2\sen x \cdot \cos x -\cos x =0$

Sacamos factor común. $\cos x (2\sen x -1)=0 \Rightarrow \begin{cases} \cos x = 0 \\ 2\sen x-1= 0 \end{cases}$

Hallamos las soluciones en cada uno de los casos:

$\cos x = 0 \Rightarrow x= \begin{cases} 90\g+k\cdot 360\g \\ 180\g+k\cdot 360\g \end{cases} \quad (k \in \Z)$
 
$2\sen x -1= 0 \Rightarrow \sen x=\dfrac{1}{2} \Rightarrow x= \begin{cases} 30\g+k\cdot 360\g \\ 150\g+k\cdot 360\g \end{cases} \quad (k \in \Z)$

\end{frame}

\begin{frame}{Ecuaciones trigonométricas}
\begin{alertblock}{Ejemplo}
Resolver la ecuación $\sen 6x - \sen 4x= 0$
\end{alertblock}
Transformamos la suma en producto:

$\sen 6x - \sen 4x= 0 \Rightarrow 2\cos \left( \dfrac{6x+4x}{2}\right)  \sen \left( \dfrac{6x-4x}{2}\right)=0 \Rightarrow 2\cos 5x \sen x =0 $

Hallamos las soluciones en cada uno de los casos:

$\cos 5x = 0 \Rightarrow 5x= \begin{cases} 90\g+k\cdot 360\g \\ 270\g+k\cdot 360\g \end{cases} \Rightarrow x= \begin{cases} 18\g+k\cdot 72\g \\ 54\g+k\cdot 72\g \end{cases} \quad (k \in \Z)$
 
$\sen x = 0 \Rightarrow x= \begin{cases} 0\g+k\cdot 360\g \\ 180\g+k\cdot 360\g \end{cases} \quad (k \in \Z)$

\end{frame}

\begin{frame}{Ecuaciones trigonométricas}
\begin{alertblock}{Ejemplo}
Resolver la ecuación $\cos 2x+\sen x =4\sen^2 x$
\end{alertblock}
Transformamos la ecuación para que haya una sola razón trigonométrica.

$\cos 2x+\sen x =4\sen^2 x \Rightarrow \cos^2x - \sen^2 x +\sen x = 4 \sen^2 x $

$ 1-\sen^2x - \sen^2 x +\sen x = 4 \sen^2 x \Rightarrow - 6\sen^2 x +\sen x +1= 0$

$\sen x = \dfrac{1\pm \sqrt{1+24}}{2} = \begin{cases} \dfrac{1}{2} \\ \\ -\dfrac{1}{3} \end{cases}$

Hallamos las soluciones en cada uno de los casos:

$\sen x = \dfrac{1}{2} \Rightarrow x= \begin{cases} 30\g+k\cdot 360\g \\ 150\g+k\cdot 360\g \end{cases} \quad (k \in \Z)$
 
$\sen x = -\dfrac{1}{3} \Rightarrow x= \begin{cases} 199,17\g+k\cdot 360\g \\ 340,53\g+k\cdot 360\g \end{cases} \quad (k \in \Z)$

\end{frame}


\end{document}