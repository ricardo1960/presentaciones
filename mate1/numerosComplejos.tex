\documentclass[8pt]{beamer}
%\usetheme{CambridgeUS}
\logo{\includegraphics[scale=0.10]{../imagenes/logoa}}
\usepackage[spanish]{babel}
%\usecolortheme{seahorse}
%\usepackage{beamerthemeblackboard}
%\usepackage{graphics}
%\usecolortheme[RGB={6,138,200}]{structure}
%\usepackage[orientation=landscape, size=custom,
 % width=16, height=9, scale=0.5]{beamerposter}
%\usepackage[utf8]{inputenc}
\usetheme{metropolis}
\metroset{titleformat=smallcaps,block=fill}
\usepackage{booktabs}
\usepackage[scale=2]{ccicons}

\usepackage{amsmath}
\usepackage{amsfonts}
\usepackage{amssymb}
\usepackage{graphicx}
\usepackage{colortbl}
\usepackage{tikz} 
\usetikzlibrary{matrix}
\usetikzlibrary{calendar,decorations.markings} 
\usetikzlibrary{shapes,positioning}

\usepackage{tkz-tab,tkz-euclide,tkz-fct}
\usetkzobj{all}  
\usepackage{tcolorbox} 
%\usepackage{enumitem} 
\usepackage{tasks} 
\usepackage{asymptote}  

\newcommand{\sen}{\mathop{\rm sen}\nolimits}
\newcommand{\arcsen}{\mathop{\rm arcsen}\nolimits}
\newcommand{\tg}{\mathop{\rm tg}\nolimits}
\newcommand{\arctg}{\mathop{\rm arctg}\nolimits}
\newcommand{\cotg}{\mathop{\rm cotg}\nolimits}
\newcommand{\arccotg}{\mathop{\rm arccotg}\nolimits}
        
\newcommand{\R}{\mathbb{R}}
\newcommand{\Z}{\mathbb{Z}}
\newcommand{\N}{\mathbb{N}}
\newcommand{\Q}{\mathbb{Q}}
\newcommand{\I}{\mathbb{I}}
\newcommand{\limite}[2]{\displaystyle \lim_{x \rightarrow #1}{#2}}
\renewcommand{\vector}[1]{\overrightarrow{#1}}
\newcommand{\g}{{}^\circ}
\newtcbox{\resultado}[1][center]{#1,colback=red!5!white,
colframe=red!75!black}

\newtcbox{\resaltado}[1][center]{#1,colback=blue!5!white,
colframe=blue!75!black}
\definecolor{titleColor}{rgb}{0.0, 0.42, 0.24}

\title{Números Complejos}
\author{Ricardo Mateos}
\institute[UHEI-IVED]{Departamento de Matemáticas \\ UHEI - IVED}
\date{Matemáticas I}
\begin{document}
%\ECFJD
\begin{frame}
\maketitle
\end{frame}

\begin{frame}
\tableofcontents
\end{frame}

\section{Números complejos}

\begin{frame}[t]{Números complejos}
\begin{exampleblock}{Ejemplo}
Resolver la siguiente ecuación: $x^2-6x+13=0$
\end{exampleblock}
\end{frame}

\begin{frame}{Números complejos}
\begin{exampleblock}{Ejemplo}
Rellenar la siguiente tabla:
\end{exampleblock}
\begin{tabular}{|p{3cm}|p{1cm}|p{1cm}|p{1cm}|p{1cm}|p{1cm}|}
\hline
& $3+2i$ & $-4+i$ & $-2-4i$ & $ 3i$ & $6 $  \\ \hline
Parte real & &  & & & \\ \hline
Parte imaginaria & & & & & \\ \hline
Opuesto & & & & &  \\ \hline
Conjugado  & & & & & \\ \hline
\end{tabular}

\end{frame}


\section{Operaciones con números complejos}
\begin{frame}[t]{Suma, resta, multiplicación y división}
\begin{exampleblock}{Ejemplo}
Dados los siguientes números complejos: $z_1= 2-i $, $z_ 2= -3+2i$, $z_3= -1+4i$ y $z_4=2i$, realizar las siguientes operaciones:
\begin{tasks}[label=\alph*)](3)
\task $z_1+z_2$
\task $z_2-3z_3$
\task $3z_1+4z_2- 3z_3$
\task $z_1\cdot z_2$
\task $z_2\cdot z_3$
\task $z_1\cdot z_2 \cdot z_3$
\task $\dfrac{z_1}{z_2}$
\task $\dfrac{z_3}{z_4}$
\task $ \dfrac{z_1 \cdot z_2}{z_3 \cdot z_4}$
\end{tasks}
\end{exampleblock}
\end{frame}

\begin{frame}[t]{Potencias de números complejos}
\begin{exampleblock}{Ejemplo}
Calcular las siguientes potencias:
\begin{tasks}[label=\alph*)](3)
\task $(2i)^5$
\task $(1-i)^4$
\task $(2+2i)^3$
\end{tasks}
\end{exampleblock}
\end{frame}

\begin{frame}[t]{Operaciones con números complejos}
\begin{exampleblock}{Ejemplo}
Hallar $x$ para que $(25-xi)^2$ sea imaginario puro.
\end{exampleblock}
\end{frame}

\begin{frame}[t]{Operaciones con números complejos}
\begin{exampleblock}{Ejemplo}
Hallar $a$ y $b$ para que se cumpla $(2-ai)\cdot(3-bi)=8+4i$
\end{exampleblock}

\end{frame}

\begin{frame}[t]{Operaciones con números complejos}
\begin{exampleblock}{Ejemplo}
Hallar $a$ y $b$ para que se cumpla $a-3i= \dfrac{2+bi}{5-3i}$
\end{exampleblock}

\end{frame}

\begin{frame}[t]{Operaciones con números complejos}
\begin{exampleblock}{Ejemplo}
Hallar  $b$ para que el producto  $(3-6i)\cdot(4+bi)$ sea un número:
\begin{tasks}[label=\alph*)](2)
\task Imaginario puro.
\task Real.
\end{tasks}
\end{exampleblock}

\end{frame}

\section{Números complejos en forma polar}
\begin{frame}[t]{Representación de números complejos}
\begin{exampleblock}{Ejemplo}
Representar los siguientes números complejos:
\begin{tasks}[label=\alph*)](2)
\task $3+2i$
\task $2-i$
\task $-4+3i$
\task $-2-3i$
\task $4$
\task $-3i$
\end{tasks}
\end{exampleblock}


\end{frame}

\begin{frame}[t]{Representación de números complejos}
\begin{exampleblock}{Ejemplo}
Hallar la solución de la siguiente ecuación y representarla en el plano complejo:
\[ z^2-4z+13=0 \]
\end{exampleblock}
\end{frame}

\begin{frame}[t]{Números complejos en forma polar}
\begin{exampleblock}{Ejemplo}
Expresar en forma polar los siguientes números complejos:
\begin{tasks}[label=\alph*)](3)
\task $1+i$
\task $-\sqrt{3}-i$
\task $2i$
\task $4$
\task $2\sqrt{3}-2i$
\task $-1+\sqrt{3}i$
\end{tasks}
\end{exampleblock}

\end{frame}

\begin{frame}[t]{Operaciones en forma polar}
\begin{exampleblock}{Ejemplo}
Realizar las siguientes operaciones en forma polar:
\begin{tasks}[label=\alph*)](2)
\task $1_{120} \cdot 2_{280}$
\task $\left(\sqrt{2} \right)_{45} \cdot \left( 2\sqrt{3} \right)_{90}$
\task $\dfrac{6_{270}}{6_{90}}$
\task $\dfrac{4_{30}}{2_{120}}$
\task $\left( 2_{30}\right)^6$
\task $\left( 1_{120}\right)^5$
\end{tasks}
\end{exampleblock}
\end{frame}

\begin{frame}[t]{Operaciones en forma polar}
\begin{exampleblock}{Ejemplo}
Realizar las siguientes operaciones:
\begin{tasks}[label=\alph*)](2)
\task $\left( \dfrac{\sqrt{2}}{2}+ \dfrac{\sqrt{2}}{2} i\right)^8$
\task $\left( 2\sqrt{3}-2i \right)^5 $
\end{tasks}
\end{exampleblock}
\end{frame}


\begin{frame}[t]{Radicales de números complejos}
\begin{exampleblock}{Ejemplo}
Calcular las siguientes raices:
\begin{tasks}[label=\alph*)](2)
\task $\sqrt[3]{-i}$
\task $\sqrt[4]{16_{120}}$
\task $\sqrt[8]{6561}$
\task $\sqrt[5]{2_{150}}$
\end{tasks}
\end{exampleblock}
\end{frame}

\begin{frame}[t]{Radicales en forma polar}
\begin{exampleblock}{Ejemplo}
Halla las siguientes raices:
\begin{tasks}[label=\alph*)](2)	
\task $\sqrt[3]{\dfrac{1-i}{1+i}}$
\task $\sqrt[5]{\dfrac{-64}{\sqrt{3}+i}}$
\end{tasks}
\end{exampleblock}
\end{frame}

\begin{frame}[t]{Ecuaciones}
\begin{exampleblock}{Ejemplo}
Resolver las siguientes ecuaciones.
\begin{tasks}[label=\alph*)](2)
\task $z^3-6z^2+12z-16=0$
\task $z^4 -256 =0 $
\task $z^3-8=0$
\task $z^4+\left(1+\sqrt{3}i\right)=0$
\end{tasks}
\end{exampleblock}
\end{frame}
\end{document}