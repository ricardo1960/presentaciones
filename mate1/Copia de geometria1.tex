\documentclass[8pt]{beamer}
%\usetheme{CambridgeUS}
%\logo{\includegraphics[scale=0.10]{../imagenes/logoa}}
\usepackage[spanish]{babel}
%\usecolortheme{seahorse}
%\usepackage{beamerthemeblackboard}
%\usepackage{graphics}
%\usecolortheme[RGB={6,138,200}]{structure}
%\usepackage[orientation=landscape, size=custom,
 % width=16, height=9, scale=0.5]{beamerposter}
%\usepackage[utf8]{inputenc}
%\usetheme{metropolis}
%\metroset{titleformat=smallcaps,block=fill}
\setbeamercolor{frametitle}{bg=titleColor,fg= white}
\newcommand{\imagen}[1]{\titlegraphic{\includegraphics[height=\paperheight]{../imagenes/#1}}}
\usetheme[titleformat=smallcaps,block=fill,sectionstyle=style2]{trigon}

% Define logos to use (comment if no logo)
\biglogo{../imagenes/logoa.jpg} % Used on titlepage only
%\smalllogo{../imagenes/logoaloratxoa.jpg} % Used on top right corner of regular frames
\usepackage{booktabs}
\usepackage[scale=2]{ccicons}

\usepackage{amsmath}
\usepackage{amsfonts}
\usepackage{amssymb}
\usepackage{graphicx}
\usepackage{colortbl}
\usepackage{tikz} 
\usetikzlibrary{matrix}
\usetikzlibrary{calendar,decorations.markings} 
\usetikzlibrary{shapes,positioning}

\usepackage{tkz-tab,tkz-euclide,tkz-fct}
\usetkzobj{all}  
\usepackage{tcolorbox} 
%\usepackage{enumitem} 
\usepackage{tasks} 
\usepackage{asymptote}  
\usepackage{cancel}
\usepackage{xfrac}

\newcommand{\sen}{\mathop{\rm sen}\nolimits}

\newcommand{\tg}{\mathop{\rm tg}\nolimits}
\newcommand{\arcsen}{\mathop{\rm arcsen}\nolimits}
\newcommand{\arctg}{\mathop{\rm arctg}\nolimits}
\newcommand{\g}{{}^\circ}     
        
\newcommand{\R}{\mathbb{R}}
\newcommand{\Z}{\mathbb{Z}}
\newcommand{\N}{\mathbb{N}}
\newcommand{\Q}{\mathbb{Q}}
\newcommand{\I}{\mathbb{I}}
\newcommand{\limite}[2]{\displaystyle \lim_{x \rightarrow #1}{#2}}
\renewcommand{\vector}[1]{\overrightarrow{#1}}

\newtcbox{\resultado}[1][center]{#1,colback=red!5!white,
colframe=red!75!black}

\newtcbox{\resaltado}[1][center]{#1,colback=blue!5!white,
colframe=blue!75!black}
\definecolor{titleColor}{rgb}{0.0, 0.42, 0.24}

\imagen{../imagenes/descartes.jpg}
\title{Números reales}
\subtitle{Matemáticas I}
\author{Departamento de Matemáticas}
\date{ UHEI - IVED}
\begin{document}
%\ECFJD
\titleframe

\begin{frame}
\tableofcontents
\end{frame}

\section{Intervalos y semirrecta}
\subsection{Distancia entre dos puntos}

\begin{frame}[t]{Distancia entre dos puntos}
\begin{alertblock}{Distancia entre dos puntos}
La distancia entre dos puntos $P(x_1,y_1)$ y $Q(x_2,y_2$ es igual al módulo del vector que une estos dos puntos:
\[ d(P,Q)= \left| \vector{PQ} \right| = \sqrt{(x_2-x_1)^2+(y_2-y_1)^2} \]
\end{alertblock}

\pause

\begin{exampleblock}{Ejemplo}
Representar en todas sus formas los siguientes conjuntos:


\begin{enumerate}

\item
  Números menores que seis.
\item
  \(\left\{ x\mathbb{\in R\ /\ }2 \leq x < 9 \right\}\)
\item
  \(\left\{ x\mathbb{\in R\ /\ \  -}5 < x < 8 \right\}\)
\item
  \(\left\{ x\mathbb{\in R\ /\ }|x| \leq 5 \right\}\)
\end{enumerate}
\end{exampleblock}

\pause

\[d(P,Q)=\sqrt{(-2-1)^2+ (1-(-2))^2}= \sqrt{18} \text{u} \]
\end{frame}

\subsection{Distancia de un punto a una recta}
\begin{frame}[t]{Distancia de un punto a una recta}

\begin{alertblock}{Distancia de un punto a una recta}
La distancia del punto $P(x_1,y_1)$ a la recta $r : Ax+By+C=0$ se calcula mediante la siguiente expresión:
\[ d(P,r)=\dfrac{|A\cdot x_1+B\cdot y_1+C|}{\sqrt{A^2+B^2}} \]
\end{alertblock}
\pause
\begin{exampleblock}{Ejemplo}
Hallar la distancia del punto $P(1,5)$ a la recta $r : \dfrac{x}{4}=\dfrac{y-1}{5}$.
\end{exampleblock}
\pause
Escribimos la ecuación de la recta en forma general:
\[ \dfrac{x}{4}=\dfrac{y-1}{5} \Rightarrow 5x=4y-4 \Rightarrow 5x-4y+4=0 \]

\pause
La distancia de $P$ a $r$ es:
\[ d(P,r)=\dfrac{|5\cdot 1 -4\cdot 5 +4 | }{\sqrt{5^2+4^2}}= \dfrac{|-11|}{\sqrt{41}}= \dfrac{11\sqrt{41}}{41} \, \text{u} \]
\end{frame}

\subsection{Distancia entre dos rectas}
\begin{frame}[t]{Distancia entre dos rectas}

\begin{itemize}[<+->]
\item Si las rectas son secantes o coincidentes la distancia entre ambas es cero.
\item Si las rectas son paralelas la distancia entre ambas es igual a la distancia de cualquier punto de una de las rectas a la otra recta.
\end{itemize}

\pause
\begin{exampleblock}{Ejemplo}
Hallar la distancia entre las rectas $r: \dfrac{x}{3}= \dfrac{y-1}{-4}$ y $s:4x+3y + 2 = 0 $. 
\end{exampleblock}

\pause
Hallamos la ecuación general de la recta $r$:
\[ \dfrac{x}{3}=\dfrac{y-1}{-4} \Rightarrow -4x=3y-3 \Rightarrow -4x-3y+3=0 \]
\pause
Estudiamos su posición relativa.
\[ \dfrac{-4}{4}=\dfrac{-3}{3} \neq \dfrac{3}{2} \]
luego son paralelas.

\pause
Tomamos un punto de la recta $r$, por ejemplo, $R(0,1)$
\pause

Hallamos la distancia entre las dos rectas:
\[ d(r,s)=d(R,s)=\dfrac{|4\cdot 0+3\cdot 1 +2|}{\sqrt{4^2+3^2}}= \dfrac{5}{5}= 1 \, u \]
\end{frame}

\section{Ángulo entre dos rectas}
\begin{frame}[t]{Ángulo entre dos rectas}
Se llama ángulo entre dos rectas al menor ángulo que forman las rectas al cortarse. Este ángulo coincide con el ángulo de sus vectores directores.
\pause
\begin{alertblock}{Cálculo a partir de sus vectores directores}
\[ \cos \alpha = \dfrac{\left| \vec{v_r}\cdot \vec{v_s}\right|}{\left| \vec{v_r}\right|\left|\vec{v_s}\right|} \qquad 0 \leq \alpha \leq 90 \]
\end{alertblock}
\pause
\begin{exampleblock}{Ejemplo}
Hallar el ángulo que forman las rectas $\begin{cases} r: 3x-4y=0 \\ s: 2x+2y+3=0 \end{cases}$
\end{exampleblock}
\pause
Hallamos los vectores directores de las dos rectas: $\begin{cases} \vec{v_r}=(4,3) \\ \vec{v_s}=(-2,2) \end{cases}$
\pause

Hallamos el ángulo:
\pause
\[ \cos \alpha = \dfrac{| 4 \cdot(-2) +3 \cdot 2 |}{\sqrt{4^2+3^2}\sqrt{(-2)^2+2^2}}= \dfrac{2}{5\sqrt{8}} \Rightarrow \alpha = 81,87 \g \]

\end{frame}

\begin{frame}[t]{Ángulo entre dos rectas}

\begin{alertblock}{Cálculo a partir de sus pendientes}
\[ \tg \alpha = \left| \dfrac{m_s - m_r}{1+m_s \cdot m_r }\right| \qquad 0 \leq \alpha \leq 90 \]
\end{alertblock}
\pause
\begin{exampleblock}{Ejemplo}
Hallar el ángulo que forman las rectas $\begin{cases} r: 3x-4y=0 \\ s: 2x+2y+3=0 \end{cases}$
\end{exampleblock}
\pause
Hallamos las pendientes de las dos rectas: $\begin{cases} m_r= \dfrac{4}{3} \\ \\ m_s=\dfrac{2}{-2}=-1 \end{cases}$
\pause

Hallamos el ángulo:
\pause
\[ \tg \alpha =\left| \dfrac{-1-\dfrac{4}{3}}{1+(-1)\cdot \dfrac{4}{3}} \right|= 7  \Rightarrow \alpha = 81,87 \g \]

\end{frame}

\section{Puntos simétricos}
\begin{frame}[t]{Punto simétrico respecto a un punto}
\begin{alertblock}{Punto simétrico respecto a otro punto}
Los puntos $A$ y $A'$ son simétricos respecto de otro punto $M$, si $M$ es el punto medio del segmento $\overline{AA'}$.
\end{alertblock}

\pause

\begin{exampleblock}{Ejemplo}
Hallar el punto simétrico de $P(1,2)$ respecto al punto $M(-1,6)$
\end{exampleblock}
\end{frame}


\section{Puntos simétricos}
\begin{frame}[t]{Punto simétrico respecto a una recta}
\begin{alertblock}{Punto simétrico respecto a otro punto}
Los puntos $A$ y $A'$ son simétricos respecto de la recta $r$ si el segmento $\overline{AA'}$ es perpendicular a $r$ y la corta en el punto medio del segmento.
\end{alertblock}
\pause
\begin{exampleblock}{Ejemplo}
Hallar el punto simétrico del punto $P(2,1)$ respecto de la recta $r: x+2y-9=0$
\end{exampleblock}
\end{frame}

\end{document}