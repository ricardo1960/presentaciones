\documentclass[8pt]{beamer}
%\usetheme{CambridgeUS}
%\logo{\includegraphics[scale=0.10]{../imagenes/logoa}}
\usepackage[spanish]{babel}
%\usecolortheme{seahorse}
%\usepackage{beamerthemeblackboard}
%\usepackage{graphics}
%\usecolortheme[RGB={6,138,200}]{structure}
%\usepackage[orientation=landscape, size=custom,
 % width=16, height=9, scale=0.5]{beamerposter}
%\usepackage[utf8]{inputenc}
%\usetheme{metropolis}
%\metroset{titleformat=smallcaps,block=fill}
\setbeamercolor{frametitle}{bg=titleColor,fg= white}
\newcommand{\imagen}[1]{\titlegraphic{\includegraphics[height=\paperheight]{../imagenes/#1}}}
\usetheme[titleformat=smallcaps,block=fill,sectionstyle=style2]{trigon}

% Define logos to use (comment if no logo)
\biglogo{../imagenes/logoa.jpg} % Used on titlepage only
%\smalllogo{../imagenes/logoaloratxoa.jpg} % Used on top right corner of regular frames
\usepackage{booktabs}
\usepackage[scale=2]{ccicons}

\usepackage{amsmath}
\usepackage{amsfonts}
\usepackage{amssymb}
\usepackage{graphicx}
\usepackage{colortbl}
\usepackage{tikz} 
\usetikzlibrary{matrix}
\usetikzlibrary{calendar,decorations.markings} 
\usetikzlibrary{shapes,positioning}

\usepackage{tkz-tab,tkz-euclide,tkz-fct}
\usetkzobj{all}  
\usepackage{tcolorbox} 
%\usepackage{enumitem} 
\usepackage{tasks} 
\usepackage{asymptote}  
\usepackage{cancel}
\usepackage{xfrac}

\newcommand{\sen}{\mathop{\rm sen}\nolimits}

\newcommand{\tg}{\mathop{\rm tg}\nolimits}
\newcommand{\arcsen}{\mathop{\rm arcsen}\nolimits}
\newcommand{\arctg}{\mathop{\rm arctg}\nolimits}
\newcommand{\g}{{}^\circ}     
        
\newcommand{\R}{\mathbb{R}}
\newcommand{\Z}{\mathbb{Z}}
\newcommand{\N}{\mathbb{N}}
\newcommand{\Q}{\mathbb{Q}}
\newcommand{\I}{\mathbb{I}}
\newcommand{\limite}[2]{\displaystyle \lim_{x \rightarrow #1}{#2}}
\renewcommand{\vector}[1]{\overrightarrow{#1}}

\newtcbox{\resultado}[1][center]{#1,colback=red!5!white,
colframe=red!75!black}

\newtcbox{\resaltado}[1][center]{#1,colback=blue!5!white,
colframe=blue!75!black}
\definecolor{titleColor}{rgb}{0.0, 0.42, 0.24}

\imagen{../imagenes/descartes.jpg}
\title{Geometría 1}
\author{Ricardo Mateos}
\institute[UHEI-IVED]{Departamento de Matemáticas \\ UHEI - IVED}
\date{Matemáticas I}
\begin{document}
%\ECFJD
\titleframe

\begin{frame}
\tableofcontents
\end{frame}

\section{Ecuaciones de la recta}

\begin{frame}[t]{Ecuaciones de la recta}
Una recta $r$ está determinada por un punto $A(x_0,y_0)$  y  un vector director $\vec{v}=(v_1,v_2)$, que marca la dirección de la recta, es decir, es  paralelo a la recta. 

Tomando cualquier otro punto $P(x,y)$ de la recta tendremos que el vector $\vector{AP}$ es paralelo al vector director de la recta $(\vec{v})$, es decir, $\vector{AP}=\lambda \vector{v}$.
\begin{center}
\begin{tikzpicture}[scale=0.65]
\tkzInit[xmin=-2,xmax=7, ymin=-1,ymax=6]
	\tkzDrawXY
	\tkzDefPoints{2/3/A, 4/4/P}
	\tkzDefPoints{-1/1.5/B, 7/5.5/C}
	\tkzDefPoints{0/0/O, 2/1/V}
	%\tkzDefPoints{-1/1/B3, -1/0/B4}
	%\tkzDefPoints{0/-2/C1, 3.2/-2/C2}
	%\tkzDefPoints{0/-1/C3, 2.4/-1/C4}
	\tkzDrawSegment[line width=1.5pt](B,C)
	\tkzDrawSegment[line width=1pt,-latex,color=blue](A,P)
	\tkzDrawSegment[line width=1pt,-latex,color=blue](O,V)
	\tkzDrawSegment[line width=1pt,-latex,color=red](O,P)
	\tkzDrawSegment[line width=1pt,-latex,color=red](O,A)
	%\tkzDrawPolygon[line width=2pt](A,C,D,B)
	%\tkzDrawSegments[color=blue](A1,G A2,G)
	%\tkzDrawSegments[color=blue](B1,E B2,A A1,B3)
	%\tkzDrawSegments[color=blue,latex-latex](B1,B2)
	%\tkzDrawSegments[color=blue,latex-latex](B3,B4)
	%\tkzDrawSegments[color=blue](C2,F C1,A A2,C4)
	%\tkzDrawSegments[color=blue,latex-latex](C1,C2)
	%\tkzDrawSegments[color=blue,latex-latex](C3,C4)
	\tkzDrawPoints(A,P)
	\tkzLabelPoint[above](P){$P$}
	\tkzLabelPoint[above](A){$A$}	
	\tkzLabelSegment[above left](A,P){$\lambda \cdot \vec{v}$}
	\tkzLabelSegment[below right](O,V){$\vec{v}$}
	\tkzLabelSegment[left](O,A){$\vector{OA}$}
	\tkzLabelSegment[right](O,P){$\vector{OP}$}
	%\tkzFct[color=red,domain=0:7.3]{(x**2-2*x+13)/12)}
	%\tkzFct[color=red,domain=0:4]{-x}
\end{tikzpicture}
\end{center}



\end{frame}

\begin{frame}{Ecuación vectorial de la recta}

En la figura anterior observamos que:
\[ \vector{OP} =  \vector{OA} +  \vector{AP} \]

De este modo tendremos que:
\[ \vector{OP}=\vector{OA}+\lambda \vec{v} \qquad (\lambda \in \R)\]

La ecuación anterior recibe el nombre de \textbf{ecuación vectorial} de la recta.

\pause
Expresandola en forma de coordenadas tendremos:

\resaltado{$(x,y)= (x_0,y_0)+\lambda (v_1,v_2)$ }
\end{frame}

\begin{frame}[t]{Ecuaciones paramétricas y continua de la recta}

Si en la ecuación anterior igualamos las componentes de ambos miembros obtendremos las \textbf{ecuaciones paramétricas} de la recta.
\resaltado{$\begin{cases} x= x_0 + \lambda \cdot v_1 \\ y= y_0 + \lambda \cdot v_2
 \end{cases} $}
\pause
 Si despejamos $\lambda$ en cada una de las ecuaciones paramétricas 
 \[ \lambda= \dfrac{x-x_0}{v_1} \qquad \lambda= \dfrac{y-y_0}{v_2} \]
 e igualamos las dos expresiones obtendremos la \textbf{ecuación continua} de la recta:
 \resaltado{$\dfrac{x-x_0}{v_1} = \dfrac{y-y_0}{v_2}$}
\end{frame}

\begin{frame}[t]{Ecuación general de la recta}

Eliminando los denominadores y pasando todos los términos a un lado de la expresión obtenemos:
\[ v_2(x-x_0) = v_1(y-y_0) \] 
\[ v_2 \cdot x- v_2 \cdot x_0 - v_1\cdot y+ v_1 \cdot y_0 =0 \] 
Tomando $A=v_2$, $B=-v_1$ y $C=v_1 \cdot y_0 - v_2 \cdot x_0$ obtenemos la \textbf{ecuación general } de la recta:
\resaltado{$Ax+By+C=0 $ }
\end{frame}
\begin{frame}[t]{Ecuaciones de la recta}
\begin{exampleblock}{Ejemplo}
Hallar las ecuaciones vectorial, paramétricas, continua y general de la recta que pasa por el punto $A(2,-1)$ y tiene como dirección la del vector $\vec{v}=(1,-2)$
\end{exampleblock}
\end{frame}

\begin{frame}[t]{Inclinación y pendiente de una recta}

La \textbf{inclinación de una recta} es el ángulo que forma con la dirección positiva del eje $OX$.

La \textbf{pendiente de una recta} el la tangente de su inclinación, es decir, la tangente del ángulo que forma con la dirección positiva del eje $OX$.

\begin{columns}
\column{0.5\textwidth}
\begin{center}
\begin{tikzpicture}[scale=0.35]
\tkzInit[xmin=-3,xmax=4, ymin=-4,ymax=6]
	\tkzDrawXY
	\tkzDefPoints{-2/-3/A,2/5/P}
	
	%\tkzDrawSegment[line width=1.5pt](B,C)
	\tkzDrawSegment[line width=1pt](A,P)
	\tkzDefPoints{-0.5/0/O,2/0/B,0/1/C}
	\tkzMarkAngle[size=1.5cm,mark=none](B,O,P)
	%\tkzDrawSegment[line width=1pt,-latex,color=blue](O,V)
	%\tkzDrawSegment[line width=1pt,-latex,color=red](O,P)
	%\tkzDrawSegment[line width=1pt,-latex,color=red](O,A)
	%\tkzDrawPolygon[line width=2pt](A,C,D,B)
	%\tkzDrawSegments[color=blue](A1,G A2,G)
	%\tkzDrawSegments[color=blue](B1,E B2,A A1,B3)
	%\tkzDrawSegments[color=blue,latex-latex](B1,B2)
	%\tkzDrawSegments[color=blue,latex-latex](B3,B4)
	%\tkzDrawSegments[color=blue](C2,F C1,A A2,C4)
	%\tkzDrawSegments[color=blue,latex-latex](C1,C2)
	%\tkzDrawSegments[color=blue,latex-latex](C3,C4)
	%\tkzDrawPoints(A,P)
	%\tkzLabelPoint[above](P){$P$}
	%\tkzLabelPoint[above](A){$A$}	
	%\tkzLabelSegment[above left](A,P){$\lambda \cdot \vec{v}$}
	%\tkzLabelSegment[below right](O,V){$\vec{v}$}
	%\tkzLabelSegment[left](O,A){$\vector{OA}$}
	%\tkzLabelSegment[right](O,P){$\vector{OP}$}
	%\tkzFct[color=red,domain=0:7.3]{(x**2-2*x+13)/12)}
	%\tkzFct[color=red,domain=0:4]{-x}
\end{tikzpicture}
\end{center}

\column{0.5\textwidth}
\begin{center}
\begin{tikzpicture}[scale=0.35]
\tkzInit[xmin=-3,xmax=4, ymin=-4,ymax=6]
	\tkzDrawXY
	\tkzDefPoints{-2/-3/A,2/5/P}
	\tkzDefPoints{-0.5/0/O,2/0/B,0/1/C}
	%\tkzDrawSegment[line width=1.5pt](B,C)
	\tkzDrawSegment[line width=1pt](A,P)
	\tkzMarkAngle[size=1.5cm,mark=none](B,O,P)
	%\tkzDrawSegment[line width=1pt,-latex,color=blue](O,V)
	%\tkzDrawSegment[line width=1pt,-latex,color=red](O,P)
	%\tkzDrawSegment[line width=1pt,-latex,color=red](O,A)
	%\tkzDrawPolygon[line width=2pt](A,C,D,B)
	%\tkzDrawSegments[color=blue](A1,G A2,G)
	%\tkzDrawSegments[color=blue](B1,E B2,A A1,B3)
	%\tkzDrawSegments[color=blue,latex-latex](B1,B2)
	%\tkzDrawSegments[color=blue,latex-latex](B3,B4)
	%\tkzDrawSegments[color=blue](C2,F C1,A A2,C4)
	%\tkzDrawSegments[color=blue,latex-latex](C1,C2)
	%\tkzDrawSegments[color=blue,latex-latex](C3,C4)
	%\tkzDrawPoints(A,P)
	%\tkzLabelPoint[above](P){$P$}
	%\tkzLabelPoint[above](A){$A$}	
	%\tkzLabelSegment[above left](A,P){$\lambda \cdot \vec{v}$}
	%\tkzLabelSegment[below right](O,V){$\vec{v}$}
	%\tkzLabelSegment[left](O,A){$\vector{OA}$}
	%\tkzLabelSegment[right](O,P){$\vector{OP}$}
	%\tkzFct[color=red,domain=0:7.3]{(x**2-2*x+13)/12)}
	%\tkzFct[color=red,domain=0:4]{-x}
\end{tikzpicture}
\end{center}

\end{columns}

\end{frame}
\end{document}