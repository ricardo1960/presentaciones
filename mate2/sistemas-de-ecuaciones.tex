\documentclass[9pt]{beamer}
\usetheme{metropolis}
%\usetheme{Warsaw}
\metroset{titleformat=smallcaps,block=fill}
\usecolortheme{seahorse}
%\usepackage[utf8]{inputenc}
\usepackage[spanish]{babel}
\usepackage{amsmath}
\usepackage{amsfonts}
\usepackage{amssymb}
\usepackage{graphicx}
\usepackage{tikz} 
\usetikzlibrary{matrix}
\usetikzlibrary{calendar,decorations.markings} 
\usetikzlibrary{shapes,positioning}
\usepackage{tcolorbox}
\usepackage{tkz-tab,tkz-euclide,tkz-fct}

\usetkzobj{all} 
\usepackage{polynom}
%\usepackage{enumitem}
\usepackage{extarrows}
%Nuevos entornos
\newenvironment{matrizgauss}{\left( \begin{array}{ccc|ccc}}{\end{array} \right)}
\newenvironment{matrizampliada}{\left( \begin{array}{ccc|c}}{\end{array} \right)}
\newenvironment{sistematres}{\left\lbrace \begin{array}{rrrrrrr}}{\end{array} \right.}
\newenvironment{sistemados}{\left\lbrace \begin{array}{rrrr}}{\end{array} \right.}
\newenvironment{adjunto}{\[ \begin{array}{lll}}{\end{array} \]}
\newenvironment{sistema}{\left\lbrace \begin{array}{rrrrrr}}{\end{array} \right. }

%nuevos comandos

\newcommand{\limite}[2]{\displaystyle \lim_{x \rightarrow #1}{#2}}
\newcommand{\limiteserie}[1]{\displaystyle \lim_{n \rightarrow +\infty}{#1}}
\newcommand{\matrizdos}[4]{\begin{pmatrix} #1 & #2 \\  #3 & #4  \end{pmatrix}}
\newcommand{\determinantedos}[4]{\begin{vmatrix} #1 & #2 \\ #3 & #4  \end{vmatrix}}
\newcommand{\matriztres}[9]{\begin{pmatrix} #1 & #2 & #3 \\ #4 & #5 & #6 \\ #7 & #8 & #9 \end{pmatrix}}
\newcommand{\determinantetres}[9]{\begin{vmatrix} #1 & #2 & #3 \\ #4 & #5 & #6 \\ #7 & #8 & #9 \end{vmatrix}}
\newcommand{\filasistematres}[6]{#1 & #2 & #3 &#4 & #5 & = & #6}
\newcommand{\integral}[1]{\displaystyle \int #1}
\newcommand{\intdef}[3]{\displaystyle \int_#1^#2 #3}


%%Comandos para abreviar los entornos%%

\newcommand{\problema}[1]{\begin{ejer} #1  \end{ejer}}
\newcommand{\sol}[1]{\begin{solu} #1 \end{solu}}

\newtcbox{\resultado}[1][center]{#1,colback=red!5!white,
colframe=red!75!black}

\newtcbox{\resaltado}[1][center]{#1,colback=blue!5!white,
colframe=blue!75!black}

\newenvironment{gaussjordandos}{\left( \begin{array}{cc|cc}}{\end{array}\right)}
\newenvironment{gaussjordantres}{\left( \begin{array}{ccc|ccc}}{\end{array}\right)}


\newcommand{\R}{\mathbb{R}}

\author{Departamento de Matemáticas}
\title{Sistemas de Ecuaciones}
%\subtitle{Definición y cálculo}
%\setbeamercovered{transparent} 
%\setbeamertemplate{navigation symbols}{} 
%\logo{\includegraphics[scale=0.05]{../../images/logoa.jpg}} 
%\institute{UHEI - IVED} 
\date{\includegraphics[scale=0.15]{imagenes/logoa.jpg}} 
%\subject{} 
\begin{document}

\begin{frame}
\titlepage
\end{frame}

\begin{frame}
\tableofcontents
\end{frame}

\section{Sistemas de ecuaciones}

\section{Sistemas de ecuaciones}
\begin{frame}{Sistemas de ecuaciones}
\begin{alertblock}{Sistemas de ecuaciones lineales}
Un sistema de ecuaciones es un conjunto de $m$ ecuaciones con $n$ incógnitas que se cumplen para unos valores de estas incógnitas.

\[ \left\lbrace \begin{array}{ccccccccc}
a_{11} x_1 & + & a_{12}x_2 & + & \cdots & + & a_{1n} & = & b_1 \\
a_{21} x_1 & + & a_{22}x_2 &+ & \cdots & + & a_{2n} & = & b_2 \\
\cdots & & \cdots & &  \cdots & & \cdots & & \cdots \\
a_{m1} x_1 & + & a_{m2}x_2 & +& \cdots & + & a_{mn} & = & b_m \\
\end{array} \right. \]
\end{alertblock}
\pause

Siendo:

\begin{itemize}[<+->]
\item Los \textbf{coeficientes del sistema} los números reales $a_{ij}$
\item Los  \textbf{términos independientes} los números reales $b_i$
\item Las  \textbf{incógnitas} los términos $x_j$
\end{itemize}
\end{frame}
\begin{frame}{Sistemas de ecuaciones}
La solución o soluciones del sistema son los valores de las incógnitas para los que se cumplen todas las ecuaciones.
\pause

Resolver un sistema de ecuaciones es hallar las soluciones.
\pause

Según la solución del sistema clasificamos el sistema de la siguiente forma:
\pause

\begin{itemize}[<+->]
\item  \textbf{Compatibles:} si tiene solución. Estos pueden ser:
\begin{itemize}
\item  \textbf{Determinado:} si tiene una única solución.
\item  \textbf{Indeterminados:} si tiene infinitas soluciones.
\end{itemize}
\item  \textbf{Incompatibles:} si no tiene solución.
\end{itemize}
\pause
Dos sistemas son equivalentes cuando tienen las mismas soluciones.
\end{frame}



\section{Resolución de sistemas de ecuaciones}

\begin{frame}{Métodos de resolución de sistemas de ecuaciones}
Para resolver sistemas de ecuaciones en este curso emplearemos tres métodos:
\pause

\begin{enumerate}[<+->]
\item Método matricial.
\item Método de Gauss.
\item Regla de Cramer.
\end{enumerate}
\end{frame}

\subsection{Método matricial}
\begin{frame}{Método matricial}
\begin{alertblock}{Expresión de un sistema con matrices}
Dado un sistema de ecuaciones:
\[ \left\lbrace \begin{array}{ccccccccc}
a_{11} x_1 & + & a_{12}x_2 & + & \cdots & + & a_{1n} x_n& = & b_1 \\
a_{21} x_1 & + & a_{22}x_2 &+ & \cdots & + & a_{2n} x_n& = & b_2 \\
\cdots & & \cdots & &  \cdots & & \cdots & & \cdots \\
a_{m1} x_1 & + & a_{m2}x_2 & +& \cdots & + & a_{mn} x_n& = & b_m \\

\end{array} \right. \]
\pause
se puede escribir matricialmente de la siguiente forma:
\resaltado{$ A\cdot X = B $}
\pause
siendo:
\[ A= \begin{pmatrix}
a_{11} & a_{12} & \cdots & a_{1n} \\
a_{21} & a_{22} & \cdots & a_{2n} \\
\cdots & \cdots & \cdots & \cdots \\
a_{m1} & a_{m2} & \cdots & a_{mn} \\
\end{pmatrix} \qquad
X= \begin{pmatrix}
x_1 \\ x_ 2 \\ \cdots \\ x_n
\end{pmatrix}
\qquad
B=\begin{pmatrix}
b_1 \\ b_ 2 \\ \cdots \\ b_m
\end{pmatrix} \]
\end{alertblock}
\end{frame}

\begin{frame}{Método matricial}
Si la matriz $A$ es regular, es decir, si existe la matriz inversa de $A$ para resolver el sistema resolvemos la ecuación matricial $A\cdot X = B$ utilizando la matriz inversa de $A$ de la siguiente forma $X=A^{-1}\cdot B$
\end{frame}

\begin{frame}{Método de la matriz inversa}
\begin{exampleblock}{Ejemplo resolviendo con matriz inversa}
Resolver el siguiente sistema de ecuaciones utilizando el método de la matriz inversa:

\[\begin{cases}
2x+4y+5z=1 \\
x+3y+3z=-1  \\
3x+3y+2z=2
\end{cases}
 \]
\end{exampleblock}
\pause

Primero hallamos la matriz inversa de $A$ si existe.

\[ |A|= \determinantetres{2}{4}{5}{1}{3}{3}{3}{3}{2} = (12+36+15)-(45+18+8)=63-71=-8 \]
\pause

El determinante es distinto de cero, por lo tanto, existe la matriz inversa y el sistema tendrá solución única.
\end{frame}

\begin{frame}{Método matriz inversa}
Hallamos la matriz inversa que es:


\[ A^{-1}=\matriztres{\frac{3}{8}}{-\frac{7}{8}}{\frac{3}{8}}{-\frac{7}{8}}{\frac{11}{8}}{\frac{1}{8}}{\frac{3}{4}}{-\frac{3}{4}}{-\frac{1}{4}} \]
\pause

Luego:

\[ \begin{pmatrix}
x \\ y \\ z
\end{pmatrix} = 
\matriztres{\frac{3}{8}}{-\frac{7}{8}}{\frac{3}{8}}{-\frac{7}{8}}{\frac{11}{8}}{\frac{1}{8}}{\frac{3}{4}}{-\frac{3}{4}}{-\frac{1}{4}} 
\begin{pmatrix}
1 \\ -1 \\ 2 
\end{pmatrix}= 
\begin{pmatrix}
2 \\ -2 \\ 1
\end{pmatrix} \]
\pause


La solución del sistema es:

\[ \left. \begin{array}{rrr} x&=& 2  \\ y&=& -2 \\ z&=&1 
\end{array} \right\rbrace \]

\end{frame}


\subsection{Método de Gauss}

\begin{frame}{Método de Gauss}
\begin{alertblock}{Sistemas escalonados}
Un sistema es escalonado cuando cualquier ecuación tiene menos incógnitas que el anterior.

\end{alertblock}
\pause

\begin{alertblock}{Método de Gauss}
El método de Gauss consiste en escalonar el sistema. 
\pause

Para escalonar el sistema siguiendo el método de Gauss podemos hacer las siguientes transformaciones en el sistema:
\pause
\begin{enumerate}[<+->]
\item Cambiar el orden de las ecuaciones.
\item Multiplicar los dos miembros de alguna de las ecuaciones por un número distinto de cero.
\item Sumar a una ecuación otra ecuación multiplicada por un número.
\end{enumerate}
\end{alertblock}
\pause

Vamos a ver como realizar el método de Gauss por medio de unos ejemplos.
\end{frame}

\begin{frame}{Método de Gauss}
\begin{exampleblock}{Método de Gauss}
Resuelve el siguiente sistema de ecuaciones mediante el metodo de Gauss.
\[ 
\begin{sistematres} 
	3x&+&2y&-&z&=&4 \\
	x&+&y&+&z&=&3 \\
	2x&-&y&-&z&=&0 \\
\end{sistematres}
\]
\end{exampleblock}
\pause
\onslide*<2>{
Escribimos la matriz ampliada del sistema de ecuaciones:

\[
\begin{matrizampliada}
	3 & 2 & -1 & 4 \\
	1 & 1 & 1 & 3 \\
	2 & -1 & -1 & 0 
\end{matrizampliada} 	
\]}
\pause
\onslide*<3->{
Vamos escalonando el sistema

$
\begin{matrizampliada}
	3 & 2 & -1 & 4 \\
	1 & 1 & 1 & 3 \\
	2 & -1 & -1 & 0 
\end{matrizampliada}$
\pause
$\xLongrightarrow[f_3=3f_3-2f_1]{f_2=3f_2-f_1}$
\pause
$\begin{matrizampliada}
	3 & 2 & -1 & 4 \\
	0 & 1 & 4 & 5 \\
	0 & -7 & -1 & -8 
\end{matrizampliada}$
\pause
$\begin{matrizampliada}
	3 & 2 & -1 & 4 \\
	0 & 1 & 4 & 5 \\
	0 & -7 & -1 & -8 
\end{matrizampliada}$
\pause
$\xLongrightarrow{f_3=f_3+7f_2}$
\pause
$\begin{matrizampliada}
	3 & 2 & -1 & 4 \\
	0 & 1 & 4 & 5 \\
	0 & 0 & 27 & 27 
\end{matrizampliada}
$}
\pause

El sistema es compatible determinado.
\end{frame}

\begin{frame}{Método de Gauss}

Escribimos de nuevo el sistema ya escalonado:
\pause

\[ 
	\begin{sistematres}
		3x&+&2y&-&z&=&4 \\
		&&y&+&4z&=&5 \\
		&&&&27z&=&27 \\
	\end{sistematres}
\]

\pause
Resolviendo el sistema de abajo hacia arriba tenemos:

\[ 27z=27 \Rightarrow z=1 \]
\pause
\[ y+4z=5 \Rightarrow y+4=5 \Rightarrow y=5-4 \Rightarrow y=1 \]
\pause
\[ 3x+2y-z=4 \Rightarrow 3x+2-1=4 \Rightarrow 3x=4-2+1 \Rightarrow 3x=3 \Rightarrow x=1 \]
\pause

Luego la solución es:

\[ (x,y,z)=(1,1,1) \] 	
\end{frame}

\begin{frame}	

\begin{exampleblock}{Método de Gauss}
Resuelve el siguiente sistema de ecuaciones mediante el metodo de Gauss.
\[ 
\begin{sistematres} 
	2x&+&y&-&z&=&4 \\
	x&-&y&+&z&=&5 \\
	5x&-&2y&+&2z&=&19 \\
\end{sistematres}
\]	
\end{exampleblock}

\pause
Escribimos la matriz ampliada del sistema de ecuaciones y escalonamos el sistema:

Hacemos las siguientes transformaciones:
\pause

$\begin{matrizampliada}
	2 & 1 & -1 & 4 \\
	1 & -1 & 1 & 5 \\
	5 & -2 & 2 & 19 
\end{matrizampliada}$
\pause
$\xLongrightarrow{f_1\leftrightarrow f_2}$
\pause
$\begin{matrizampliada}
	1 & -1 & 1 & 5 \\
	2 & 1 & -1 & 4 \\
	5 & -2 & 2 & 19 
\end{matrizampliada}$
\pause
$\xLongrightarrow[f_3=f_3-5f_1]{f_2=f_2-2f_1}$

\pause
$\begin{matrizampliada}
	1 & -1 & 1 & 5 \\
	0 & 3 & -3 & -6 \\
	0 & 3 & -3 & -6 
\end{matrizampliada}$
\pause
$\xLongrightarrow{f_3=f_3-f_2}$
\pause
$\begin{matrizampliada}
	1 & -1 & 1 & 5 \\
	0 & 3 & -3 & -6 \\
	0 & 0 & 0 & 0 
\end{matrizampliada}$
\pause

Eliminamos la tercera fila por ser todos sus elementos cero, con lo cual nos quedan dos filas y tres ecuaciones. Por lo tanto, el sistema es compatible indeterminado.
\end{frame}
\begin{frame}{Método de Gauss}

Escribimos de nuevo el sistema ya escalonado eliminando la tercera ecuación ya que todos sus elementos son cero.:

\[ 
	\begin{sistematres}
		x&-&y&+&z&=&5 \\
		&&3y&-&3z&=& -6\\
	\end{sistematres}
\]

\pause
Resolviendo el sistema de abajo hacia arriba tenemos:
\pause
\[ 3y-3z = -6  \Rightarrow 3y=-6+3z \Rightarrow y=-2+z  \]
\pause
\[ x-y+z=5 \Rightarrow x+2-z+z=5 \Rightarrow x=5-2+z-z \Rightarrow x=3 \]
\pause
Haciendo $z=\lambda $ la solución es:
\pause

\[ (x,y,z)=(3,-2+\lambda ,\lambda ) \] 	

\end{frame}


\subsection{Método de Cramer}
\begin{frame}{Regla de Cramer}
\begin{alertblock}{Regla de Cramer}
El método de Cramer se puede utilizar sólo en el caso de que el sistema sea cuadrado, es decir, el número de incógnitas coincida con el número de ecuaciones y el determinante de la matriz de los coeficientes sea distinto de cero.

En este caso el valor de cada incógnita viene dado por la expresión: $x_i=\dfrac{|A_i|}{|A|}$

donde $|A_i|$ representa el determinante que resulta de sustituir en el determinante de la matriz de los coeficientes la columna correspondiente a esa incógnita por los términos independientes.
\end{alertblock}
\end{frame}

\begin{frame}{Regla de Cramer}
\begin{exampleblock}{Regla de Cramer}
Resolver el siguiente sistema de ecuaciones utilizando el método de Cramer: 
\[ \begin{cases}
3x-2y-z=11 \\
x+2y-2z=7 \\
-3x-y+2z=-13
\end{cases} \]
\end{exampleblock}

\pause
Primero calculamos el valor del determinante de la matriz de los coeficientes que llamaremos $|A|$
\pause

\[ |A| =\begin{vmatrix}
3&	-2 & -1 \\
1 & 2 & -2 \\
-3&-1 & 2
\end{vmatrix}=(12-12+1)-(6+6-4)=1-8=-7 \neq 0 \]
por lo tanto cumple las condiciones para  resolverlo por el método de Cramer.
\end{frame}

\begin{frame}{Regla de Cramer}

Calculamos los determinantes de las incógnitas y luego hallamos el valor de las incógnitas.
\pause
$ |A_x|= \begin{vmatrix}
\color{red} 11 & -2 & -1 \\
\color{red} 7 & 2 & -2 \\
\color{red} -13 & -1 & 2
\end{vmatrix}= (44-52+7)-(13+22-14)=-1-20=-21 $

\pause
$ |A_y|=\begin{vmatrix}
3 & \color{red} 11 & -1 \\
1 & \color{red} 7 & -2 \\
-3 & \color{red} -13 & 2
\end{vmatrix}=(42+66+13)-(21+78+22)=121-121=0 $

\pause
$ |A_z|= \begin{vmatrix}
3 & -2 & \color{red} 11 \\
1 & 2 & \color{red} 7 \\
-3 & -1 & \color{red} -13
\end{vmatrix}=(-78+42-11)-(-66-21+26)=-47+61=14 $

\pause
Luego: 
\[ x=\dfrac{-21}{-7}=3 \qquad y=\dfrac{0}{-7}=0 \qquad z=\dfrac{14}{-7}=-2 \]
\end{frame}

\subsection{Generalización de la regla de Cramer}


\section{Teorema de Rouché-Frobenius}

\begin{frame}{Teorema de Rouché-Frobenius}
\begin{alertblock}{Teorema de Rouché-Frobenius}
Un sistema de $m$ ecuaciones lineales con $n$ incógnitas es compatible (tiene solución) si y solo si el rango de la matriz de los coeficientes coincide con el rango de la matriz ampliada.

El sistema es compatible $\Leftrightarrow  \text{rango}(A)=\text{rango}(A^*)$

\end{alertblock}
\pause
\begin{alertblock}{Consecuencias del teorema de Rouché-Fröbenius}

\begin{itemize}[<+->]
\item Si $ \text{rango}(A) \neq \text{rango}(A^*)$ entonces el sistema es incompatible.
\item Si $ \text{rango}(A)=\text{rango}(A^*)$ entonces el sistema es compatible.
\begin{itemize}[<+->]
\item Si $\text{rango}(A)=\text{rango}(A^*)=n$ entonces el sistema de determinado.
\item Si $\text{rango}(A)=\text{rango}(A^*)<n$ entonces el sistema es indeterminado.

\end{itemize}
\end{itemize}

\end{alertblock}
\end{frame}

\begin{frame}{Teorema de Rouché-Frobenius}
\begin{exampleblock}{Teorema de Rouché-Frobenius}
Resuelve el sistema de ecuaciones:
$\begin{sistematres}
x & &&+&z&=&2 \\
-x&+&y&+&2z&=&0 \\
-x&+&2y&+&5z&=&2
\end{sistematres} $
\end{exampleblock}
\pause
Escribimos la matriz de los coeficientes y la matriz ampliada del sistema:

$A=\begin{pmatrix}
1 & 0 & 1 \\
-1 & 1 & 2 \\
-1 & 2 & 5 
\end{pmatrix}$ $A^*=\begin{pmatrix}
1 & 0 & 1 & 2 \\
-1 & 1 & 2 & 0 \\
-1 & 2 & 5 & 2
\end{pmatrix}$
\pause
$\left. \begin{array}{l}
\text{Estudiamos el rango de la matriz } A:\\
\begin{vmatrix}
1 & 0 \\
-1 & 1
\end{vmatrix}=1-0=1 \neq 0 
\pause
\xLongrightarrow{\text{Orlamos}}
\pause
\begin{vmatrix}
1 & 0 & 1 \\
-1 & 1 & 2 \\
-1 & 2 & 5 
\end{vmatrix}=0 
\pause
 \Rightarrow \text{rg}(A)=2.\\
\pause
\text{Estudiamos el rango de la matriz} A^*\\

\begin{vmatrix}
1 & 0 \\
-1 & 1
\end{vmatrix}=1-0=1 \neq 0 
\xLongrightarrow{\text{Orlamos}}
\pause
\begin{vmatrix}
1 & 0 & 2 \\
-1 & 1 & 0 \\
-1 & 2 & 2 
\end{vmatrix}=0 
\Rightarrow \text{rg}(A^*)=2
\end{array} \right\rbrace \Rightarrow $ Sistema Compatible Indeterminado.

\end{frame}

\begin{frame}
Eliminamos la tercera ecuación y pasamos la $z$ al otro miembro:

$\begin{sistematres}
x&&&=2-z \\
-x&+&y&=&-2z
\end{sistematres} $

Resolvemos este sistema:

$x=2-z$

$-2+z+y=-2z \rightarrow y=2-3z$

Llamando a $z=\lambda$ obtenemos las soluciones del sistema:

\resultado{$ (x,y,z)=(2-\lambda, 2-3\lambda, \lambda) $}

\end{frame}

\begin{frame}{Discusión sistema de ecuaciones con parámetros}
\begin{exampleblock}{Discusión sistema de ecuaciones con parámetros}
Discutir, y resolver cuando sea posible, el sistema de ecuaciones lineales según los valores del parámetro $m$

\[ \left\lbrace \begin{array}{rcl}
mx+y & = & 1 \\
x+my & = & m \\
2mx+2y & = & m+1
\end{array} \right. \]
\end{exampleblock}
\pause

Escribimos la matriz de los coeficientes y la matriz ampliada.

\[ A=\begin{pmatrix}
m & 1 \\
1 & m \\
2m & 2
\end{pmatrix} \qquad A^*=\begin{pmatrix}
m & 1 & 1 \\
1 & m & m \\
2m & 2 & m+1
\end{pmatrix} \]
\pause
Veamos los posibles rangos de las matrices y a partir de estos datos iniciamos el estudio.
$\text{rango} (A) \leq 2 \qquad \text{rango} (A^*) \leq 3 $
\pause
\end{frame}

\begin{frame}{Discusión sistema de ecuaciones con parámetros}
Empezamos estudiando el rango de $A^*$ ya que si el rango de $A^*$ es 3, el rango de $A$ sería 2 y el sistema es incompatible.
\pause

$ \begin{vmatrix}
m & 1 & 1 \\
1 & m & m \\
2m & 2 & m+1
\end{vmatrix}= m^3-m^2-m+1 $
\pause

Igualamos a cero y resolvemos: $m^3-m^2-m+1= 0 \Rightarrow \begin{cases} m=1 \\ m=-1 \end{cases}$
\pause

Por lo tanto, $\forall m \in \R - \{\pm 1\} \left. \begin{array}{l} \text{rango} (M)=3 \\  \text{rango} (A)= 2 \end{array} \right\rbrace \Rightarrow$ Sistema Incompatible.

\pause
Ahora estudiamos los casos en que $m=-1$ o $m=1$
\end{frame}

\begin{frame}{Discusión sistema de ecuaciones con parámetros}
Para $m=-1 \qquad A=\begin{pmatrix}
-1 & 1  \\
1 & -1  \\
-2 & 2  
\end{pmatrix} 
\qquad A^*=\begin{pmatrix}
-1 & 1 & 1 \\
1 & -1 & -1 \\
-2 & 2 & 0 
\end{pmatrix}$
\pause

Estudiamos el rango de $A$.
\pause

Las tres filas son proporcionales, luego $\text{rango}(A)=1$

\pause
Estudiamos el rango de $A^*$
\pause

$\begin{vmatrix}
-1 & 1 \\
-2 & 0
\end{vmatrix}= 2 \neq 0 \Rightarrow \text{rango}(A^*)=2 $
\pause

Por lo tanto, para $m=-1$ Sistema Incompatible.
\end{frame}

\begin{frame}{Discusión sistema de ecuaciones con parámetros}
Para $m=1 \qquad A=\begin{pmatrix}
1 & 1  \\
1 & 1  \\
2 & 2  
\end{pmatrix} 
\qquad A^*=\begin{pmatrix}
1 & 1 & 1 \\
1 & 1 & 1 \\
2 & 2 & 2 
\end{pmatrix}$.
\pause

Estudiamos el rango de $A$. 
\pause

Las tres filas son proporcionales, luego $\text{rango}(A)=1$
\pause

Estudiamos el rango de $A^*$
\pause

Las tres filas son proporcionales, luego $\text{rango}(A^*)=1$
\pause

Por lo tanto, para $m=1$ Sistema Compatible Indeterminado.
\end{frame}

\begin{frame}{Discusión sistema de ecuaciones con parámetros}
Resolvemos en este último caso.
\pause
Como los rangos son 1, nos quedamos con una sola ecuación.

$x+y=1 \Rightarrow \left\lbrace \begin{array}{l}
x= 1- \lambda \\
y= \lambda 
\end{array} \right.$
\pause

\end{frame}

\begin{frame}
\begin{exampleblock}{Ejemplo}
Discutir la compatibilidad del siguiente sistema de ecuaciones:
\[
S=\begin{sistematres}
	x & + & ay & + & z & = & 1 \\
	2x & + & y & - & z & = & 1 \\
	3x & + & y & + & az & = 2 
\end{sistematres}
\]
en función del parámetro $a$.
\end{exampleblock}

\pause
\textbf{Solución:}

Formamos la matriz de los coeficientes y la matriz ampliada del sistema:
\[
A=\begin{pmatrix}
	1 & a & 1 \\
	2 & 1 & -1 \\
	3 & 1 & a
\end{pmatrix}
\qquad
A^{*}=\begin{pmatrix}
	1 & 1 & 1 & 1 \\
	2 & 1 & -1 & 1 \\
	3 & 1 & a & 2
\end{pmatrix} 
\]

\pause
Calculamos el determinante de la matriz $A$. 

\[ \begin{vmatrix}
	1 & a & 1 \\
	2 & 1 & -1 \\
	3 & 1 & a
\end{vmatrix}= a-3a+2-3-2a^2+1=-2a^2-2a
\]
\end{frame}

\begin{frame}
Si el valor del determinante es distinto de cero el rango de $A$ será $3$, en caso contrario será menor que $3$.

\pause
Igualamos a cero:

\[ -2a^2-2a=0 \rightarrow \left\lbrace \begin{array}{l} a=0 \\ a=-1 \end{array} \right. \]
\pause

Ahora discutimos el sistema según los valores de $a$.

Si $a\neq 0 \wedge a \neq -2\rightarrow rg(A)=3$. 

\pause
Sabemos que $rg(A^*)\geq rg(A) =3$. Como la matriz $A^*$ solo tiene $3$ filas $rg(A^*)\leq 3$, por lo tanto, $rg(A^*)=3$.

\pause
En consecuencia: Si $a \neq 0  \wedge a \neq -2 \rightarrow rg(A)=rg(A^*)=3=\text{número de incógnitas} \rightarrow $ Sistema Compatible Determinado.
\end{frame}

\begin{frame}
Si $a=0$ 

\[ A=\begin{pmatrix}
	1 & 0 & 1 \\
	2 & 1 & -1 \\
	3 & 1 & 0
\end{pmatrix}
\qquad
M=\begin{pmatrix}
	1 & 0 & 1 & 1 \\
	2 & 1 & -1 & 1 \\
	3 & 1 & 0 & 2
\end{pmatrix} 
\]

\pause
Calculamos el rango de $A$.

\[ \begin{vmatrix}
	 1 & 0 \\
 	2 & 1
\end{vmatrix}= 1-0=1 \neq 0 \rightarrow rg(A)=2\]

\pause
Estudiamos el rango de $A^*$

\[ \begin{vmatrix}
	1 & 0& 1 \\
	2 & 1 & 1 \\
	3 & 1 & 2
\end{vmatrix}= 2+0+2-3-1-0=0 \rightarrow rg(A^*)=2
\]

\pause
En este caso $r(A)=rg(A^*)=2 < \text{número de incognitas} \rightarrow$Sistema Compatible Indeterminado
\end{frame}

\begin{frame} 

Si $a=-1$ 

\[ A=\begin{pmatrix}
	1 & -1 & 1 \\
	2 & 1 & -1 \\
	3 & 1 & -1
\end{pmatrix}
\qquad
A^*=\begin{pmatrix}
	1 & -1 & 1 & 1 \\
	2 & 1 & -1 & 1 \\
	3 & 1 & -1 & 2
\end{pmatrix} 
\]

\pause
Calculamos el rango de $A$.

\[ \begin{vmatrix}
	 1 & -1 \\
 	2 & 1
\end{vmatrix}= 1+2=3 \neq 0 \rightarrow rg(A)=2\]

\pause
Estudiamos el rango de $A^*$

\[ \begin{vmatrix}
	1 & -1 & 1 \\
	2 & 1 & 1 \\
	3 & 1 & 2
\end{vmatrix}= 2-3+2-3-1+4=1 \rightarrow rg(A^*)=3
\]

En este caso $r(A)=2 \neq rg(A^*)=3 \rightarrow \text{Sistema Incompatible}$

\pause

Resumiendo 
$\left\lbrace 
\begin{array}{lll} 
a\neq 1 \wedge a \neq -2 & rg(A)=rg(A^*)=3=\text{nº incógnitas} & \text{S.C.D.} \\
a=0 & rg(A)=rg(A^*)=2 <\text{nº incógnitas} & \text{S.C.I.} \\
a=-1 & rg(A)=2 \quad rg(A^*)=3 & \text{S.I.}
\end{array}
\right.$
\end{frame}

\begin{frame}
\begin{exampleblock}
Dado el sistema de ecuaciones:
\[
S=\begin{sistematres}
	x & + & y & + & z & = & 2 \\
	2x & + & y & & & = & 0 \\
	3x & + & 2y & + & az & = &2a 
\end{sistematres}
\]

\begin{enumerate}
\item Demostrar que es compatible para todos los valores de $a$.
\item Resolver en los casos en que sea compatible indeterminado.
\end{enumerate}
\end{exampleblock}

\end{frame}

\begin{frame}
\begin{exampleblock}{Ejemplo}

Un museo ofrece entradas con tarifas distintas: adulto, niño y jubilado. La suma de las tarifas de adulto y jubilado es cinco veces la tarifa de niño. Además, se sabe que un grupo de 5 adultos, 3 niños y 3 jubilados, ha pagado 222 euros; y otro grupo de 3 adultos, 2 niños y 4 jubilados, 168 euros.
\end{exampleblock}
\end{frame}

\begin{frame}
\begin{exampleblock}{ejemplo}
Tres nietos desean hacer un regalo de 60 euros a su abuela y deciden reunir esta cantidad de la siguiente forma: Luis, el mayor, aporta el triple de lo que aportan los otros dos juntos. Carmen aporta 3 euros por cada dos que aporta Pedro.
a) Plantear el sistema de ecuaciones lineales.
b) Resolver el sistema.
c) ¿Cuánto aporta cada nieto?
\end{exampleblock}
\end{frame}

\begin{frame}
\begin{exampleblock}{Ejemplo}
Una fábrica de tabletas de chocolate ha usado 200 kilogramos de chocolate y 100 litros de leche en la producción de dos tipos de tabletas A y B. Cada tableta de tipo A usa 0,2 kilogramos de chocolate y 0,1 litros de leche y cada tableta de tipo B usa $m$ kilogramos de chocolate y 0,2 litros de leche.
a) 
b) 
\begin{enumerate}
\item Plantea un sistema de ecuaciones (en función de $m$) donde las incógnitas $x$ e $y$ sean el número de tabletas producidas de tipo A y B, respectivamente. ¿Para qué valores de $m$ el sistema tiene solución? En caso de existir solución, ¿es siempre única?
\item Si cada tableta de tipo B precisa de 0,4 kg de chocolate y se produjeron 200 tabletas de tipo B, ¿cuántas se habrán producido de tipo A?
\end{enumerate}

\end{exampleblock}
\end{frame}

\end{document}


\begin{cejercicios}
\begin{ejer}
Considera el siguiente sistema de ecuaciones lineales,
\[ \begin{sistematres}
x & - & y & + & mz &= & 0 \\
mx & + & 2y & + & z & = & 0 \\
-x & + & y & + & 2mz & = & 0
\end{sistematres} \]



\bex
\itemps {Halla los valores del parámetro $m$ para los que el sistema tiene una única solución.}{}
\itemps {Halla los valores del parámetro $m$ para los que el sistema tiene alguna solución distinta de la solución nula.}{}
\itemps {Resuelve el sistema para $m=2$}{}
\eex
\end{ejer}

\begin{ejer}
Considera el siguiente sistema de ecuaciones lineales,
\[ \begin{sistematres}
x & + & (m+1)y & + & 2z &= & -1 \\
mx & + & y & + & z & = & m \\
(1-m)x & + & 2y & + & z & = & -m-1
\end{sistematres} \]

\bex
\itemps {Discute el sistema según los valores del parámetro $m$.}{}
\itemps {Resuelve el sistema para $m=2$. Para dicho valor de $m$, calcula, si es posible, una solución en la que $z=2$}{}
\eex
\end{ejer}

\end{cejercicios}


\newpage
\section{Problemas}


\begin{exampleblock}
En un comercio de bricolaje se venden listones de madera de tres longitudes: 0,90m, 1,50m y 2,40m, cuyos precios respectivos son 4 euros, 6 euros y 10 euros. Un cliente ha comprado 19 listones, con una longitud total de 30m, que le han costado 126 euros en total.

Plantee y resuelva el sistema de ecuaciones necesario para determinar cuántos listones de cada longitud ha comprado ese cliente.
\end{exampleblock}

\textbf{Solución:}

Las incógnitas serán:
\[ \left\lbrace 
\begin{array}{ccc} 
	x&=&\text{Listones de 0,90m} \\ 
	y&=&\text{Listones de 1,50m} \\
	z&=&\text{Listones de 2,40m}
\end{array}
\right.
\]

Según las condiciones del problema:
\[
\left\lbrace
\begin{array}{ccc}
\text{Cantidad total de listones = 19}& \rightarrow  & x+y+z=19 \\
\text{Longitud total de los listones = 30m} &  \rightarrow & 0,90x+1,50y+2,40z= 30 \\
\text{Precio total de los listones = 126 euros} & \rightarrow & 4x+6y+10z=126 
\end{array}
\right. \]

El sistema resultante es:

\[
\begin{sistematres}
x&+&y&+&z&=&19 \\
0,9x&+&1,5y&+&2,4z&=& 30 \\
4x&+&6y&+&10z&=&126
\end{sistematres}
\]

que es un sistema lineal donde:

\[ A=\begin{pmatrix}
1 & 1 & 1 \\
0,9 & 1,5 & 2,4 \\
4&6&10 
\end{pmatrix} \Rightarrow 
|A| = \begin{vmatrix}
1 & 1 & 1 \\
0,9 & 1,5 & 2,4 \\
4&6&10 
\end{vmatrix} = 15+9,6+5,4-6-14,4-9=0,6 \]

Luego el sistema es compatible determinado.

La solución, por la regla de Cramer, es:

\[
x=\dfrac{1}{0,6} \begin{vmatrix}
 19& 1&1 \\
 30 & 1,5 & 2,4 \\
 126 & 6 & 10
\end{vmatrix}= \dfrac{4,8}{0,6}= 8
\]
\[
y=\dfrac{1}{0,6} \begin{vmatrix}
 1& 19&1 \\
0,9 & 30 & 2,4 \\
4 & 126 & 10
\end{vmatrix}= \dfrac{2,4}{0,6}= 6
\]
\[
z=\dfrac{1}{0,6} \begin{vmatrix}
 1& 1&19 \\
 0,9 & 1,5 & 30 \\
 4 & 6 & 126
\end{vmatrix}= \dfrac{4,2}{0,6}= 7
\]

Ha comprado 8 listones de 0,9m, 6 listones de 1,5m y 7 listones de 2,4m.

\newpage
\section{Ejercicios}

\begin{cejercicios}

\begin{ejer}
Dado el siguiente sistema de ecuaciones:
\[ \left\lbrace \begin{array}{ccc}
x+y+az & = & 1 \\
x+ay+z & = & a \\
ax+y+z & = & 1
\end{array}
\right.
 \]
\bex
\itemps {Discutir el sistema en función del parámetro $a$}{$\begin{array}{l}
a \in \R -\{-2,1\} \\ \text{S. Compatible Determinado} \\
a=1 \\ \text{S. Compatible Indeterminado} \\
a=-2 \\ \text{S. Compatible Indeterminado} \\
\end{array} $}
\itemps {Si es posible, resolverlo para el valor de $ a= -2 $}{$x=\lambda$, $y=1+\lambda$, $z=\lambda$}
\eex
\end{ejer}

\begin{ejer}

\bex
\itemps {Discute el siguiente sistema de ecuaciones en función del parámetro $a$
\[ \begin{sistematres} 
\filasistematres x + y + {az}  1  \\
\filasistematres{x}{+}{ay}{+}{z}{a} \\
\filasistematres{ax}{+}{y}{+}{z}{1}
\end{sistematres}\]}{•}
\itemps {Si es posible, resuélvalo para el valor de $a=-2$}{•}
\eex
\end{ejer}

\begin{ejer}
Dado el sistema:
\[ \begin{sistematres} 
\filasistematres{(a+1)x}{+}{y}{+}{z}{a+1} \\
\filasistematres{x}{+}{(a+1)y}{+}{z}{a+3} \\
\filasistematres{x}{+}{y}{+}{(a+1)z}{-2a-4}
 \end{sistematres} \]
\bex
\itemps {Estudie su compatibilidad según los distintos valores del número real $a$}{•}
\itemps {Resuélvalo, si es posible, en el caso $a=3$}{•}
\eex
\end{ejer}

\begin{ejer}
Dado el sistema:
\[ \begin{sistematres} 
\filasistematres{x}{+}{(a-1)y}{+}{z}{1} \\
\filasistematres{ax}{+}{(2a-2)y}{+}{2z}{0} \\
\filasistematres{(a+1)x}{+}{(3a-3)y}{+}{(a+3)z}{0}
 \end{sistematres} \]
\bex
\itemps {Estudie su compatibilidad según los distintos valores del número real $a$}{•}
\itemps {Resuélvalo, si es posible, en el caso $a=1$}{•}
\eex
\end{ejer}

\begin{ejer}
Dado el sistema:
\[ \begin{sistematres} 
\filasistematres{ax}{-}{ay}{+}{3z}{a} \\
\filasistematres{-2x}{+}{3y}{-}{2z}{-1} \\
\filasistematres{2x}{-}{y}{+}{z}{a}
 \end{sistematres} \]
\bex
\itemps {Estudie su compatibilidad según los distintos valores del número real $a$}{•}
\itemps {Resuélvalo, si es posible, en el caso $a=1$}{•}
\eex
\end{ejer}

\section{Sistemas con parámetros}




\textbf{Solución:}

\begin{enumerate}
\item 

Formamos la matriz de los coeficientes y la matriz ampliada del sistema:
\[
A=\begin{pmatrix}
	1 & 1 & 1 \\
	2 & 1 & 0 \\
	3 & 2 & a
\end{pmatrix}
\qquad
M=\begin{pmatrix}
	1 & 1 & 1 & 2 \\
	2 & 1 & 0 & 0 \\
	3 & 2 & a & 2a
\end{pmatrix} 
\]

Calculamos el determinante de la matriz $A$. 

\[ \begin{vmatrix}
	1 & 1& 1 \\
	2 & 1 & 0 \\
	3 & 2 & a
\end{vmatrix}= a+0+4-3-0-2a=-a+1
\]

Si el valor del determinante es distinto de cero el rango de $A$ será $3$, en caso contrario será menor que $3$.
Igualamos a cero:

\[ -a+1=0 \rightarrow a=1 \]

Ahora discutimos el sistema según los valores de $a$.

Si $a\neq 1 \rightarrow rg(A)=3$. Sabemos que $rg(M)\geq rg(A) =3$. Como la matriz $M$ solo tiene $3$ filas $rg(M)\leq 3$, por lo tanto, $rg(M)=3$.

En consecuencia: Si $a \neq 1 \rightarrow rg(A)=rg(M)=3=\text{número de incógnitas} \rightarrow $ Sistema Compatible Determinado.

Si $a=1$ 

\[ A=\begin{pmatrix}
	1 & 1 & 1 \\
	2 & 1 & 0 \\
	3 & 2 & 1
\end{pmatrix}
\qquad
M=\begin{pmatrix}
	1 & 1 & 1 & 2 \\
	2 & 1 & 0 & 0 \\
	3 & 2 & 1 & 2
\end{pmatrix} 
\]

Calculamos el rango de $A$.

\[ \begin{vmatrix}
	 1 & 1 \\
 	2 & 1
\end{vmatrix}= 1-2=-1 \neq 0 \rightarrow rg(A)=2\]

Estudiamos el rango de $M$

\[ \begin{vmatrix}
	1 & 1& 2 \\
	2 & 1 & 0 \\
	3 & 2 & 2
\end{vmatrix}= 2+0+8-6-0-4=0 \rightarrow rg(M)=2
\]

En este caso $r(A)=rg(M)=2 < \text{número de incognitas} \rightarrow$Sistema Compatible Indeterminado

Resumiendo $\left\lbrace 
\begin{array}{lll} 
a\neq 1 & rg(A)=rg(M)=3=\text{nº incógnitas} & \text{S.C.D.} \\
a=1 & rg(A)=rg(M)=2 <\text{nº incógnitas} & \text{S.C.I.}
\end{array}
\right.$

\item Resolvemos cuando $a=1$.

Como el rango de las matrices es $2$ tenemos que eliminar una ecuación y pasar una incógnita al otro lado.

Como el determinante de orden $2$ que es distinto de cero incluye la primera, la segunda fila y la primera y la segunda columna, eliminamos la tercera ecuación y pasamos la incógnita $z$ al otro miembro, quedando el sistema de la siguiente forma:

\[ \begin{sistematres}
	x&+&y&=&2-z \\
	2x&+&y&=&0
\end{sistematres} \]

Lo resolvemos por Cramer:

\[ A=\begin{pmatrix}
	1 & 1 \\
	2 & 1
\end{pmatrix}
\Longrightarrow
\begin{vmatrix}
	1 & 1 \\
	2 & 1
\end{vmatrix}= 1-2=-1
\]
\[ x=\dfrac{1}{-1} 
\begin{vmatrix}
	2-z & 1 \\
	0 & 1
\end{vmatrix} =-2+z \qquad 
y=\dfrac{1}{-1} 
\begin{vmatrix}
	1 & 2-z  \\
	2 & 0
\end{vmatrix} =4-2z \]

Haciendo $z=\lambda $, la solución es: $(x,y,z)=(-2+\lambda , 4 - 2 \lambda , \lambda)$.

\end{enumerate}


