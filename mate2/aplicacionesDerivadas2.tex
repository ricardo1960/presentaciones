\documentclass[8pt]{beamer}
%\usetheme{CambridgeUS}
\logo{\includegraphics[scale=0.10]{../imagenes/logoa}}
\usepackage[spanish]{babel}
%\usecolortheme{seahorse}
%\usepackage{beamerthemeblackboard}
%\usepackage{graphics}
%\usecolortheme[RGB={6,138,200}]{structure}
%\usepackage[orientation=landscape, size=custom,
 % width=16, height=9, scale=0.5]{beamerposter}
%\usepackage[utf8]{inputenc}
\usetheme{metropolis}
\metroset{titleformat=smallcaps,block=fill}
\usepackage{booktabs}
\usepackage[scale=2]{ccicons}

\usepackage{amsmath}
\usepackage{amsfonts}
\usepackage{amssymb}
\usepackage{graphicx}
\usepackage{colortbl}
\usepackage{tikz} 
\usetikzlibrary{matrix}
\usetikzlibrary{calendar,decorations.markings} 
\usetikzlibrary{shapes,positioning}

\usepackage{tkz-tab,tkz-euclide,tkz-fct}
\usetkzobj{all}  
\usepackage{tcolorbox} 
%\usepackage{enumitem} 
\usepackage{tasks} 
\usepackage{asymptote}  
\newcommand{\sen}{\mathop{\rm sen}\nolimits}

\newcommand{\tg}{\mathop{\rm tg}\nolimits}
\newcommand{\arcsen}{\mathop{\rm arcsen}\nolimits}
\newcommand{\arctg}{\mathop{\rm arctg}\nolimits}
\newcommand{\g}{{}^\circ}     
        
\newcommand{\R}{\mathbb{R}}
\newcommand{\Z}{\mathbb{Z}}
\newcommand{\N}{\mathbb{N}}
\newcommand{\Q}{\mathbb{Q}}
\newcommand{\I}{\mathbb{I}}
\newcommand{\limite}[2]{\displaystyle \lim_{x \rightarrow #1}{#2}}
\renewcommand{\vector}[1]{\overrightarrow{#1}}

\newtcbox{\resultado}[1][center]{#1,colback=red!5!white,
colframe=red!75!black}

\newtcbox{\resaltado}[1][center]{#1,colback=blue!5!white,
colframe=blue!75!black}
\definecolor{titleColor}{rgb}{0.0, 0.42, 0.24}

\title{Aplicaciones de las derivadas 2}
\author{Ricardo Mateos}
\institute[UHEI-IVED]{Departamento de Matemáticas \\ UHEI - IVED}
\date{Matemáticas Aplicadas a las CC.SS. II}
\begin{document}
%\ECFJD
\begin{frame}
\maketitle
\end{frame}

\begin{frame}
\tableofcontents
\end{frame}



\begin{frame}[t]{Monotonia y extremos relativos}
\begin{exampleblock}{Ejemplo}
Dada la función $f(x)=ax^3+bx+c$ , calcular el valor de $a$, $b$ y $c$ para
que:
\begin{tasks}[label=\alph*)](1)
\task La función pase por el origen de coordenadas y tenga en el punto $(1,-1)$ un mínimo local.
\task Para los valores obtenidos en el apartado anterior, determine los intervalos de crecimiento y decrecimiento de la función.
\end{tasks}
\end{exampleblock}
\end{frame}

\begin{frame}[t]{Optimización}
\begin{exampleblock}{Ejemplo}
El coste de producción de $x$ unidades de un determinado producto viene dado por la función $C(x)=\dfrac{1}{4}x^2+35x+25$ y su precio de venta es $p=50-\dfrac{x}{4}$
 euros. Hallar:

\begin{tasks}[label=\alph*)](1)
\task El número de unidades que debe venderse diariamente para que el beneficio sea máximo.
\task El precio al que deben venderse las unidades obtenidas en el apartado a).
\task El beneficio máximo.
\end{tasks}

\end{exampleblock}
\end{frame}



\begin{frame}[t]{Optimización}
\begin{exampleblock}{Ejemplo}
El crecimiento (en cm) de una variedad de trigo, $C(x)$, en función de la cantidad de fertilizante (en gramos por metro cuadrado) utilizada, $x$, viene dado por la función:
\[ C(x)=2x^3-Ax^2+Bx+35 \qquad 0 \leq x \leq 4 \] 

Determinar las constantes $A$ y $B$ sabiendo que el crecimiento alcanza su mínimo con una dosis de 3 gramos por metro cuadrado y que para esta dosis las plantas de trigo crecen 8 cm.
\end{exampleblock}
\end{frame}


\begin{frame}[t]{Optimización}
\begin{exampleblock}{Ejemplo}
Las ventas de un producto (en miles de euros), $V(t)$, en los 6 primeros años desde que se lanzó al mercado, evolucionan de acuerdo con la siguiente función:
\[ V(t)=4t^3-24t^2 +36t+100 \qquad 0 \leq t \leq 6 \]
Se pide determinar:
\begin{tasks}[label=\alph*)](1)
\task Estudiar el crecimiento y decrecimiento de las ventas a lo largo de los 6 años.
\task ¿En qué año se produjeron las ventas máximas y las mínimas?.
\end{tasks}
\end{exampleblock}
\end{frame}


\begin{frame}[t]{Optimización}
\begin{exampleblock}{Ejemplo}
Un determinado vino tiene un tiempo de crianza en bodega de entre 1 y 4 años. La graduación del vino, $G(x)$, en términos del tiempo de crianza, $x$, viene dada por la función
\[ G(x)= x^3-Ax^2+6Bx+2 \qquad 1 \leq x \leq 4 \]
Determinar, justificando la respuesta, las constantes $A$ y $B$ sabiendo que la máxima graduación se consigue exactamente a los 2 años, edad en que el vino alcanza los 22 grados.


\end{exampleblock}
\end{frame}

\begin{frame}[t]{Curvatura y puntos de inflexión}
\begin{exampleblock}{Ejemplo}
El diámetro de cierta variedad de manzana oscila entre los $2$y los $5$ cm. El precio (en céntimos de euro), $P(x)$, que se le paga al agricultor por un kilogramo de estas manzanas viene determinado por su diámetro, x, de acuerdo con la siguiente función:
\[ P(x)=-2x^3+15x^2-24x+30 \qquad 2 \leq x \leq 5 \]
\begin{tasks}[label=\alph*)](1)
\task Determinar para qué diámetros se alcanzan los precios máximo y mínimo de las manzanas. 
\task ¿Cuáles son estos precios máximo y mínimo?.
\end{tasks}

\end{exampleblock}
\end{frame}

\begin{frame}[t]{Calculo de los máximos y mínimos en un intervalo}
\begin{exampleblock}{Ejemplo}
Dada la función $f(x)=\dfrac{x}{x^2+1}$, calcula:
\begin{tasks}[label=\alph*)](1)
\task Intervalos de crecimiento y decrecimiento.
\task Extremos relativos.
\task Asíntotas.
\task Haz una representación aproximada.
\end{tasks}
\end{exampleblock}
\end{frame}

\begin{frame}[t]{Optimización}
\begin{exampleblock}{Ejemplo}
El número de vehículos que ha pasado cierto día por el peaje de una autopista viene dado por la función: 
\[ N8t)=\begin{cases}  \left( \dfrac{t-3}{3} \right)^2+2 & \text{si} \quad 0<t \leq 9 \\ 10-\left( \dfrac{t-15}{3} \right)^2 & \text{si} \quad 9<t \leq 24 \end{cases}  \]
donde $N$ indica el número de vehículos y $t$ el tiempo transcurrido en horas desde las $0:00$ h.
\begin{tasks}[label=\alph*)](1)
\task ¿Es continua esta función? Justificar la respuesta.
\task ¿Entre qué horas aumentó el número de vehículos que pasaba por el peaje? Justificar la respuesta.
\task ¿A qué hora pasó el mayor número de vehículos? ¿Cuántos fueron?
\end{tasks}
\end{exampleblock}
\end{frame}

\end{document}