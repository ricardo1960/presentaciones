\documentclass[8pt]{beamer}
%\usetheme{CambridgeUS}
\logo{\includegraphics[scale=0.10]{../imagenes/logoa}}
\usepackage[spanish]{babel}
%\usecolortheme{seahorse}
%\usepackage{beamerthemeblackboard}
%\usepackage{graphics}
%\usecolortheme[RGB={6,138,200}]{structure}
%\usepackage[orientation=landscape, size=custom,
 % width=16, height=9, scale=0.5]{beamerposter}
%\usepackage[utf8]{inputenc}
\usetheme{metropolis}
\metroset{titleformat=smallcaps,block=fill}
\usepackage{booktabs}
\usepackage[scale=2]{ccicons}

\usepackage{amsmath}
\usepackage{amsfonts}
\usepackage{amssymb}
\usepackage{graphicx}
\usepackage{colortbl}
\usepackage{tikz} 
\usetikzlibrary{matrix}
\usetikzlibrary{calendar,decorations.markings} 
\usetikzlibrary{shapes,positioning}

\usepackage{tkz-tab,tkz-euclide,tkz-fct}
\usetkzobj{all}  
\usepackage{tcolorbox} 
%\usepackage{enumitem} 
\usepackage{tasks} 
\usepackage{asymptote}           
\newcommand{\R}{\mathbb{R}}
\newcommand{\limite}[2]{\displaystyle \lim_{x \rightarrow #1}{#2}}
\renewcommand{\vector}[1]{\overrightarrow{#1}}

\newtcbox{\resultado}[1][center]{#1,colback=red!5!white,
colframe=red!75!black}

\newtcbox{\resaltado}[1][center]{#1,colback=blue!5!white,
colframe=blue!75!black}
\definecolor{titleColor}{rgb}{0.0, 0.42, 0.24}

\title{Geometría}
\author{Ricardo Mateos}
\institute[UHEI-IVED]{Departamento de Matemáticas \\ UHEI - IVED}
\date{Matemáticas II}
\begin{document}
%\ECFJD
\begin{frame}
\maketitle
\end{frame}

\begin{frame}
\tableofcontents
\end{frame}

\section{Ecuaciones de la recta}

\begin{frame}[t]{Ecuaciones de la recta}
\begin{exampleblock}{Ejemplo recta que pasa por dos puntos}
Hallar las ecuaciones de la recta que pasa por los puntos $A(1,1,-1)$ y $B(0,2,3)$. Hallar otro punto de la recta. ¿Pertenece el punto $C(-1,-1,1)$ a dicha recta?. 
\end{exampleblock}



\textbf{Solución}

Si la recta pasa por dos puntos, el vector que une los dos puntos será un vector director de la recta. 

Por lo tanto, para calcular la recta utilizaremos uno de los puntos y el vector que une los dos puntos como vector director.

Calculamos el vector $\vector{AB}=(-1,1,4)$.

Hallamos las ecuaciones de la recta en todas sus formas:
\end{frame}

\begin{frame}{Ecuaciones de la recta}

\textbf{\color{titleColor!70}Ecuación vectorial:}  $(x,y,z)=(1,1,-1)+(-1,1,4)\lambda$

\textbf{\color{titleColor!70}Ecuaciones paramétricas:} $\begin{cases} x=1-\lambda \\ y=1+\lambda \\ z= -1+4 \lambda\end{cases}$

\textbf{\color{titleColor!70}Ecuación continua:} $\dfrac{x-1}{-1}=\dfrac{y-1}{1}=\dfrac{z+1}{4}$ 

Para hallar otro punto de la recta damos al parámetro un valor cualquiera, por ejemplo $\lambda=-2$ y obtenemos el punto $P(3,-1,-9)$

Para saber si el punto $C$ pertenece a la recta miramos si cumple las ecuaciones de la recta:

\[ \dfrac{-1-1}{-1} \neq \dfrac{-1-1}{1} \]

por lo tanto el punto $C $ no pertenece a la recta.
\end{frame}

\begin{frame}[t]{Ecuaciones de la recta}
\begin{exampleblock}{Ejemplo recta que pasa por un punto y es paralela a otra}
Hallar las ecuaciones de la recta que pasa por el punto $A(1,2,-2)$ y es paralela a la recta $s \equiv  \dfrac{x-2}{1}=\dfrac{y}{-1}=\dfrac{z+2}{2}$. 
\end{exampleblock}

Si las rectas son paralelas tienen la misma dirección, por lo tanto, cualquier vector director de una de ellas es vector director de la otra.

Hallamos el vector director de la recta $s$: $\vec{v}=(1,-1,2)$

Con este vector y el punto hallamos las distintas ecuaciones de la recta buscada:

\textbf{\color{titleColor!70}Ecuación vectorial:}  $(x,y,z)=(1,2,-2)+(1,-1,2)\lambda$

\textbf{\color{titleColor!70}Ecuaciones paramétricas:} $\begin{cases} x=1+\lambda \\ y=2-\lambda \\ z= 2+2 \lambda\end{cases}$

\textbf{\color{titleColor!70}Ecuación continua:} $\dfrac{x-1}{1}=\dfrac{y-2}{-1}=\dfrac{z+2}{2}$ 

\end{frame}

\section{Ecuaciones del plano}

\begin{frame}[t]{Ecuaciones del plano}
\begin{exampleblock}{Ecuaciones del plano que contiene a tres puntos}
Hallar las ecuaciones del plano que contiene a los puntos $A(1,1,-1)$, $B(1,2,1)$ y $C(2,-1,3)$
\end{exampleblock}

Si el plano contiene a los tres puntos entonces los vectores que unen esos tres puntos son vectores directores del plano.

Hallamos los vectores  $\vector{AB}$ y  $\vector{AC}$. 

$\vector{AB}=(0,1,2)$, $\vector{AC}=(1,-2,4)$

Con estos vectores y el punto $A$ hallamos la ecuación general del plano.

$\begin{vmatrix}
x-1 & 0 & 1 \\ y-1 & 1 & -2 \\ z+1 & 2 & 4 
\end{vmatrix}=0 \Rightarrow 8x-8+2y-2-z-1=0$
 \resultado{$\pi \equiv 8x+2y-z-11=0$}
\end{frame}


\begin{frame}[t]{Ecuaciones del plano}
\begin{exampleblock}{Ecuaciones del plano que contiene a un punto y una recta}
Hallar las ecuaciones del plano que contiene al punto $A(2,-1,0)$ y a la recta $r \equiv \begin{cases} x= 2 + 2 \lambda \\ y= 1+3\lambda \\ z= -2-\lambda \end{cases}$
\end{exampleblock}

Si un plano contiene a una recta y a un punto, el vector de la recta es un vector director del plano y el vector que une un punto de la recta con el punto también será vector director del plano.

Calculamos un punto de la recta y un vector director de la misma:

$B(2,1,-2)$, $\vec{v}=(2,3,-1)$

Hallamos el vector que une los dos puntos: $\vec{AB}=(0,2,-2)$

Con estos dos vectores y el punto $A$ hallamos la ecuación general del plano:

$\begin{vmatrix}
x-2 & 2 & 0 \\ y+1 & 3 & 2 \\ z & -1 & -2 
\end{vmatrix}=0 \Rightarrow -4x+8+4y+4+4z=0$
\resultado{$ \pi \equiv -4x+4y+4z+12=0$}
\end{frame}

\begin{frame}[t]{Ecuaciones del plano}
\begin{exampleblock}{Pasar de unas ecuaciones del plano a otras}
\begin{tasks}[label=\alph*)](1)
\task Escribe la ecuación general del plano $\pi \equiv \begin{cases} x= 1-\lambda +2 \mu \\ y=2\lambda +\mu  \\ z= 2 +3\mu \end{cases}$
\task Halla las ecuaciones paramétricas del plano $\pi \equiv x-3y+2z-4=0 $
\end{tasks}
\end{exampleblock}

\begin{tasks}[label=\alph*)](1)
\task Tomamos un punto y los vectores directores del plano para hallar la ecuación general del plano:

$\begin{vmatrix}
x-1 & -1 & 2 \\ y & 2 & 1 \\ z-2 & 0 & 3 
\end{vmatrix}=0 \Rightarrow 6x-6+3y-5z+10=0$
\resultado{$  \pi \equiv 6x+3y-5z+10=0$}
Para hallar las ecuaciones paramétricas despejamos una de las incógnitas en función de las otras.
\end{tasks}
\end{frame}

\begin{frame}[t]{Ecuaciones del plano}
\begin{tasks}[label=\alph*),resume](1)
\task Para hallar las ecuaciones paramétricas despejamos una de las incógnitas en función de las otras.

\resultado{$\pi \equiv \begin{cases} x= 4+3\lambda -2\mu \\ y=\lambda \\ z= \mu \end{cases} $}
\end{tasks}
\end{frame}

\section{Posición relativa de dos planos}

\begin{frame}[t]{Posición relativa de dos planos}
\begin{exampleblock}{Ejemplo posición relativa de dos planos}
Hallar la posición relativa de los siguientes pares de planos:
\begin{tasks}[label=\alph*)](1)
\task $\pi_1 \equiv x-2y+2z -3=0 \qquad \pi_2 \equiv 2x-4y+4z-3=0$
\task $\pi_1 \equiv 2x-y-z +2=0 \qquad \pi_2 \equiv x+4y-3z-2=0$
\end{tasks}
\end{exampleblock}

\begin{tasks}[label=\alph*)](1)
\task $\dfrac{1}{2}= \dfrac{-2}{-4}=\dfrac{2}{4}\neq \dfrac{-3}{-3} \Rightarrow $ 
\resultado{Son paralelos.}

\vspace{0.5cm}
\task $\dfrac{2}{1}\neq \dfrac{-1}{-4} \Rightarrow $
Son secantes.
\end{tasks}

\end{frame}

\begin{frame}[t]{Recta determinada por la intersección de dos planos}
\begin{exampleblock}{Ejemplo recta intersección de dos planos}
Considere los planos de ecuaciones $\pi_1 \equiv x-y+z = 0$ y $\pi_2 \equiv x+y-z = 2$.
\begin{tasks}[label=\alph*)](1)
\task Compruebe que los planos se cortan y calcule la ecuación de la recta $r$ determinada por la intersección de ambos planos.
\task Compruebe que el punto $A = (3,2,1)$ no está en $\pi_1$ ni en $\pi_2$ y calcule la
ecuación del plano $\pi_3$ que contiene a la recta $r$ y pasa por el punto $A$.
\end{tasks}
\end{exampleblock}
\begin{tasks}[label=\alph*)](1)
\task $\dfrac{1}{1} \neq \dfrac{-1}{1} \Rightarrow $ Son secantes. 

Resolviendo el sistema:
\resultado{ $r \equiv \begin{cases} x= 1 \\ y = 1+\lambda \\ z=\lambda \end{cases}$}
\end{tasks}
\end{frame}

\begin{frame}[t]{Recta determinada por la intersección de dos planos}
\begin{tasks}[label=\alph*),resume](1)

\task Comprobamos si el punto cumple las ecuaciones de los planos:

$3-2+1 \neq 0 \qquad 3+2-1 \neq 2 $.

El punto no pertenece a ninguno de los planos.

Para calcular el plano tomamos el punto $A$, el vector director de la recta y el vector que une uno de los puntos de la recta con el punto $A$

$\begin{vmatrix}
x-3 & 0 & 2 \\ y-2 & 1 & 1 \\ z-1 & 1 & 1 
\end{vmatrix}=0 \Rightarrow +2-2z+2=0$
 \resultado{$ \pi \equiv y-z+2=0$}
\end{tasks}

\end{frame}

\begin{frame}[t]{Ecuación implicita de la recta. Haz de planos.}
\begin{exampleblock}{Ejemplo ecuación implicita de la recta y haz de planos.}
Dada la recta $r \equiv \begin{cases} -x+y+2=0 \\ 2x-y+3z-1=0 \end{cases}$.
\begin{tasks}[label=\alph*)](1)
\task Hallar un punto y un vector director de la recta $r$.
\task Escribir las ecuaciones paramétricas de la recta $r$.
\task Determinar el haz de planos que contiene a dicha recta.
\task Del haz de planos que contiene a la recta $r$ hallar el que contiene al punto $P(1,1,1)$
\end{tasks}
\end{exampleblock}

\begin{tasks}[label=\alph*)](1)
\task Resolvemos el sistema en función de un paramétro y tenemos la recta:

\[ r \equiv \begin{cases} x= -1 -3\lambda \\ y=-3-3\lambda  \\ z= \lambda \end{cases} \]

Luego un punto es $(-1,-3,0)$ y un vector director es $(-3,-3,1)$

\task Las ecuaciones paramétricas son las expresadas en el apartado anterior.
\end{tasks}
\end{frame}

\begin{frame}[t]{Ecuación implicita de la recta. Haz de planos.}

\begin{tasks}[label=\alph*),resume](1)
\task El haz de planos que contiene a la recta es:

\[ -x+y+2+t (2x-y+3z-1)=0 \]
\task Para calcular el plano que contiene al punto sustituimos los valores del punto en el haz de planos y calculamos el valor de $t$.

\[ -1+1+2+t (2\cdot 1-1+3 \cdot 1-1)=0 \Rightarrow 2+3t=0 \Rightarrow t=-\frac{2}{3} \]

El plano buscado es : 

\[\pi \equiv -x+y+2-\frac{2}{3} (2x-y+3z-1)=0 \Rightarrow \pi \equiv -7x+5y-6z+8=0  \]
\end{tasks}

\end{frame}
\section{Posición relativa de tres planos.}

\begin{frame}{Posición relativa de tres planos}
\begin{exampleblock}{Ejemplo posición relativa de tres planos}
Estudia la posición relativa de los planos dados por las siguientes ecuaciones:
\begin{tasks}[label=\alph*)](2)
\task $\begin{cases} \pi_1 \equiv 2x+y-z-2=0 \\ \pi_2 \equiv 3x+2y-z-3=0 \\ \pi_3 \equiv 2x+3y+z-1=0 \end{cases} $
\task $\begin{cases} \pi_1 \equiv x+3y-z+5=0 \\ \pi_2 \equiv 2x+5y+z+4=0 \\ \pi_3 \equiv x-y-5z+3=0 \end{cases} $
\end{tasks}

\end{exampleblock}
\end{frame}

\begin{frame}{Posición relativa de tres planos}
\begin{exampleblock}{Ejemplo posición relativa de tres planos}
Se dan los planos \[ \begin{cases} \pi_1 \equiv  x + y + z = a - 1 \\ \pi_2 \equiv  2x + y + az = a \\\pi_3 \equiv  x + ay + z = 1 \end{cases} \]
\begin{tasks}[label=\alph*)](1)
\task Determinad la posición relativa de los tres planos en función del parámetro $a$. 
\task  Para $a = 1$, calculad, si existe, la recta de corte entre los planos $\pi_1$ y $\pi_3$.
\task  Para $a = 2$, calculad, si existe, la recta de corte entre los planos $\pi_1$ y $\pi_2$.
\end{tasks}
\end{exampleblock}

\end{frame}

\section{Posición relativa de dos rectas}

\begin{frame}{Posición relativa de dos rectas}
\begin{exampleblock}{Ejemplo}
Hallar la posición relativa de las rectas:
\[ r \equiv \dfrac{x-2}{1}=\dfrac{y+1}{1}=\dfrac{z+4}{-3} \qquad s \equiv \begin{cases} x+z=2 \\ -2x+y-2z = 1 \end{cases} \]
\end{exampleblock}
\end{frame}

\begin{frame}{Posición relativa de dos rectas}
\begin{exampleblock}{Ejemplo}
Calcular el valor del parámetro real $a$ para que las rectas $r$ y $s$ se corten y calcular este punto.
\[ r \equiv \begin{cases} 4x+z=a \\ x+y = 2 \end{cases} \qquad s \equiv \begin{cases} x+y+z=0 \\ x+2z = 2a \end{cases} \]
\end{exampleblock}
\end{frame}


\section{Posición relativa de recta y plano}

\begin{frame}{Posición relativa de recta y plano}
\begin{exampleblock}{Ejemplo posición relativa de recta y plano}
Dadas la recta $r \equiv \dfrac{x-5}{-1}=\dfrac{y-3}{1}=\dfrac{z-5}{-1}$ y el plano $\pi \equiv 3x-4y-8z+35=0$, hallar su posición relativa. Calcula el punto de corte, si son secantes o el plano que contiene a la recta y es paralelo al plano, si son paralelos.  
\end{exampleblock}
\end{frame}

\begin{frame}{Posición relativa de recta y plano}
\begin{exampleblock}{Ejemplo posición recta y plano}
Dados el plano $\pi \equiv  kx + y - z = 0$ y la recta $r \equiv \dfrac{x-4}{2}=\dfrac{y-2}{1}=\dfrac{z+2}{-1}$.

Determinar los valores del parámetro $k \in  \R$ para que el plano $\pi$ contenga a $r$.
\end{exampleblock}
\end{frame}
\section{Proyecciones de puntos sobre rectas y planos}

\begin{frame}{Proyección de un punto sobre una recta}
\begin{exampleblock}{Ejemplo}
Hallar la proyección del punto $P(0,-3,0)$ sobre:
\begin{tasks}[label=\alph*)](1)
\task la recta $\dfrac{x-1}{1}=\dfrac{y-2}{-3}=\dfrac{z}{2}$
\task el plano $x+2y-z-6=0$
\end{tasks}
\end{exampleblock}
\end{frame}


\section{Puntos simétricos}


\subsection{Punto simétrico respecto a una recta}
\begin{frame}{Punto simétrico respecto a una recta}
\begin{exampleblock}{Ejemplo}
Hallar el simétrico del punto $P(2,4,9)$ respecto de la recta $\begin{cases} x= 2-\lambda \\ y = 1+2\lambda \\ z= 3+2\lambda \end{cases} $
\end{exampleblock}
\end{frame}

\subsection{Punto simétrico respecto a un plano}
\begin{frame}{Punto simétrico respecto a un plano}
\begin{exampleblock}{Ejemplo}
Hallar el simétrico del punto $P(2,4,9)$ respecto del plano $3x-2y+z+7=0$
\end{exampleblock}
\end{frame}


\section{Problemas geométricos}
\begin{frame}{Recta que pasa por un punto y corta perpendicularmente a otra}
\begin{exampleblock}{Ejemplo}
Hallar las ecuaciones de la recta que pasa por el punto $P(2,-1,1)$ y corta perpendicularmente a la recta $r \equiv \dfrac{x-2}{2}=\dfrac{y-1}{2}=\dfrac{z}{1}$
\end{exampleblock}
\end{frame}

\begin{frame}{Perpendicular común a dos rectas que se cruzan}
\begin{exampleblock}{Ejemplo}
Dadas las rectas $r \equiv \begin{cases} x=1+\lambda \\ y=2-3\lambda \\ z=-8 \end{cases}$ y $s  \equiv \begin{cases} x+y+z=2 \\ x-y-z = 4 \end{cases}$:
\begin{tasks}[label=\alph*)](1)
\task Demostrar que se cruzan.
\task Hallar la perpendicular común a ambas rectas.
\end{tasks}
\end{exampleblock}
\end{frame}

\end{document}