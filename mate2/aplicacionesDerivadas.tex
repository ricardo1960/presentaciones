\documentclass[8pt]{beamer}
%\usetheme{CambridgeUS}
\logo{\includegraphics[scale=0.10]{../imagenes/logoa}}
\usepackage[spanish]{babel}
%\usecolortheme{seahorse}
%\usepackage{beamerthemeblackboard}
%\usepackage{graphics}
%\usecolortheme[RGB={6,138,200}]{structure}
%\usepackage[orientation=landscape, size=custom,
 % width=16, height=9, scale=0.5]{beamerposter}
%\usepackage[utf8]{inputenc}
\usetheme{metropolis}
\metroset{titleformat=smallcaps,block=fill}
\usepackage{booktabs}
\usepackage[scale=2]{ccicons}

\usepackage{amsmath}
\usepackage{amsfonts}
\usepackage{amssymb}
\usepackage{graphicx}
\usepackage{colortbl}
\usepackage{tikz} 
\usetikzlibrary{matrix}
\usetikzlibrary{calendar,decorations.markings} 
\usetikzlibrary{shapes,positioning}

\usepackage{tkz-tab,tkz-euclide,tkz-fct}
\usetkzobj{all}  
\usepackage{tcolorbox} 
%\usepackage{enumitem} 
\usepackage{tasks} 
\usepackage{asymptote}  
\usepackage{cancel}
\usepackage{xfrac}

\newcommand{\sen}{\mathop{\rm sen}\nolimits}

\newcommand{\tg}{\mathop{\rm tg}\nolimits}
\newcommand{\arcsen}{\mathop{\rm arcsen}\nolimits}
\newcommand{\arctg}{\mathop{\rm arctg}\nolimits}
\newcommand{\g}{{}^\circ}     
        
\newcommand{\R}{\mathbb{R}}
\newcommand{\Z}{\mathbb{Z}}
\newcommand{\N}{\mathbb{N}}
\newcommand{\Q}{\mathbb{Q}}
\newcommand{\I}{\mathbb{I}}
\newcommand{\limite}[2]{\displaystyle \lim_{x \rightarrow #1}{#2}}
\renewcommand{\vector}[1]{\overrightarrow{#1}}

\newtcbox{\resultado}[1][center]{#1,colback=red!5!white,
colframe=red!75!black}

\newtcbox{\resaltado}[1][center]{#1,colback=blue!5!white,
colframe=blue!75!black}
\definecolor{titleColor}{rgb}{0.0, 0.42, 0.24}

\title{Aplicaciones de las derivadas}
\author{Ricardo Mateos}
\institute[UHEI-IVED]{Departamento de Matemáticas \\ UHEI - IVED}
\date{Matemáticas II}
\begin{document}
%\ECFJD
\begin{frame}
\maketitle
\end{frame}

\begin{frame}
\tableofcontents
\end{frame}

\section{Regla de L'Hopital}

\begin{frame}[t]{Regla de L`Hopital}
\begin{exampleblock}{Ejemplo}
Calcular los siguientes límites:
\begin{tasks}[label=\alph*)](2)
\task $\limite{0}{\dfrac{e^x - \cos x}{\ln(1+x)}}$
\task $\limite{0}{\dfrac{e^x-x-\cos(3x)}{\sin^2x}}$
\task $\limite{0}{\dfrac{1-\cos x}{\left( e^x-1 \right)^2}}$
\task $\limite{0}{\dfrac{e^{x^2}-1}{\cos x-1}}$
\task $\limite{\frac{\pi}{2}}{\left( \sen x \right)^{\tg x}}$
\task $\limite{0}{\left( \cos x \right)^{\sfrac{1}{\sen^2 x}}}$
\end{tasks}
\end{exampleblock}
\end{frame}

\begin{frame}[t]{Regla de L'Hopital}
\begin{exampleblock}{Ejemplo}
Sabiendo que $\limite{0}{\dfrac{a\cdot \sen x-x\cdot e^x}{x^2}}$ es finito, calcula el valor de $a$ y el de dicho límite.
\end{exampleblock}
\end{frame}

\begin{frame}[t]{Regla de L'Hopital}
\begin{exampleblock}{Ejemplo}
Hallar el valor de $k$ para que \[\limite{0}{\dfrac{e^x-e^{-x}+kx}{x -\sen x}}=2 \].
\end{exampleblock}
\end{frame}

\section{Optimizacion}

\begin{frame}[t]{Optimización}
\begin{exampleblock}{Ejemplo}
Sean tres números reales positivos cuya suma es 90 y uno de ellos es la media de los
otros dos. Determina los números de forma que el producto entre ellos sea máximo.
\end{exampleblock}
\pause

\end{frame}


\begin{frame}[t]{Optimización}
\begin{exampleblock}{Ejemplo}
Descomponer el número 12 en dos sumandos positivos de forma que el producto
del primero por el cuadrado del segundo sea máximo.
\end{exampleblock}
\end{frame}

\begin{frame}[t]{Optimización}
\begin{exampleblock}{Ejemplo}
Un espejo plano, cuadrado, de 80 cm de lado, se ha roto por una esquina siguiendo una línea recta. El trozo desprendido tiene forma de triángulo rectángulo de catetos 32 cm y 40 cm respectivamente. En el espejo roto recortamos una pieza rectangular $R$, uno de cuyos vértices es el punto $(x,y)$ (véase la figura).
\begin{tasks}[label=\alph*)](1)
\task Hallad el área de la pieza rectangular obtenida como función de $x$, cuando \\ $0 \leq x \leq 32$.
\task Calculad las dimensiones que tendrá $R$ para que su área sea máxima. 
\task Calculad el valor de dicha área máxima.
\end{tasks}
\end{exampleblock}
\begin{center}
\begin{tikzpicture}[scale=0.35]
	\tkzInit[xmin=-1,xmax=9, ymin=-1,ymax=9]
	\tkzDefPoints{0/0/A, 0/8/B, 8/0/C, 8/8/D, 0/4/E, 3.2/0/F, 2.4/1/G, 2.4/8/H, 8/1/I}
	\tkzDefPoints{0/1/A1, 2.4/0/A2}
	\tkzDefPoints{-2/4/B1, -2/0/B2}
	\tkzDefPoints{-1/1/B3, -1/0/B4}
	\tkzDefPoints{0/-2/C1, 3.2/-2/C2}
	\tkzDefPoints{0/-1/C3, 2.4/-1/C4}
	\tkzDrawSegments[line width=2pt](E,F G,H G,I)
	\tkzDrawPolygon[line width=2pt](A,C,D,B)
	\tkzDrawSegments[color=blue](A1,G A2,G)
	\tkzDrawSegments[color=blue](B1,E B2,A A1,B3)
	\tkzDrawSegments[color=blue,latex-latex](B1,B2)
	\tkzDrawSegments[color=blue,latex-latex](B3,B4)
	\tkzDrawSegments[color=blue](C2,F C1,A A2,C4)
	\tkzDrawSegments[color=blue,latex-latex](C1,C2)
	\tkzDrawSegments[color=blue,latex-latex](C3,C4)
	%\tkzDrawPoint(G)
	%\tkzLabelPoint[above left](G){$P_1$}	
	\tkzLabelSegment[left](B1,B2){$40$}
	\tkzLabelSegment[right](B3,B4){$y$}
	\tkzLabelSegment[below](C1,C2){$32$}
	\tkzLabelSegment[above](C3,C4){$x$}
	%\tkzFct[color=red,domain=0:7.3]{(x**2-2*x+13)/12)}
	%\tkzFct[color=red,domain=0:4]{-x}
\end{tikzpicture} 
\end{center}
\end{frame}

\begin{frame}[t]{Optimización}
\begin{exampleblock}{Ejemplo}


En una nave industrial se quiere instalar una pantalla de cine (ver figura). La forma de la nave es la descrita por la gráfica de la función $f(x) = 12 - \dfrac{x^2}{3} \geq 0$. Calcula los valores positivos $(x,y)$ que hacen máxima el área de la pantalla.
\begin{center}
\begin{tikzpicture}[scale=0.25]
	\tkzInit[xmin=-7,xmax=7, ymin=-1,ymax=13]
	\tkzDefPoints{-7/0/A, 7/0/B, 0/-1/C, 0/13/D, -3/0/E, 3/0/F, -3/9/G, 3/9/H}
	\tkzDrawSegments[line width=1pt](A,B C,D)
	\tkzDrawPolygon[line width=1pt,fill=gray](E,G,H,F)
	\tkzFct[color=red,domain=-6:6]{12-x**2/3}
	\tkzDrawPoint(H)
	\tkzLabelPoint[above right](H){$(x,y)$}
	\tkzLabelPoint[above left,color=red](G){$y=f(x)$}
\end{tikzpicture} 
\end{center}
\end{exampleblock}
\end{frame}
\end{document}