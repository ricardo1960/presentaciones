\documentclass[8pt,handout]{beamer}
%\usetheme{CambridgeUS}
\logo{\includegraphics[scale=0.10]{../imagenes/logoa}}
\usepackage[spanish]{babel}
%\usecolortheme{seahorse}
%\usepackage{beamerthemeblackboard}
%\usepackage{graphics}
%\usecolortheme[RGB={6,138,200}]{structure}
%\usepackage[orientation=landscape, size=custom,
 % width=16, height=9, scale=0.5]{beamerposter}
%\usepackage[utf8]{inputenc}
\usetheme{metropolis}
\metroset{titleformat=smallcaps,block=fill}
\usepackage{booktabs}
\usepackage[scale=2]{ccicons}

\usepackage{amsmath}
\usepackage{amsfonts}
\usepackage{amssymb}
\usepackage{graphicx}
\usepackage{colortbl}
\usepackage{tikz} 
\usetikzlibrary{matrix}
\usetikzlibrary{calendar,decorations.markings} 
\usetikzlibrary{shapes,positioning}

\usepackage{tkz-tab,tkz-euclide,tkz-fct}
\usetkzobj{all}  
\usepackage{tcolorbox} 
\usepackage{enumitem} 
\usepackage{tasks} 
\usepackage{asymptote}  
\usepackage{cancel}
\newcommand{\sen}{\mathop{\rm sen}\nolimits}

\newcommand{\tg}{\mathop{\rm tg}\nolimits}
\newcommand{\arcsen}{\mathop{\rm arcsen}\nolimits}
\newcommand{\arctg}{\mathop{\rm arctg}\nolimits}
\newcommand{\g}{{}^\circ}     
        
\newcommand{\R}{\mathbb{R}}
\newcommand{\Z}{\mathbb{Z}}
\newcommand{\N}{\mathbb{N}}
\newcommand{\Q}{\mathbb{Q}}
\newcommand{\I}{\mathbb{I}}
\newcommand{\limite}[2]{\displaystyle \lim_{x \rightarrow #1}{#2}}
\renewcommand{\vector}[1]{\overrightarrow{#1}}

\newtcbox{\resultado}[1][center]{#1,colback=red!5!white,
colframe=red!75!black}

\newtcbox{\resaltado}[1][center]{#1,colback=blue!5!white,
colframe=blue!75!black}
\definecolor{titleColor}{rgb}{0.0, 0.42, 0.24}

\title{Límites}
\author{Ricardo Mateos}
\institute[UHEI-IVED]{Departamento de Matemáticas \\ UHEI - IVED}
\date{Matemáticas II}
\begin{document}
%\ECFJD
\begin{frame}
\maketitle
\end{frame}

\begin{frame}
\tableofcontents
\end{frame}

\begin{frame}{Límites}
\begin{exampleblock}{Ejemplo}
Utilizando la gráfica calcular los limites de las siguientes funciones:
 \begin{tasks}[label=\alph*)](2)
\task 
En $x=-1$, $x=0$, $x=1$ y $x=2$

\begin{tikzpicture}[scale=0.5]
\tkzInit[xmin=-2,xmax=3,ymin=-4,ymax=2]
\tkzGrid \tkzDrawXY
\tkzFct[domain=-1.5:1,color=blue]{2*x-1}
\tkzDrawPoint[fill=blue,color=blue](1,1)
\tkzFct[domain=1:2.5,color=blue]{-x*x+1}
\tkzDrawPoint[fill=white,color=blue](1,0)
\end{tikzpicture}
\task
En  $x=0$, $x=1$, $x=2$ y $x=3$

\begin{tikzpicture}[scale=0.25]
\tkzInit[xmin=-6,xmax=6,ymin=-6,ymax=6]
\tkzGrid \tkzDrawXY
\tkzFct[domain=-6:1.9]{1/(x-2)}
\tkzFct[domain=2.1:6]{1/(x-2)}
\tkzVLine[style=dashed,color=red]{2}
\end{tikzpicture}
\end{tasks}
\end{exampleblock}

\begin{tasks}[label=\alph*)](2)
\task 
$\limite{-1}{f(x)}= -3$

$\limite{0}{f(x)}= -1$

$ \left. \begin{array}{l}
\limite{1^{-}}{f(x)} = 1 \\
\limite{1^{+}}{f(x)}= 0
\end{array} \right\rbrace  \nexists \limite{1}{f(x)}$

$\limite{2}{f(x)}=-3$

\task
$\limite{0}{g(x)}= 0.5$

$\limite{1}{g(x)}=1$

$ \left. \begin{array}{l}
\limite{2^{-}}{g(x)} = -\infty \\
\limite{2^{+}}{g(x)}= +\infty
\end{array} \right\rbrace  \nexists \limite{1}{f(x)}$

$\limite{3}{g(x)}=1$

\end{tasks}
\end{frame}



\begin{frame}{Límite en un punto}
\begin{exampleblock}{Ejemplo}
Calcular los siguientes límites:
\begin{tasks}[label=\alph*)](2)
\task $\limite{3}{\left( 2 \sqrt{x+1} \right)}$
\task $\limite{3}{\dfrac{x^2+1}{x-3}}$
\task $\limite{1}{\dfrac{x^2-1}{x-1}}$
\task $\limite{2}{\left( \dfrac{x+1}{x} \right)^{\frac{x-2}{x}}}$
\end{tasks}
\end{exampleblock}
Para hallar el límite sustituimos la $x$ por el valor al cual tiende el límite.
\begin{tasks}[label=\alph*)](1)
\task $\limite{3}{\left( 2 \sqrt{x+1} \right)}= \left( 2 \sqrt{3+1} \right)= 4 $
\task $\limite{3}{\dfrac{x^2+1}{x-3}}=\dfrac{3^2+1}{3-3}=\dfrac{10}{0} = \infty$
\task $\limite{1}{\dfrac{x^2-1}{x-1}}\dfrac{1^2-1}{1-1}=\left[\dfrac{0}{0}\right]_{\text{Indet.}}$
\task $\limite{2}{\left( \dfrac{x+1}{x} \right)^{\frac{x-2}{x}}}=\left( \dfrac{2+1}{2} \right)^{\frac{2-2}{2}}=\left( \dfrac{3}{2} \right)^{0}=1$
\end{tasks}
\end{frame}

\begin{frame}{Límite en un punto}
\begin{exampleblock}{Ejemplo}
Dada la función $f(x)=\begin{cases} 2^x+1 & x \leq 0 \\ -3x+2 & 0 <x < 2 \\  \dfrac{-2x}{x-1} & x \geq 2\end{cases} $, calcular los siguientes límites:
\begin{tasks}[label=\alph*)](2)
\task $\limite{-1}{f(x)}$
\task $\limite{0}{f(x)}$
\task $\limite{1}{f(x)}$
\task $\limite{2}{f(x)}$
\end{tasks}
\end{exampleblock}

\begin{tasks}[label=\alph*)](2)
\task $\limite{-1}{f(x)}= 2^{-1}+1=\dfrac{3}{2}$
\task 
$\left. \begin{array}{l} \limite{0^{-}}{f(x)} = 2^0+1
=2 \\ \limite{0^{+}}{f(x)}= -3 \cdot 0 +2 = 2 
\end{array} \right\rbrace \Rightarrow$

$ \limite{0}{f(x)}=2 $
\task $\limite{1}{f(x)}=-3 \cdot 1 +2 = -1$

\task 
$\left. \begin{array}{l} \limite{2^{-}}{f(x)} = -3 \cdot 2 +2
=-4 \\ \limite{2^{+}}{f(x)}= \dfrac{-2 \cdot 2}{2-1} = -4 
\end{array} \right\rbrace \Rightarrow$

$ \limite{2}{f(x)}=-4 $
\end{tasks}
\end{frame}


\begin{frame}{Indeterminación $\frac{0}{0}$}
\begin{exampleblock}{Ejemplo}
Calcular los siguientes límites:
\begin{tasks}[label=\alph*)](2)
\task $\limite{-1}{\dfrac{x^3-2x^2+2x+5}{x^2-6x-7}}$
\task $\limite{2}{\dfrac{x^3-5x+2}{x^3+2x^2-3x-10}}$
\task $\limite{1}{\dfrac{\sqrt{x+3}-2}{x^2-1}}$
\task $\limite{2}{\dfrac{x-2}{3-\sqrt{x+7}}}$
\end{tasks}
\end{exampleblock}

\begin{tasks}[label=\alph* )](1)

\task $\limite{-1}{\dfrac{x^3-2x^2+2x+5}{x^2-6x-7}} = \dfrac{(-1)^3-2(-1)^2+2\cdot(-1)+5}{(-1)^2-6\cdot (-1)-7} = \left[\dfrac{0}{0}\right]_{\text{Indet.}}$

Descomponemos numerador y denominador y simplificamos:

$\limite{-1}{\dfrac{\bcancel{(x+1)}(x^2-3x+5)}{\bcancel{(x+1)}(x-7)}} = \dfrac{(-1)^2-3\cdot(-1)+5}{ (-1)-7}=-\dfrac{9}{8}$ 

\task $\limite{2}{\dfrac{x^3-5x+2}{x^3+2x^2-3x-10}}= \dfrac{2^3-5 \cdot 2+2}{2^3+2\cdot 2^2-3\cdot 2-10}= \left[\dfrac{0}{0}\right]_{\text{Indet.}} $

$\limite{-1}{\dfrac{\bcancel{(x-2)}(x^2+2x-1)}{\bcancel{(x-2)}(x^2+4x+5)}} = \dfrac{2^2+2\cdot2-1}{ 2^2+4\cdot 2 +5}=\dfrac{7}{17}$ 
\end{tasks}
\end{frame}

\begin{frame}{Indeterminación $\frac{0}{0}$}
\begin{tasks}[label=\alph*),resume](1)

\task $\limite{1}{\dfrac{\sqrt{x+3}-2}{x^2-1}} = \dfrac{\sqrt{1+3}-2}{1^2-1} = \left[\dfrac{0}{0}\right]_{\text{Indet.}} $

\vspace{10pt}
Multiplicamos por el conjugado y simplificamos.

\vspace{10pt}
$\limite{1}{\dfrac{(\sqrt{x+3}-2)(\sqrt{x+3}+2)}{(x^2-1)(\sqrt{x+3}+2)}} = \limite{1}{\dfrac{x+3-4}{(x^2-1)(\sqrt{x+3}+2)}} = $

\vspace{10pt}
$= \limite{1}{\dfrac{\bcancel{(x-1)}}{\bcancel{(x-1)}(x+1)(\sqrt{x+3}+2)}}= \dfrac{1}{2 \cdot 4}= \dfrac{1}{8}$

\task  $\limite{2}{\dfrac{x-2}{3-\sqrt{x+7}}} =  \dfrac{2-2}{3-\sqrt{2+7}} =  \left[\dfrac{0}{0}\right]_{\text{Indet.}}$

\vspace{10pt}
$\limite{2}{\dfrac{(x-2)(3+\sqrt{x+7})}{(3-\sqrt{x+7})(3+\sqrt{x+7})}}= \limite{2}{\dfrac{(x-2)(3+\sqrt{x+7})}{9-(x+7)}}= $

\vspace{10pt}
$= \limite{2}{\dfrac{\bcancel{(x-2)}(3+\sqrt{x+7})}{-\bcancel{(x-2)}}} = \dfrac{3+3}{-1}= -6 $

\end{tasks}

\end{frame}

\begin{frame}{Límite en un punto}
\begin{exampleblock}{Ejemplo}
Hallar el valor de $a$ para que este límite exista y calcularlo para ese valor de $a$.
\[ \limite{-1}{\dfrac{ax^3+3x^2-2x-3}{x^2+3x+2}} \]
\end{exampleblock}

Calculamos el valor del límite:

$\limite{-1}{\dfrac{ax^3+3x^2-2x-3}{x^2+3x+2}}= \dfrac{a\cdot (-1)^3+3\cdot (-1)^2-2\cdot (-1)-3}{(-1)^2+3\cdot (-1)+2}=\dfrac{-a+2}{0} $

Para que este límite exista el numerador tiene que ser $0$ para que sea indeterminado.

Por lo tanto, $-a+2=0 \Rightarrow a= 2$

Con este valor calculamos el límite.

$\limite{-1}{\dfrac{2x^3+3x^2-2x-3}{x^2+3x+2}}= \limite{-1}{\dfrac{\bcancel{(x+1)}(2x^2+x-3)}{\bcancel{(x+1)}(x+2)}}=$

\vspace{10pt}
$=\dfrac{2\cdot(-1)^2+(-1)-3}{(-1)+2}= -2$


\end{frame}

\begin{frame}{Límites en el infinito}
\begin{exampleblock}{Ejemplo}
Utilizando la gráfica calcular los limites cuando $x$ tiende a $+\infty$ y $-\infty$:
 \begin{tasks}[label=\alph*)](2)
\task 

\begin{tikzpicture}[scale=0.35]
\tkzInit[xmin=-2,xmax=3,ymin=-5,ymax=5]
\tkzGrid \tkzDrawXY
\tkzFct[domain=-2:2,color=blue]{1-x*x*x}
%\tkzDrawPoint[fill=blue,color=blue](1,1)
%\tkzFct[domain=1:2.5,color=blue]{-x*x+1}
%\tkzDrawPoint[fill=white,color=blue](1,0)
\end{tikzpicture}
\task

\begin{tikzpicture}[scale=0.25]
\tkzInit[xmin=-6,xmax=6,ymin=-6,ymax=8]
\tkzGrid \tkzDrawXY
\tkzFct[domain=-6:-0.1]{(x+2)/x)}
\tkzFct[domain=0.1:6]{(x+2)/x}
\tkzHLine[style=dashed,color=red]{1}
\end{tikzpicture}
\end{tasks}
\end{exampleblock}

\begin{tasks}[label=\alph*)](2)
\task $\limite{+\infty}{f(x)}=-\infty$

$\limite{-\infty}{f(x)}=+\infty$

\task $\limite{+\infty}{g(x)}=1$

$\limite{-\infty}{g(x)}=1$

\end{tasks}
\end{frame}

\begin{frame}{Límites en el infinito}
\begin{exampleblock}{Ejemplo}
Conocidos los siguientes límites:
\[ \limite{+\infty}{f(x)}=1 \quad \limite{-\infty}{f(x)}=-\infty \quad \limite{+\infty}{g(x)}=+\infty \quad \limite{-\infty}{g(x)}=-1 \]
calcular los siguientes límites:
\begin{tasks}[label=\alph*)](2)
\task $\limite{+\infty}{\left[ f(x)+g(x) \right] }$
\task $\limite{-\infty}{\left[ f(x)-g(x) \right] }$
\task $\limite{-\infty}{\left[ f(x)\cdot g(x) \right] }$
\task $\limite{+\infty}{\sqrt{ f(x)} }$
\task $\limite{+\infty}{\dfrac{f(x)}{g(x)} }$
\task $\limite{-\infty}{\dfrac{f(x)}{g(x)} }$
\task $\limite{+\infty}{\left[ g(x) \right]^3 }$
\task $\limite{+\infty}{\left[ g(x) \right]^{-4} }$
\task $\limite{+\infty}{\log f(x) }$
\task $\limite{-\infty}{\left[ f(x) \right]^{g(x)} }$
\end{tasks}
\end{exampleblock}
\end{frame}

\begin{frame}{Límites en el infinito}
\begin{tasks}[label=\alph*)](2)
\task* $\limite{+\infty}{\left[ f(x)+g(x) \right] } = 1+\infty= +\infty$
\task* $\limite{-\infty}{\left[ f(x)-g(x) \right] } = -\infty+1=-\infty$
\task* $\limite{-\infty}{\left[ f(x)\cdot g(x) \right] }=(-\infty)(-1)=+\infty $
\task* $\limite{+\infty}{\sqrt{ f(x)} }= \sqrt{1}=1 $
\task $\limite{+\infty}{\dfrac{f(x)}{g(x)} }= \dfrac{1}{+\infty}=0$
\task $\limite{-\infty}{\dfrac{f(x)}{g(x)} }=\dfrac{-\infty}{-1}=+\infty$
\task $\limite{+\infty}{\left[ g(x) \right]^3 }= (+\infty)^3=+\infty$
\task $\limite{+\infty}{\left[ g(x) \right]^{-4} }=(+\infty)^{-4} = 0$
\task $\limite{+\infty}{\log f(x) }= \log(1)=0$
\task $\limite{-\infty}{\left[ f(x) \right]^{g(x)} } =(-\infty)^{-1}=0$
\end{tasks}

\end{frame}
\begin{frame}
\begin{exampleblock}{Ejemplo}
Calcular los siguientes límites:
\begin{tasks}[label=\alph*)](2)
\task $\limite{\infty}{\dfrac{x-1}{x^2+3} }$
\task $\limite{\infty}{\dfrac{x^3-x+1}{5x^2+3x-2} }$
\task $\limite{\infty}{\dfrac{2x^3+3x^2-1}{4x^3-x+3} }$
\task $\limite{\infty}{\dfrac{\sqrt{2x^3+x-2}}{x^2+3} }$
\task $\limite{\infty}{\dfrac{\sqrt{4x^2+x-2}+x}{5x+3} }$
\task $\limite{\infty}{\dfrac{\sqrt[3]{x^3+1}}{\sqrt{4x^2+3}} }$
\end{tasks}
\end{exampleblock}

Para calcular estos límites comparamos los grados del numerador y denominador.

\begin{tasks}[label=\alph*)](2)
\task $\limite{\infty}{\dfrac{x-1}{x^2+3} }= 0$
\task $\limite{\infty}{\dfrac{x^3-x+1}{5x^2+3x-2} }= \infty$
\task $\limite{\infty}{\dfrac{2x^3+3x^2-1}{4x^3-x+3} }= \dfrac{2}{4}=\dfrac{1}{2}$
\task $\limite{\infty}{\dfrac{\sqrt{2x^3+x-2}}{x^2+3} }= 0$
\task* $\limite{\infty}{\dfrac{\sqrt{4x^2+x-2}+x}{5x+3} }= \dfrac{2+1}{5}=\dfrac{3}{5}$
\task* $\limite{\infty}{\dfrac{\sqrt[3]{x^3+1}}{\sqrt{4x^2+3}} }= \dfrac{1}{2}$
\end{tasks}
\end{frame}

\begin{frame}{Límites en el infinito}
\begin{exampleblock}{Ejemplo}
Calcular los siguientes límites:
\begin{tasks}[label=\alph*)](2)
\task $\limite{\infty}{\sqrt{2x^3+x-2}-x}$
\task $\limite{\infty}{\sqrt{2x^3+x-2}-x^2}$
\task $\limite{\infty}{\sqrt{x^2+x-2}-x}$
\task $\limite{\infty}{\sqrt{4x^2+x-2}-\sqrt{x^2+1}}$
\end{tasks}
\end{exampleblock}

\begin{tasks}[label=\alph*)](1)
\task $\limite{\infty}{\sqrt{2x^3+x-2}-x} = +\infty$
\task $\limite{\infty}{\sqrt{2x^3+x-2}-x^2}= - \infty$
\task $\limite{\infty}{\sqrt{x^2+x-2}-x}= \infty - \infty (Indet.)$

\vspace{10pt}
$\limite{\infty}{\dfrac{(\sqrt{x^2+x-2}-x)(\sqrt{x^2+x-2}+x)}{\sqrt{x^2+x-2}+x}}= 
\limite{\infty}{\dfrac{x^2+x-2-x^2}{\sqrt{x^2+x-2}+x}}= $

\vspace{10pt}
$=\limite{\infty}{\dfrac{x-2}{\sqrt{x^2+x-2}+x}}=\dfrac{1}{1+1}=\dfrac{1}{2}$

\end{tasks}

\end{frame}

\begin{frame}{Límites en el infinito}
\begin{tasks}[label=\alph*),resume](1)
\task $\limite{\infty}{\sqrt{4x^2+x-2}-\sqrt{x^2+1}}=\infty - \infty (Indet.)$

\vspace{10pt}
 $\limite{\infty}{\dfrac{(\sqrt{4x^2+x-2}-\sqrt{x^2+1})(\sqrt{4x^2+x-2}+\sqrt{x^2+1})}{\sqrt{4x^2+x-2}+\sqrt{x^2+1}}}=$
 
 \vspace{10pt}
$\limite{\infty}{\dfrac{(4x^2+x-2)-(x^2+1)}{\sqrt{4x^2+x-2}+\sqrt{x^2+1}}}= \limite{\infty}{\dfrac{x-3}{\sqrt{4x^2+x-2}+\sqrt{x^2+1}}}= \dfrac{1}{3}$
\end{tasks}
\end{frame}

\begin{frame}{Límites en el infinito}
\begin{exampleblock}{Ejemplo}
Calcular los siguientes límites:
\begin{tasks}[label=\alph*)](2)
\task $\limite{\infty}{\left( \dfrac{2x^2+x-2}{x^2+1}\right)^{3+x}}$
\task $\limite{\infty}{\left( \dfrac{x^2+4x-2}{2x^2+1}\right)^{\frac{x^2+1}{x}}}$
\task $\limite{\infty}{\left( \dfrac{2x-2}{2x+1}\right)^{3+x}}$
\task $\limite{\infty}{\left( \dfrac{2x^2+x-2}{2x^2+1}\right)^{\frac{x^2-1}{2x}}}$
\end{tasks}
\end{exampleblock}

\begin{tasks}[label=\alph*)](1)
\task $\limite{\infty}{\left( \dfrac{2x^2+x-2}{x^2+1}\right)^{3+x}}=2^{\infty}=\infty$
\task $\limite{\infty}{\left( \dfrac{x^2+4x-2}{2x^2+1}\right)^{\frac{x^2+1}{x}}}=\left( \dfrac{1}{2}\right)^{\infty}=0$
\task $\limite{\infty}{\left( \dfrac{2x-2}{2x+1}\right)^{3+x}}$
\task $\limite{\infty}{\left( \dfrac{2x^2+x-2}{2x^2+1}\right)^{\frac{x^2-1}{2x}}}$
\end{tasks}
\end{frame}

\begin{frame}{Límites en el infinito}

\begin{tasks}[label=\alph*),resume](1)

\task $\limite{\infty}{\left( \dfrac{2x-2}{2x+1}\right)^{3+x}}=1^{\infty} (Indet.)$

\vspace{10pt}

Aplicamos la fórmula:

\vspace{10pt}

$\limite{\infty}{\left( \dfrac{2x-2}{2x+1}-1\right)\cdot(3+x)}= 
\limite{\infty}{\left( \dfrac{2x-2-2x-1}{2x+1}\right)\cdot(3+x)}= $

\vspace{10pt}

$= \limite{\infty}{\left( \dfrac{-3}{2x+1}\right)\cdot(3+x)}=
\limite{\infty}{\left( \dfrac{-9-3x}{2x+1}\right)}= -\dfrac{3}{2}$

\vspace{10pt}

Luego el límite es: 

\vspace{10pt}
$\limite{\infty}{\left( \dfrac{2x-2}{2x+1}\right)^{3+x}}= e^{-\frac{3}{2}}$


\end{tasks}
\end{frame}

\begin{frame}{Límites en el infinito}
\begin{tasks}[label=\alph*),resume](1)

\task $\limite{\infty}{\left( \dfrac{2x^2+x-2}{2x^2+1}\right)^{\frac{x^2-1}{2x}}}=1^{\infty} (Indet.)$

\vspace{10pt}


$\limite{\infty}{\left( \dfrac{2x^2+x-2}{2x^2+1}-1\right)\cdot \left( \frac{x^2-1}{2x} \right)}= $
\vspace{10pt}


$\limite{\infty}{\left( \dfrac{2x^2+x-2-2x^2-1}{2x^2+1}\right)\cdot\left( \frac{x^2-1}{2x} \right)}= $

\vspace{10pt}

$= \limite{\infty}{\left( \dfrac{x-3}{2x^2+1}\right)\cdot\left( \frac{x^2-1}{2x} \right)}=
\limite{\infty}{\left( \dfrac{x^3-3x^2-x+3}{2x^3-2x}\right)}= $

\vspace{10pt}

$=\dfrac{1}{2}$

\vspace{10pt}

Luego el límite es: 

\vspace{10pt}
$\limite{\infty}{\left( \dfrac{2x-2}{2x+1}\right)^{3+x}}= e^{\frac{1}{2}}$


\end{tasks}
\end{frame}

\begin{frame}{Límites en el infinito}
\begin{exampleblock}{Ejemplo}
Calcular los siguientes límites:
\begin{tasks}[label=\alph*)](2)
\task $\limite{\infty}{\dfrac{2^x+10}{3^{x+1}}}$
\task $\limite{\infty}{\dfrac{3^x+10}{3^{x+1}}}$
\task $\limite{\infty}{\dfrac{3^{x+10}}{3^{x+1}}}$
\task $\limite{\infty}{\dfrac{3^{x+1}}{3^x+10}}$

\end{tasks}
\end{exampleblock}

\begin{tasks}[label=\alph*)](1)
\task $\limite{\infty}{\dfrac{2^x+10}{3^{x+1}}}= 0$
\task $\limite{\infty}{\dfrac{3^x+10}{3^{x+1}}}= \limite{\infty}{\dfrac{3^x+10}{3^x\cdot 3^1}}= \dfrac{1}{3} $
\task $\limite{\infty}{\dfrac{3^{x+10}}{3^{x+1}}}= \limite{\infty}{\dfrac{3^x\cdot 3^{10}}{3^x\cdot 3^1}}= \dfrac{3^{10}}{3^1}= 3^9$
\task $\limite{\infty}{\dfrac{3^{x+1}}{3^x+10}}= \limite{\infty}{\dfrac{3^x\cdot 3^1}{3^x+10}}= 3$

\end{tasks}

\end{frame}
\end{document}