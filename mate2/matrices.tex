\documentclass[9pt]{beamer}
\usetheme{metropolis}
%\usetheme{Warsaw}
\metroset{titleformat=smallcaps,block=fill}
\usecolortheme{seahorse}
%\usepackage[utf8]{inputenc}
\usepackage[spanish]{babel}
\usepackage{amsmath}
\usepackage{amsfonts}
\usepackage{amssymb}
\usepackage{graphicx}
\usepackage{tikz} 
\usetikzlibrary{matrix}
\usetikzlibrary{calendar,decorations.markings} 
\usetikzlibrary{shapes,positioning}

\usepackage{tkz-tab,tkz-euclide,tkz-fct}

\usetkzobj{all} 
\usepackage{polynom}

\newcommand{\R}{\mathbb{R}}

\author{Departamento de Matemáticas}
\title{Matrices}
%\subtitle{Definición y cálculo}
%\setbeamercovered{transparent} 
%\setbeamertemplate{navigation symbols}{} 
%\logo{\includegraphics[scale=0.05]{../../images/logoa.jpg}} 
%\institute{UHEI - IVED} 
\date{\includegraphics[scale=0.15]{imagenes/logoa.jpg}} 
%\subject{} 
\begin{document}

\begin{frame}
\titlepage
\end{frame}

\begin{frame}
\tableofcontents
\end{frame}

\section{Definiciones. Tipos de matrices}

\subsection{Definiciones}

\begin{frame}{Definición}
\begin{alertblock}{Definición}
Una {\bf matriz de dimensión $m \times n$}  es un conjunto de números ordenados en m filas y n columnas de la forma:
 \[ A=\begin{pmatrix} 
 a_{11} & a_{12} & \cdots & a_{1n} \\
 a_{21} & a_{22} & \cdots & a_{2n} \\
  \vdots  & \vdots  & \cdots & \vdots \\
 a_{m1} & a_{m2} & \cdots & a_{mn} \\
\end{pmatrix}
\]
\end{alertblock}
\pause

Cada elemento genérico de la matriz se designa por $a_{ij}$, donde $i$ es el número de fila que ocupa y $j$ es el número de fila.

\pause

Dos matrices de la misma dimensión $A_{mxn}=(a_{ij})$ y $B_{mxn}=(b_{ij})$ son iguales si $a_{ij}=b_{ij}\quad \forall i,j$
\end{frame}

\begin{frame}{Ejemplos}
\begin{exampleblock}{Ejemplo}

Dada la matriz
$A= \begin{pmatrix}  -1 & 1  \\  2 & -3 \\  3 & -5 \\ \end{pmatrix}$ , escribir su dimensión y los elementos $a_{12}$, $a_{22}$ y $a_{31}$

\end{exampleblock}

\pause

La matriz tiene 3 filas y  2 columnas, luego su dimensión es $3x2$.

\pause 

$a_{12} = 1$, $a_{22}= -3$ y $a_{31}=3$

\end{frame}

\begin{frame}{Ejemplos}
\begin{exampleblock}{Ejemplo}

Hallar los valores de $x$ e $y$ para que las matrices $ \begin{pmatrix}  x & 1 \\  2 & y \\ \end{pmatrix}$, $ \begin{pmatrix}  3 & x-2 \\  y & x-1 \\ \end{pmatrix}$ sean iguales.


\end{exampleblock}

 \pause
 Para que sean iguales cada elemento de la primera matriz tiene que ser igual a su correspondiente de la segunda.
\pause 
 
 Por lo tanto, $\begin{cases} x= 3 \cr 1= x-2 \cr 2= y \cr y=x-1 \end{cases}$
 
 \pause
 Resolviendo tenemos que $x=3 $ e $ y=2$.

\end{frame}

 \subsection{Tipos de matrices}
\begin{frame}{Tipos de matrices}

\begin{itemize}[<+-| alert@+>]
	\item  {\bf Matriz fila}: es la matriz que solo tiene una fila $A_{1n}$.
 
  		$A= \begin{pmatrix}  -1 & 1  & 2  \\ \end{pmatrix}$
  
    \item  {\bf Matriz columna}: es la matriz que solo tiene una fila $A_{m1}$
   
  		$A= \begin{pmatrix}  -1 \\  1  \\  2  \\  -3   \\\end{pmatrix}$

  
   \item  {\bf Matriz nula}: es la matriz en la que todos sus elementos son nulos.
 
  $A= \begin{pmatrix}  0 & 0  \\  0  & 0 \\   0 & 0 \\\end{pmatrix}$
\end{itemize}
\end{frame}

\begin{frame}{Tipos de matrices}


\begin{itemize}[<+-| alert@+>] 
 	\item  {\bf Matriz rectangular}: es la matriz que tiene distinto número de filas que de columnas.
 
  		$A= \begin{pmatrix}  -1 & 1  & 2  \\  -3  & 3 & -5 \\\end{pmatrix}$

	\item {\bf Matriz cuadrada}: es la matriz que tiene igual número de filas que de columnas (m=n). En este caso podemos decir que la matriz es de orden n. 

  	$B= \begin{pmatrix}  -1 & 1  & 2  \\  -3  & 3 & -5 \\  2  & 1 & 6 \\\end{pmatrix}$

\pause  
  	En una matriz cuadrada definimos:
  \begin{itemize}[<+->]
  \item {\bf Diagonal principal}: Son los elementos cuyo número de fila coincide con el de columna: $a_{ii}$
  \item {\bf Diagonal secundaria}: Son todos los elementos tales que $i+j=n+1$.
  \end{itemize}
  
\end{itemize}
\pause
\begin{columns}
\column{0.5\textwidth}
\begin{center}
\begin{tikzpicture}[baseline=(A.center)]
\tikzset{BarreStyle/.style =   {opacity=.4,line width=4 mm,line cap=round,color=red}}
  \matrix (A) [matrix of math nodes,ampersand replacement=\&,column sep=0 mm,left delimiter={(},right delimiter={)}] {a_{11} \& a_{12} \& \cdots \& a_{1n}  \\
  a_{21} \& a_{22}  \& \cdots \& a_{2n}  \\
  \cdots \& \cdots \& \vdots \& \vdots \\
  a_{n1} \& a_{n2}  \& \cdots \& a_{nn}  \\ };
\draw [BarreStyle] (A-1-1.north west)  to (A-4-4.south east) ;
\end{tikzpicture}

Diagonal principal
\pause
\end{center}
\column{0.5\textwidth}
\begin{center}
\begin{tikzpicture}[baseline=(A.center)]
\tikzset{BarreStyle/.style =   {opacity=.4,line width=4 mm,line cap=round,color=blue}}
  \matrix (A) [matrix of math nodes,ampersand replacement=\&,column sep=0 mm,left delimiter={(},right delimiter={)}] {a_{11} \& a_{12} \& \cdots \& a_{1n}  \\
  a_{21} \& a_{22}  \& \cdots \& a_{2n}  \\
  \cdots \& \cdots \& \vdots \& \vdots \\
  a_{n1} \& a_{n2}  \& \cdots \& a_{nn}  \\ };
\draw [BarreStyle] (A-1-4.north east)  to (A-4-1.south west) ;
\end{tikzpicture}

Diagonal secundaria
\end{center}
\end{columns}
\end{frame}

\begin{frame}{Matrices cuadradas}

Dentro de las matrices cuadradas podemos distinguir las siguientes:

\pause
 \textbf{Matrices triangular superior:} Son las que todos sus elementos por debajo de la diagonal principal son cero.

\[ \begin{pmatrix} 
1 & -1 & 3 & 0  \\
  \color{red}0 & 3  & 1 & -1  \\
 \color{red}0 & \color{red}0 & 2 & 1 \\
  \color{red}0 & \color{red}0  & \color{red}0 & -4  \\ 
\end{pmatrix} \]


\pause
 \textbf{Matrices triangular inferior:} Son las que todos sus elementos por encima de la diagonal principal son cero.

\[
  \begin{pmatrix}1 & \color{red}0 & \color{red}0 & \color{red}0  \\
  -1 & 3  & \color{red}0 & \color{red}0  \\
  1 & 2 & -2 & \color{red}0 \\
 4 & 0  & 6 & -2  \\ 
\end{pmatrix} \]

\end{frame}

\begin{frame}


 \textbf{Matrices diagonal:} Son las que todos sus elementos fuera de la diagonal principal son cero.

\[
  \begin{pmatrix} 1 & \color{red}0 & \color{red}0 & \color{red}0  \\
  \color{red}0 & 3  & \color{red}0 & \color{red}0  \\
  \color{red}0 & \color{red}0 & -2 & \color{red}0 \\
 \color{red}0 & \color{red}0  & \color{red}0 & -2  \\ 
\end{pmatrix}
\] 

\pause
 \textbf{Matrices escalar:} Son las que todos sus elementos  de la diagonal principal son iguales y el resto cero.

\[
  \begin{pmatrix} -2 & \color{red}0 & \color{red}0 & \color{red}0  \\
  \color{red}0 & -2  & \color{red}0 & \color{red}0  \\
  \color{red}0 & \color{red}0 & -2 & \color{red}0 \\
 \color{red}0 & \color{red}0  & \color{red}0 & -2  \\ 
\end{pmatrix}
\] 


\end{frame}

\begin{frame}{Matrices cuadradas}
\textbf{Matrices identidad:} Son las que todos sus elementos  de la diagonal principal son 1 y el resto cero.

\[
  \begin{pmatrix} 1 & \color{red}0 & \color{red}0 & \color{red}0  \\
  \color{red}0 & 1  & \color{red}0 & \color{red}0  \\
  \color{red}0 & \color{red}0 & 1 & \color{red}0 \\
 \color{red}0 & \color{red}0  & \color{red}0 & 1  \\ 
\end{pmatrix}
\] 

\pause
La matriz identidad se llama $I_n$, siendo $n$ el orden de la matriz.

\[ I_2= \begin{pmatrix}
1 & 0 \\ 
0 & 1 
\end{pmatrix} \qquad 
I_3=\begin{pmatrix} 
1 & 0 & 0 \\
0 & 1 & 0 \\
0 & 0 & 1 

\end{pmatrix}
\]
\end{frame}

\section{Operaciones con matrices}

\subsection{Suma}

\begin{frame}{Operaciones con matrices}
\begin{alertblock}{Suma de matrices}
Dadas dos matrices de la misma dimensión $A=(a_{ij})$ y $B=(b_{ij})$ llamamos suma a otra matriz $C=(c_{ij})$ definida por 
$$(c_{ij})=(a_{ij})+(b_{ij})=(a_{ij}+b_{ij})$$
\end{alertblock}
\pause
Y cumple las siguientes propiedades

	\begin{enumerate}[<+-| alert@+>] 
		\item Asociativa: $A+(B+C)=(A+B)+C$
		\item Conmutativa: $A+B=B+A$
		\item Elemento neutro: $A+0=A$
		\item Elemento inverso: $A-A=0$
	\end{enumerate}
	
\end{frame}

\begin{frame}{Operaciones con matrices}
\begin{exampleblock}{Ejemplo suma}
Dadas las matrices $A=\begin{pmatrix}
	2 & 1 \\
	1 & 1 \\
	3 & 2
\end{pmatrix}$ y $ B=
     \begin{pmatrix}
	1 & 0 \\
	-1 & -2 \\
	2 & 3
\end{pmatrix}    $, hallar su suma.
\end{exampleblock}
\pause
\[ A+B=   \begin{pmatrix}
	2 & 1 \\
	1 & 1 \\
	3 & 2
\end{pmatrix} 
     +
     \begin{pmatrix}
	1 & 0 \\
	-1 & -2 \\
	2 & 3
\end{pmatrix}    = 
\pause
\begin{pmatrix}
	3 & 1 \\
	0 &-1 \\
	5 & 5
\end{pmatrix}    \]


\end{frame}

\subsection{Producto por un número real}

\begin{frame}{Operaciones con matrices}


\begin{alertblock}{Producto por un número real}
Dados un número real $k\in \R$ y una matriz $A=(a_{ij})$ se define el producto de $k\cdot A$ de la siguiente manera:
\[ k\cdot A=k\cdot (a_{ij})= (k\cdot a_{ij})\]
\end{alertblock}
\pause
El producto de una matriz por un número real tiene las siguientes propiedades:

\end{frame}

\begin{frame}{Operaciones con matrices}

\begin{exampleblock}{Ejemplo producto por un número real}
Dada la matriz $A=\begin{pmatrix}
	2 & 1 \\
	1 & 1 \\
	3 & 2
\end{pmatrix}$, calcular $5A$.
\end{exampleblock}
\pause

 \[ 5A=5\cdot\begin{pmatrix}
	2 & 1 \\
	1 & 1 \\
	3 & 2
\end{pmatrix}= \begin{pmatrix}
	10 & 5 \\
	5 & 5 \\
	15 & 10
\end{pmatrix}  \]
\end{frame}

\subsection{Producto de matrices}
\begin{frame}{Operaciones con matrices}
\begin{alertblock}{Producto de matrices}
Dos matrices $A$ y $B$ se pueden multiplicar si el número de columnas de $A$ es igual al número de filas de $B$, en tal caso, se define la multiplicación de la siguiente manera
\[ {A \atop {m\ast n}}\cdot {B \atop {n\ast p}} \]
\[ \underbrace{A}_{m \ast n} \cdot \underbrace{B}_{n\ast p}=\underbrace{C}_{m\ast p} \]
Para calcular el elemento $c_{ij}$ hay que utilizar la fila $i$ de $A$ y la columna $j$ de B.
\[ \begin{pmatrix}
	a_{i1} & a_{i2}&a_{i3}& \cdots &a_{in} \\
\end{pmatrix}\cdot \begin{pmatrix}
	a_{j1} \\ a_{j2}\\a_{j3}\\ \cdots \\a_{jn} \\
\end{pmatrix}=a_{i1}\cdot a_{j1}+ a_{i2} \cdot a_{2j} + \cdots +a_{in}\cdot a_{nj}\]
\end{alertblock}
\end{frame}

\begin{frame}{Operaciones con matrices}

Propiedades del producto de matrices:

\pause
\begin{itemize}[<+- |alert@+>]
\item \textbf{Asociativa:} $A \cdot(B \cdot C)= (A \cdot B) \cdot C$
\item \textbf{Distributiva respecto a la suma por la izquierda:} $A \cdot (B + C)= A \cdot B + A \cdot C$
\item \textbf{Distributiva respecto a la suma por la derecha:} $(A  +B )\cdot  C)= A \cdot C + B \cdot C$
\item \textbf{Elemento neutro:} $A \cdot I = A \qquad I \cdot A = A$
\end{itemize}

\onslide*<6>{\textbf{{\Large El producto de matrices no es conmutativo $A \cdot B \neq B\cdot A$}}}

\end{frame}
\begin{frame}{Operaciones con matrices}
\begin{exampleblock}{Ejemplo producto matrices}
Dadas las matrices $A=\begin{pmatrix}
	2 & 1 \\
	1 & 1 \\
	3 & 2
\end{pmatrix}$ y $B= \begin{pmatrix}
	10 & 5 &5\\
	5 & 5 &-2
\end{pmatrix} $, calcular $A\cdot B$ y $B \cdot A$, si se pueden.
\end{exampleblock}
\onslide*<2-5>{
El número de columnas de $A$ es $2$ y el número de filas de $B$ es $2$, por lo tanto el producto $A \cdot B$ se puede realizar.}


\onslide*<3-5>{
\[  A\cdot B \]

$ \begin{pmatrix}
	2 & 1 \\
	1 & 1 \\
	3 & 2
\end{pmatrix}\cdot \begin{pmatrix}
	10 & 5 &5\\
	5 & 5 &-2
\end{pmatrix} $}
 \onslide*<4>{
$= \begin{pmatrix}
	2\cdot 10 + 1 \cdot 5 & 2\cdot 5 + 1 \cdot 5 & 2\cdot 5 + 1 \cdot (-2)\\
	1\cdot 10 + 1 \cdot 5 & 1\cdot 5 + 1 \cdot 5 & 1\cdot 5 + 1 \cdot (-2)\\
	3\cdot 10 + 2 \cdot 5 & 3\cdot 5 + 2 \cdot 5 & 3\cdot 5 + 2 \cdot (-2)
\end{pmatrix}$  }
\onslide*<5>{
$ = \begin{pmatrix}
	25 & 15& 8 \\
	15 & 10 & 3\\
	40 & 25 & 11
\end{pmatrix}  $\\}

\onslide*<6-9>{
El número de columnas de $B$ es $3$ y el número de filas de $A$ es $3$, por lo tanto el producto $B \cdot A $ se puede realizar.}


\onslide*<7-9>{
\[  B\cdot A \]
$ \begin{pmatrix}
    10 & 5 &5\\
	5 & 5 &-2
\end{pmatrix}\cdot \begin{pmatrix}
	2 & 1 \\
	1 & 1 \\
	3 & 2
\end{pmatrix} $}
 \onslide*<8>{
$= \begin{pmatrix}
	10\cdot 2 + 5 \cdot 1 + 5\cdot 3 & 10 \cdot 1 + 5\cdot 1 + 5 \cdot 2\\
	5\cdot 2 + 5 \cdot (-2) + 5\cdot 3 & 5 \cdot 1 + 5\cdot 1 + (-2) \cdot 2\\
\end{pmatrix}$  }
\onslide*<9>{
$ = \begin{pmatrix}
	40 & 25 \\
	15 & 6 \\
\end{pmatrix}  $}

\onslide*<10>{
$A\cdot B  = \begin{pmatrix}
	25 & 15& 8 \\
	15 & 10 & 3\\
	40 & 25 & 11
\end{pmatrix}  $

$B \cdot A =  \begin{pmatrix}
	40 & 25 \\
	15 & 6 \\
\end{pmatrix}  $}


\end{frame}

\begin{frame}{Producto de matrices}

En el producto de matrices también hay que tener en cuenta lo siguiente:
\begin{itemize}[<+-|alert@+>]

\item Si $A \cdot B = A \cdot C \Rightarrow B$ no tiene necesariamente que ser igual a $C$.
\item Si $A \cdot B = O$ puede ser que tanto $A$ como $B$ sean distintas de la matriz $O$
\end{itemize}
\end{frame}

\begin{frame}{Ejemplos}

\begin{exampleblock}{Ejemplo}
Dadas las matrices 
\[A=\begin{pmatrix}
	2&2&-1 \\
	-1&-1&1 \\
	-1&-2&2
\end{pmatrix}
\qquad
I=\begin{pmatrix}
	1&0&0 \\
	0&1&0 \\
	0&0&1
\end{pmatrix} \]

se pide:

 Calcular la matriz $(A-I)^2$
\end{exampleblock}

\pause
 Primero calculamos $A-I=\begin{pmatrix}
	2&2&-1 \\
	-1&-1&1 \\
	-1&-2&2
\end{pmatrix}
-\begin{pmatrix}
	1&0&0 \\
	0&1&0 \\
	0&0&1
\end{pmatrix}=
\begin{pmatrix}
	1 & 2 & -1 \\
	-1&-2&1 		\\
	-1&-2 & 1
\end{pmatrix} $ 

\pause
Ahora calculamos $(A-I)^2=\begin{pmatrix}
	1 & 2 & -1 \\
	-1&-2&1 		\\
	-1&-2 & 1
\end{pmatrix} \cdot\begin{pmatrix}
	1 & 2 & -1 \\
	-1&-2&1 		\\
	-1&-2 & 1
\end{pmatrix} =
\begin{pmatrix}
	0 & 0 & 0 \\
	0&0&0 		\\
	0&0 & 0
\end{pmatrix} $


\end{frame}






\begin{frame}{Ejemplos}

\begin{exampleblock}{Ejemplo}
Sean las matrices 
$A=\begin{pmatrix}
	2 & 1 \\
	1 & 1
\end{pmatrix}$, 
$B=\begin{pmatrix}
 	1 & x \\ 
 	x & 0
\end{pmatrix}$ y  
$C=\begin{pmatrix}
	0 & -1 \\
	-1 & 2
\end{pmatrix}
$

\begin{enumerate}
\item Encuentre el valor o los valores de $x$ de forma que $B^2=A$
\item Determine $x$ para que $A+B+C=3I_2$
\end{enumerate}

\end{exampleblock}

\begin{enumerate}
\item
\onslide*<2-4>{
 Hallamos $B^2$. 
\[ \begin{pmatrix}
	1 & x \\
	x & 0 
\end{pmatrix} \cdot 
\begin{pmatrix}
	1 & x \\
	x & 0 
\end{pmatrix}=
\begin{pmatrix}
	x^2+1 & x \\
	x & x^2
\end{pmatrix}
\]
}
\onslide*<3-4>{
Igualamos las dos matrices : 
\[ \begin{pmatrix}
	x^2+1 & x \\
	x & x^2
\end{pmatrix} =
\begin{pmatrix}
	2 & 1 \\
	1 & 1
\end{pmatrix} \]
}
\onslide*<4>{
De aquí obtenemos el siguiente sistema de ecuaciones:
\[
\left\lbrace
\begin{array}{l}
x^2+1=2 \\
x=1 \\
x=1 \\
x^2=1
\end{array}
\right. \Rightarrow x=1 \]
}
\item
\onslide*<5-6>{
 Hallamos $A+B+C$

\[ 
\begin{pmatrix}
	2 & 1 \\
	1 & 1
\end{pmatrix} +  
\begin{pmatrix}
 	1 & x \\ 
 	x & 0
\end{pmatrix}
+\begin{pmatrix}
	0 & -1 \\
	-1 & 2
\end{pmatrix}=
\begin{pmatrix}
	3 & x \\
	x & 3
\end{pmatrix}
\]}

\onslide*<6>{

Igualando esta matriz a $3I_2$ obtenemos:
\[
\begin{pmatrix}
	3 & x \\
	x & 3
\end{pmatrix}=
\begin{pmatrix}
	3 & 0 \\
	0 & 3
\end{pmatrix}
 \Rightarrow \left\lbrace
\begin{array}{l}
3=3 \\
x=0 \\
x=0 \\
3=3
\end{array}
\right. \Rightarrow x=0 \]
}
\end{enumerate}


\end{frame}

\section{Trasposición de matrices}
\subsection{Matriz traspuesta}
\begin{frame}{Matriz traspuesta}
\begin{alertblock}{Matriz traspuesta}
Se llama \textbf{matriz traspuesta} de una matriz $A$ de dimensión $m \times n$ a la matriz que se obtiene de cambiar en $A$ las filas por columnas o las columnas por filas. Se representa por $A^t$ y su dimensión es $n \times m$. Si la matriz es cuadrada el orden de su traspuesta es el mismo.
\end{alertblock}

\pause
Las propiedades de la trasposición de matrices son las siguientes:
\begin{itemize}[<+- |alert@+>]
\item $\left( A^t \right)^t = A$
\item $(A+B)^t = A^t + B^t$
\item $(k\cdot A)^t = k  \cdot A^t $ siendo $ k \in \R$
\item $(A \cdot B)^t = B^t \cdot A^t$
\end{itemize}

\end{frame}

\begin{frame}{Matriz traspuesta}
\begin{exampleblock}{Ejemplo}
Dada la matriz $A=\begin{pmatrix} 1 & 2 & -1 \\ 0 & 2 & 3 \\ 1 & -2 & -1 \end{pmatrix}$, hallar su traspuesta.
\end{exampleblock}

Para hallar la traspuesta cambiamos las filas por columnas, por lo tanto, la matriz traspuesta de $A$ será:

$A^t = \begin{pmatrix} 1 & 0 & 1 \\ 2 & 2 & -2 \\ -1 & 3 & -1 \end{pmatrix}$

\end{frame}
\subsection{Matriz simétrica y antisimétrica}

\begin{frame}{Matriz simétrica}
\begin{alertblock}{Matriz simétrica}
Una matriz es simétrica si se cumple que $A=A^t$
\end{alertblock}

\begin{exampleblock}{Ejemplo}
Dada la matriz $A=\begin{pmatrix} 1 & 2 & -1 \\ 2 & -2 & 3 \\ -1 & 3 & 0 \end{pmatrix}$, comprobar que es simétrica.
\end{exampleblock}

Hallamos  la traspuesta de $A$:

$A^t = \begin{pmatrix} 1 & 2 & -1 \\ 2 & -2 & 3 \\ -1 & 3 & 0 \end{pmatrix}$

Observamos que $A=A^t$, por lo tanto la matriz $A$ es simétrica.
\end{frame}

\begin{frame}{Matriz antisimétrica}
\begin{alertblock}{Matriz antisimétrica}
Una matriz es antisimétrica si se cumple que $A=-A^t$
\end{alertblock}

\begin{exampleblock}{Ejemplo}
Dada la matriz $A=\begin{pmatrix} 0 & 2 & -1 \\ -2 & 0 & 3 \\ 1 & -3 & 0 \end{pmatrix}$, comprobar que es antisimétrica.
\end{exampleblock}

Hallamos  la traspuesta de $A$:

$A^t = \begin{pmatrix} 0 & -2 & 1 \\ 2 & 0 & -3 \\ -1 & 3 & 0 \end{pmatrix}$

Observamos que $A=-A^t$, por lo tanto la matriz $A$ es antisimétrica.
\end{frame}

\end{document}

































 

 
 
  
 
