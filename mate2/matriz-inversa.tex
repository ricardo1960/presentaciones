\documentclass[9pt]{beamer}
\usetheme{metropolis}
%\usetheme{Warsaw}
\metroset{titleformat=smallcaps,block=fill}
\usecolortheme{seahorse}
%\usepackage[utf8]{inputenc}
\usepackage[spanish]{babel}
\usepackage{amsmath}
\usepackage{amsfonts}
\usepackage{amssymb}
\usepackage{graphicx}
\usepackage{tikz} 
\usetikzlibrary{matrix}
\usetikzlibrary{calendar,decorations.markings} 
\usetikzlibrary{shapes,positioning}

\usepackage{tkz-tab,tkz-euclide,tkz-fct}
\usetkzobj{all}

\usepackage{tcolorbox}
\usepackage{varwidth}
\newtcbox{\resultado}[1][center]{#1,colback=red!5!white,
colframe=red!75!black,varwidth upper}

\newtcbox{\resaltado}{colback=blue!5!white,
colframe=blue!75!black,varwidth upper}

\usepackage{polynom}

\newcommand{\R}{\mathbb{R}}

\newenvironment{gaussjordandos}{\left( \begin{array}{cc|cc}}{\end{array}\right)}
\newenvironment{gaussjordantres}{\left( \begin{array}{ccc|ccc}}{\end{array}\right)}

\author{Departamento de Matemáticas}
\title{Matriz Inversa}
%\subtitle{Definición y cálculo}
%\setbeamercovered{transparent} 
%\setbeamertemplate{navigation symbols}{} 
%\logo{\includegraphics[scale=0.05]{../../images/logoa.jpg}} 
%\institute{UHEI - IVED} 
\date{\includegraphics[scale=0.15]{imagenes/logoa.jpg}} 
%\subject{} 
\begin{document}

\begin{frame}
\titlepage
\end{frame}

\begin{frame}
\tableofcontents
\end{frame}

\section{Definición}

\begin{frame}{Definición}

\begin{alertblock}{Definición de matriz inversa}
La matriz inversa de una matriz cuadrada $A$ de orden $n$ es otra matriz $A^{-1}$ de orden $n$ que verifica:
\[
A\cdot A^{-1}=A^{-1} \cdot A = I
\]
Las matrices que tienen inversa se llaman \textbf{matrices regulares} y las que no tienen inversa se llaman \textbf{matrices singulares}.
\end{alertblock}
\end{frame}


\begin{frame}{Métodos de calculo de la matriz inversa}

Para calcular la matriz inversa vamos a utilizar tres procedimientos:
\begin{enumerate}[<+-|alert@+>]
\item Mediante la definición.
\item Método de Gauss-Jordan.
\item Por determinantes.
\end{enumerate}

\end{frame}

\section{Calculo mediante la definición}

\begin{frame}{Cálculo mediante la definición}
En este procedimento escribiremos cada elemento de la matriz inversa como una incógnita y mediante la definición de matriz inversa plantearemos un sistema de ecuaciones que luego resolveremos calculando de esta manera cada uno de los elementos de la matriz inversa.

\end{frame}

\begin{frame}{Cálculo mediante la definición}

\begin{exampleblock}{Ejemplo}
Dada la matriz    $A= \begin{pmatrix} 
 2& -1 \\
 -7& 4 \\
\end{pmatrix}$, hallar su inversa, si existe.
\end{exampleblock}
\onslide*<2-4>{
La matriz $A$ es de orden $2$, por lo tanto, la inversa también será de orden $2$.

Es decir, $A^{-1}=\begin{pmatrix} 
 x& y \\
 z& t \\ 
\end{pmatrix}$}

\onslide*<3-4>{
Utilizaremos ahora la definición de matriz inversa y calcularemos las incógnitas y la matriz inversa.

$A\cdot A^{-1}= I \Leftrightarrow \begin{pmatrix} 
 2& -1 \\
 -7& 4 \\
\end{pmatrix} 
\begin{pmatrix} 
 x& y \\
 z& t \\ 
\end{pmatrix}
=\begin{pmatrix} 
 1& 0 \\
 0& 1 \\ 
\end{pmatrix}$}

\onslide*<4>{
Operando e igualando las dos matrices:

$\begin{pmatrix} 
 2x-z& 2y-t \\
 -7x+4z& -7y+4t \\ 
\end{pmatrix}=\begin{pmatrix} 
 1& 0 \\
 0& 1 \\ 
\end{pmatrix} \Leftrightarrow
\left. 
\begin{array}{l}
 2x-z = 1 \\
 2y-t = 0 \\
 -7x+4z = 0 \\
 -7y+4t = 1 \\
\end{array}
\right\rbrace
\Rightarrow 
\begin{array}{l}
 x = 4 \\
 y =1 \\
 z =7 \\
 t = 2 \\
\end{array} $}

\onslide*<5>{
Luego la matriz inversa de $A=\begin{pmatrix} 
 2& -1 \\
 -7& 4 \\
\end{pmatrix}$ es:
$A^{-1}= \begin{pmatrix}
 4& 1 \\
 7& 2 \\
\end{pmatrix}$} 
\end{frame}
\section{Calculo de la matriz inversa por el método de Gauss-Jordan}

\begin{frame}{Método de Gauss-Jordan}


En este método calcularemos la matriz inversa de una matriz mediante transformaciones elementales.

\pause

Este método consiste en lo siguiente:
\pause
\begin{enumerate}[<+-|alert@+>]
\item Tomamos una matriz formada por la matriz $A$ y la matriz identidad del mismo orden. Esta matriz la simbolizaremos por $\left( \begin{array}{c | c}
A &  I 
\end{array} \right) $
\item Realizamos transformaciones elementales para llegar a la matriz: $\left( \begin{array}{c | c} I & B 
\end{array} \right) $
\end{enumerate}
\onslide*<5->{
La matriz $B$ es la inversa de la matriz $A$, es decir $A^{-1}$}

\pause
Las transformaciones elementales son las siguientes:
\pause
\begin{enumerate}[<+-|alert@+>]
\item Intercambiar dos filas. $F_i \leftrightarrow F_j$
\item Multiplicar una fila por un número distinto de cero. $F_i \rightarrow kF_i$
\item Sumar dos filas multiplicadas por sendos números y sustituir una de estas filas por el resultado. $F_i \rightarrow kF_i+tF_j$.
\end{enumerate}

\end{frame}

\begin{frame}{Método de Gauss-Jordan}

\begin{exampleblock}{Ejemplo}
 Hallar la inversa de  la matriz:$\begin{pmatrix}
1 & 2  \\
2 & 3   \end{pmatrix}$.
\end{exampleblock}

\pause
$
\begin{gaussjordandos}
1 & 2  & 1& 0 \\
  2& 3  & 0& 1 \\
\end{gaussjordandos}
$
\pause
\onslide*<3>{$\alert{F_2=F_2-2F_1}$}
\onslide*<4->{$F_2=F_2-2F_1$}
\pause
$
\begin{gaussjordandos}
 1& 2  & 1& 0 \\
 0& -1  & -2& 1 \\
\end{gaussjordandos}
$
\pause
$\alert{F_1=F_1+2F_1} $
\pause
$\begin{gaussjordandos}
 1& 0  & -3& 2 \\
 0& -1  & -2& 1 \\
\end{gaussjordandos}$
\pause
$\alert{F_2=-F_2}$
\pause
$
\begin{gaussjordandos}
 1& 0 &  -3& 2 \\
 0& 1  & 2& -1 \\
\end{gaussjordandos}
$

Luego la matriz inversa de $A$ es $A^{-1}=\begin{pmatrix}
  -3& 2 \\
  2& -1 \\
\end{pmatrix}$

\end{frame}

\begin{frame}
\begin{exampleblock}{Ejemplo}
Hallar la inversa de la matriz $B=\begin{pmatrix}
1 & -1 & 1 \\
2 & 0 & 1 \\
-1 & 1 & 1
\end{pmatrix}$
\end{exampleblock}

\onslide*<2-4>{
$ \begin{gaussjordantres} 
1 & -1 & 1 & 1 & 0 &0 \\
2 & 0 & 1 & 0 & 1 & 0 \\
-1 & 1 & 1 & 0 & 0 & 1
\end{gaussjordantres} $
}
\onslide*<3-4>{
$\begin{array}{l}  \\  \rightarrow f_2=f_2-2f_1 \\  \rightarrow f_3=f_3+f_1 \end{array}$
}
\onslide*<4-6>{
$\begin{gaussjordantres} 
1 & -1 & 1 & 1 & 0 &0 \\
0 & 2 & -1 & -2 & 1 & 0 \\
0 & 0 & 2 & 1 & 0 & 1
\end{gaussjordantres}$}
\onslide*<5-6>{
$\begin{array}{l} \rightarrow f_1=2f_1+f_2 \\ \\ \\ \end{array}$
}
\onslide*<6-8>{
$\begin{gaussjordantres} 
2 & 0 & 1 & 0 & 1 &0 \\
0 & 2 & -1 & -2 & 1 & 0 \\
0 & 0 & 2 & 1 & 0 & 1
\end{gaussjordantres} $
}
\onslide*<7-8>{
$\begin{array}{l} \rightarrow f_1=2f_1-f_3 \\ \rightarrow f_2=2f_2+f_3 \\ \\ \end{array} $
}
\onslide*<8-10>{
$\begin{gaussjordantres} 
4 & 0 & 0 & -1 & 2 & -1 \\
0 & 4 & 0 & -3 & 2 & 1 \\
0 & 0 & 2 & 1 & 0 & 1
\end{gaussjordantres}$
}
\onslide*<9-10>{
$\begin{array}{l} \rightarrow f_1=f_1/4 \\ \rightarrow f_2=f_2/4 \\ \rightarrow f_3=f_3/2 \end{array}  $
}
\onslide*<10-11>{
$\begin{gaussjordantres} 
1 & 0 & 0 & -1/4 & 1/2 & -1/4 \\
0 & 1 & 0 & -3/4 & 1/2 & 1/4 \\
0 & 0 & 1 & 1/2 & 0 & 1/2
\end{gaussjordantres} $
}

\onslide*<11>{
$B^{-1}=\begin{pmatrix}
-1/4 & 1/2 & -1/4 \\
 -3/4 & 1/2 & 1/4 \\
 1/2 & 0 & 1/2
\end{pmatrix} $
}
\end{frame}
\section{Cálculo de la matriz inversa por determinantes}

\begin{frame}{Cálculo de la matriz inversa por determinantes}
La condición necesaria y suficiente para que una matriz tenga inversa es que su determinante sea distinto de cero.
\[ \exists A^{-1} \Leftrightarrow |A| \neq 0 \]

\pause

Si existe la matriz inversa se puede calcular de la siguiente forma:

\begin{center}
\resaltado{
\[ A^{-1}= \dfrac{1}{|A|}\cdot \left[ Adj (A) \right]^t = \dfrac{1}{|A|}\cdot Adj  \left( A^t \right) \]
} 
\end{center}

\end{frame}

\begin{frame}{Por determinantes}

\begin{exampleblock}{Ejemplo por determinantes}
Hallar la matriz inversa de :
$A=\begin{pmatrix}
1 & 0 & 0 \\
0 & 0 & -1 \\
2 & -1 & 1
\end{pmatrix} $
\end{exampleblock}
\onslide*<2-4>{
Hallamos el determinante de $A$.}

\onslide*<3-4>{
$|A|=0+0+0-0-1-0=-1 $}

\onslide*<4>{
Como el determinante es distinto de cero la matriz tiene inversa.}

\onslide*<5>{
Hallamos la matriz adjunta de $A$.}

\onslide*<6-7>{
$\begin{matrix}
A_ {11}=\begin{vmatrix}
0 & -1 \\
-1 & 1
\end{vmatrix}=-1 & \pause A_ {12}=-\begin{vmatrix}
0 & -1 \\
2 & 1
\end{vmatrix}=-2 & A_ {13}=\begin{vmatrix}
0 & 0 \\
2 & -1
\end{vmatrix}=0 \\
A_ {21}=-\begin{vmatrix}
0 & 0 \\
-1 & 1
\end{vmatrix}=0 & A_ {22}=\begin{vmatrix}
1 & 0 \\
2 & 1
\end{vmatrix}=1 & A_ {23}=-\begin{vmatrix}
1 & 0 \\
2 & -1
\end{vmatrix}=1 \\
A_ {31}=\begin{vmatrix}
0 & 0 \\
0 & -1
\end{vmatrix}=0 & A_ {32}=-\begin{vmatrix}
1 & 0 \\
0 & -1
\end{vmatrix}=1 & A_ {33}=\begin{vmatrix}
1 & 0 \\
0 & 0
\end{vmatrix}=0
\end{matrix} $}

\onslide*<7->{
$(Adj A)= \begin{pmatrix}
-1 & -2 & 0 \\
0 & 1 & 1 \\
0 & 1 & 0
\end{pmatrix} $}
\onslide*<8->{ 
$\qquad \left( Adj A \right)^t=\begin{pmatrix}
-1 & 0 & 0 \\
-2 & 1 & 1 \\
0 & 1 & 0
\end{pmatrix}  $}

\onslide*<9->{
\resultado{
$A^{-1}= \dfrac{\left( Adj A \right)^t}{|A|}=\begin{pmatrix}
1 & 0 & 0 \\
2 & -1 & -1 \\
0 & -1 & 0
\end{pmatrix}  $ 
}}


\end{frame}
\end{document}






