\documentclass[9pt]{beamer}
\usetheme{metropolis}
%\usetheme{Warsaw}
\metroset{titleformat=smallcaps,block=fill}
\usecolortheme{seahorse}
%\usepackage[utf8]{inputenc}
\usepackage[spanish]{babel}
\usepackage{amsmath}
\usepackage{amsfonts}
\usepackage{amssymb}
\usepackage{graphicx}
\usepackage{tikz} 
\usetikzlibrary{matrix}
\usetikzlibrary{calendar,decorations.markings} 
\usetikzlibrary{shapes,positioning}

\usepackage{tkz-tab,tkz-euclide,tkz-fct}
\usetkzobj{all}

\usepackage{tcolorbox}
\usepackage{varwidth}
\newtcbox{\resultado}{colback=red!5!white,
colframe=red!75!black,varwidth upper}

\newtcbox{\resaltado}{colback=blue!5!white,
colframe=blue!75!black,varwidth upper}

\usepackage{polynom}

\newcommand{\R}{\mathbb{R}}

\newenvironment{gaussjordandos}{\left( \begin{array}{cc|cc}}{\end{array}\right)}
\newenvironment{gaussjordantres}{\left( \begin{array}{ccc|ccc}}{\end{array}\right)}

\author{Departamento de Matemáticas}
\title{Ecuaciones matriciales}
%\subtitle{Definición y cálculo}
%\setbeamercovered{transparent} 
%\setbeamertemplate{navigation symbols}{} 
%\logo{\includegraphics[scale=0.05]{../../images/logoa.jpg}} 
%\institute{UHEI - IVED} 
\date{\includegraphics[scale=0.15]{imagenes/logoa.jpg}} 
%\subject{} 
\begin{document}
\begin{frame}{Ecuaciones matriciales}
Una ecuación matricial es una igualdad en la que intervienen matrices, siendo una de ellas o varias desconocidas.

Las ecuaciones matriciales se pueden resolver de forma directa o con la ayuda de la matriz inversa.
\end{frame}


\begin{frame}{Resolución de una ecuación matricial directamente}
\begin{exampleblock}{Ejemplo}
Resuelve la siguiente ecuación matricial $X \cdot A -B=C$, siendo:
\[ A= \begin{pmatrix} 1 & -2 \\ 2 & 3 \end{pmatrix} \qquad B=\begin{pmatrix} 0 & -2 \\ -2 & 4 \end{pmatrix} \qquad C= \begin{pmatrix} -1 & -10 \\ 3 & -6 \end{pmatrix} \] 
\end{exampleblock}

\pause
La matriz $X$ tiene que ser una matriz cuadrada de orden 2. Es decir $X=\begin{pmatrix} x & y \\ z & t \end{pmatrix}$

\pause
Planteamos la ecuación $X\cdot A -B = C$ y operamos:

\pause
$\begin{pmatrix} x & y \\ z & t \end{pmatrix} \begin{pmatrix} 1 & -2 \\ 2 & 3 \end{pmatrix} - \begin{pmatrix} 0 & -2 \\ -2 & 4 \end{pmatrix} = \begin{pmatrix} -1 & -10 \\ 3 & -6 \end{pmatrix}$

\pause
$\begin{pmatrix} x+2y & -2x+3y \\ z+2t & -2z+3t \end{pmatrix}- \begin{pmatrix} 0 & -2 \\ -2 & 4 \end{pmatrix} = \begin{pmatrix} -1 & -10 \\ 3 & -6 \end{pmatrix}$

\pause
$\begin{pmatrix} x+2y & -2x+3y+2 \\ z+2t+2 & -2z+3t-4 \end{pmatrix} = \begin{pmatrix} -1 & -10 \\ 3 & -6 \end{pmatrix}$

\pause
Igualamos las dos matrices y conseguimos el siguiente sistema de ecuaciones:

$\begin{cases} x+2y = -1 \\ -2x+3y+2=-10  \\ z+2t+2= 3 \\ -2z+3t-4=-6 \end{cases}$

\pause
Resolvemos el sistema:

$\begin{cases} x= 3 \\ y = -2 \\ z=1 \\ t =0 \end{cases}$

\pause
La matriz $X$ es: 

\resultado{$X= \begin{pmatrix} 3 & -2 \\ 1 & 0 \end{pmatrix}$}

\end{frame}
\end{document}