\documentclass[8pt]{beamer}
%\usetheme{CambridgeUS}
\logo{\includegraphics[scale=0.10]{../imagenes/logoa}}
\usepackage[spanish]{babel}
%\usecolortheme{seahorse}
%\usepackage{beamerthemeblackboard}
%\usepackage{graphics}
%\usecolortheme[RGB={6,138,200}]{structure}
%\usepackage[orientation=landscape, size=custom,
 % width=16, height=9, scale=0.5]{beamerposter}
%\usepackage[utf8]{inputenc}
\usetheme{metropolis}
\metroset{titleformat=smallcaps,block=fill}
\usepackage{booktabs}
\usepackage[scale=2]{ccicons}

\usepackage{amsmath}
\usepackage{amsfonts}
\usepackage{amssymb}
\usepackage{graphicx}
\usepackage{colortbl}
\usepackage{tikz} 
\usetikzlibrary{matrix}
\usetikzlibrary{calendar,decorations.markings} 
\usetikzlibrary{shapes,positioning}

\usepackage{tkz-tab,tkz-euclide,tkz-fct}
\usetkzobj{all}  
\usepackage{tcolorbox} 
%\usepackage{enumitem} 
\usepackage{tasks} 
\usepackage{asymptote}  
\usepackage{cancel}
\newcommand{\sen}{\mathop{\rm sen}\nolimits}

\newcommand{\tg}{\mathop{\rm tg}\nolimits}
\newcommand{\arcsen}{\mathop{\rm arcsen}\nolimits}
\newcommand{\arctg}{\mathop{\rm arctg}\nolimits}
\newcommand{\g}{{}^\circ}     
        
\newcommand{\R}{\mathbb{R}}
\newcommand{\Z}{\mathbb{Z}}
\newcommand{\N}{\mathbb{N}}
\newcommand{\Q}{\mathbb{Q}}
\newcommand{\I}{\mathbb{I}}
\newcommand{\limite}[2]{\displaystyle \lim_{x \rightarrow #1}{#2}}
\renewcommand{\vector}[1]{\overrightarrow{#1}}

\newtcbox{\resultado}[1][center]{#1,colback=red!5!white,
colframe=red!75!black}

\newtcbox{\resaltado}[1][center]{#1,colback=blue!5!white,
colframe=blue!75!black}
\definecolor{titleColor}{rgb}{0.0, 0.42, 0.24}
\definecolor{miverde}{RGB}{59,93,43}
\definecolor{verdep}{RGB}{139,179,106}

\title{Derivadas}
\author{Ricardo Mateos}
\institute[UHEI-IVED]{Departamento de Matemáticas \\ UHEI - IVED}
\date{Matemáticas II}
\begin{document}
%\ECFJD
\begin{frame}
\maketitle
\end{frame}

\begin{frame}
\tableofcontents
\end{frame}
\section{Tasa de variación media}

\begin{frame}{Tasa de variación media}

\begin{alertblock}{Tasa de variación media}
La \textbf{tasa de variación media} de una función $f(x)$ en un intervalo $[a,b]$ es igual al incremento de la función $f(x)$ entre el incremento de la variable independiente $x$:
\[ \text{T.V.M.}[a,b] = \dfrac{f(b)-f(a)}{b-a} \]
\end{alertblock}

\pause
La tasa de variación media de una función en un punto mide el aumento o la disminución de la función en dicho intervalo.
\end{frame}

\begin{frame}{Tasa de variación media}
\begin{exampleblock}{Ejemplo}
Hallar la tasa de variación media de la función $f(x)=3t^2-2t+1$, en los intervalos $[0,2]$ y $[2,6]$	
\end{exampleblock}
\pause

$\text{T.V.M.}[0,2] = \dfrac{f(2)-f(0)}{2-0}= \dfrac{9-1}{2-0}=4$

\pause

$\text{T.V.M.}[2,6] = \dfrac{f(6)-f(2)}{6-2}= \dfrac{97-9}{6-2}=22$


\end{frame}
\begin{frame}{Interpretación geométrica de la T.V.M}
\begin{tikzpicture}

	\tkzInit[xmin=-1,xmax=8, ymin=-1,ymax=5]
	\tkzDrawX[label={x},noticks]
	\tkzDrawY[label={y},noticks]
	\tkzDefPoints{-0.2/4/A, -0.2/1/B, 7.5/4/C, 7.5/1/D, 7/4.2/E, 7/-0.2/F, 7/4/G, 1/-0.2/H, 1/1/I}
	\tkzDefPoints{7.2/4.1/P, 0.8/0.9/Q}
	\tkzDrawSegments[dotted](A,C B,D E,F H,I)
	\tkzDrawSegment[color=blue](P,Q)
	\tkzDrawPoint(G)
	\tkzLabelPoint[above left](G){$P_1$}	
	\tkzLabelSegment[left,pos=0](A,C){$f(b)$}
	\tkzLabelSegment[left,pos=0](B,D){$f(a)$}
	\tkzLabelSegment[below,pos=0](F,E){$b$}
	\tkzLabelSegment[below,pos=0](H,I){$a$}
	\tkzFct[color=red,domain=0:7.3]{(x**2-2*x+13)/12)}
	%\tkzFct[color=red,domain=0:4]{-x}
\end{tikzpicture}

La tasa de variación media es la pendiente de la recta secante que pasa por los puntos $(a,f(a))$ y $(b,f(b))$
\end{frame}

\section{Derivada de una función en un punto}
\begin{frame}{Derivada de una función en un punto}
\begin{alertblock}{Derivada de una función en un punto}
La derivada de una función en un punto $a$ se llama $f'(a)$ o $ \dfrac{dy}{dx}$ y es igual a:

\[ f'(a)=\lim_{x \rightarrow a} \dfrac{f(x)-f(a)}{x-a} \]

si este límite existe y es finito.

\pause
Si en la fórmula anterior hacemos que $x-a = h \Rightarrow \begin{cases} x=a+h \\ h \rightarrow 0 \end{cases}$

quedaría de la siguiente forma:

\[ f'(a)=\lim_{h \to 0} \dfrac{f(a+h)-f(a)}{h} \]
\end{alertblock}

\end{frame}

\begin{frame}{Derivada de una función en un punto}
\begin{exampleblock}{Ejemplo}
Hallar la derivada de la función $f(x)=x^2+2x-2$ en el punto $x=1$
\end{exampleblock}

\pause

Aplicamos la primera fórmula:
\[ f(1)= 1^2+2\cdot 1 -2 = 1 \]
\pause
y por lo tanto 

\pause
\[ f'(1)=\lim_{x \rightarrow 1} \dfrac{(x^2+ 2x-2) -1 }{x-1}= \lim_{x \rightarrow 1} \dfrac{x^2+ 2x-3 }{x-1} = \lim_{x \rightarrow 1} \dfrac{\cancel{(x-1)}(x+3) }{\cancel{x-1}}= 4 \] 
\pause
Aplicamos la segunda expresión:

\pause
Calculamos $f(1)$ y $f(1+h)$
\[ f(1)=1^2+2\cdot 1 -2 = 1 \]
\[ f(1+h)=(1+h)^2+2(1+h)-2= 1+2h+h^2+2+2h-2= h^2+4h+1 \]
\pause
Hallamos el límite:
\pause
\[ \displaystyle f'(1)=\lim_{h \to 0} {\dfrac{h^2+4h+1-1}{h}}= \lim_{h \to 0 } {\dfrac{h^2+4h}{h}}=\lim_{h \to 0} {\dfrac{\cancel{h}(h+4)}{\cancel{h}}}=\lim_{h \to 0} (h+4) = 4 \]
\pause
La derivada de $f(x)$ en el punto $x=1$ vale $4$.

\end{frame}

\begin{frame}{Interpretación geométrica de la derivada}
\begin{tikzpicture}

	\tkzInit[xmin=-1,xmax=8, ymin=-1,ymax=5]
	\tkzDrawX[label={x},noticks]
	\tkzDrawY[label={y},noticks]
	\tkzDefPoints{-0.2/4/A, -0.2/1.08/B, 7.5/4/C, 7.5/1.08/D, 7/4.2/E, 7/-0.2/F, 7/4/G, 2/-0.2/H, 2/1.08/I}
	\tkzDrawSegments[dotted](B,D H,I)
	%\tkzDrawPoint(G)
	%\tkzLabelPoint[above left](G){$P_1$}	
	%\tkzLabelSegment[left,pos=0](A,C){$f(b)$}
	\tkzLabelSegment[left,pos=0](B,D){$f(a)$}
	%\tkzLabelSegment[below,pos=0](F,E){$b$}
	\tkzLabelSegment[below,pos=0](H,I){$a$}
	%\tkzAxeXY[color=black,orig=false,label options={font=\tiny},swap]
	\tkzFct[color=red,domain=0:7.3]{(x**2-2*x+13)/12)}
	\tkzFct[color=blue,domain=0.8:7.2]{(2*x+9)/12}
\end{tikzpicture}
\pause

La derivada de una función en un punto coincide con la pendiente de la recta tangente a la gráfica de la función en dicho punto. 

$f'(a)=$ pendiente de la recta tangente.
\end{frame}
\section{Derivadas laterales}
\begin{frame}{Derivadas laterales}

La derivada de una función en un punto es un límite, por lo tanto, para que exista la derivada de una función en un punto tienen que existir los límites laterales y ser iguales.

\pause
\begin{alertblock}{Derivadas laterales}
Las derivadas laterales de una función en un punto son:

\begin{itemize}
\item Derivada por la derecha:
\[ f'(a^+)=\lim_{h \rightarrow 0^+} \dfrac{f(a+h)-f(a)}{h} \]
\pause
\item Derivada por la izquierda:
\[ f'(a^-)=\lim_{h \rightarrow 0^-} \dfrac{f(a+h)-f(a)}{h} \]
\end{itemize}


\end{alertblock}
\pause
Para que exista la derivada de la función en el punto tienen que existir ambas derivadas y ser iguales.
\end{frame}

\begin{frame}{Derivadas laterales}
\begin{exampleblock}{Ejemplo}
Hallar las derivadas laterales de la función $f(x)=\begin{cases} x^2+1 & x \leq 1 \\ 2x & x>1 \end{cases}$ en el punto $x=1$
\end{exampleblock}
\pause
Primero hallamos cuanto vale $f(1)$ y luego hallamos sus derivadas laterales.

$f(1)=1^2+1 = 2$

\pause
$\displaystyle f'(1^+)=\lim_{h \rightarrow 0^+} \dfrac{2(1+h)-2}{h}= \lim_{h \rightarrow 0^+} \dfrac{2+2h-2}{h}=\lim_{h \rightarrow 0^+} \dfrac{2\cancel{h}}{\cancel{h}}= 2 $

\pause
$\displaystyle f'(1^-)=\lim_{h \rightarrow 0^-} \dfrac{((1+h)^2+1) -2}{h}= \lim_{h \rightarrow 0^-} \dfrac{1+2h+h^2+1-2}{h}=\lim_{h \rightarrow 0^-} \dfrac{\cancel{h}(h+2)}{\cancel{h}}= 2 $

\pause
Luego existe la derivada de la función en el punto y es igual a $2$ : $f'(1)=2$

\end{frame}

\begin{frame}{Derivabilidad y continuidad}
\begin{alertblock}{Derivabilidad y continuidad}
Para que una función sea derivable en un punto $x=a$ la función tiene que ser continua en ese punto.

Esta es la condición necesaria pero no suficiente para que una función sea derivable en un punto.
\end{alertblock}

\pause
\begin{itemize}
\item \textbf{Si una función en un punto es continua entonces puede ser derivable o no ser derivable.}
\pause
\item \textbf{Si una función no es continua en un punto entonces no es derivable.}
\end{itemize}

\end{frame}

\begin{frame}{Derivabilidad y continuidad}
\begin{exampleblock}{Ejemplo}
Estudiar la continuidad y derivabilidad de la función $f(x)=3x+|x+3|$ en el punto $x=-3$	
\end{exampleblock}
\pause
Para calcular la continuidad y derivabilidad de esta función vamos a expresarla como una función definida a trozos.

$f(x)=3x+|x+3| = \begin{cases} 2x-3 & x < -3 \\ 4x+3 & x \geq -3 \end{cases} $

\pause
Estudiamos la continuidad:
\[ f(-3)= 4\cdot(-3) +3 = -9 \]
\[ \displaystyle \left. \begin{array}{l} \displaystyle \lim_{x \rightarrow 3^-} (2x-3) = -9 \\ \displaystyle \lim_{x \rightarrow 3^+} (4x+3) = -9 \end{array} \right\rbrace \Rightarrow \lim_{x \rightarrow 3} f(x) = -9 \] 
\pause
Como vemos existe la función y el límite en el punto y además son iguales, por lo tanto, la función es continua en el punto $x=-3$. Por lo tanto, la función puede ser derivable en dicho punto.
\end{frame}
\begin{frame}{Derivabilidad y continuidad}
A continuación, estudiamos su derivabilidad:
\pause
\[ \displaystyle \left. \begin{array}{l}\displaystyle  f'(-3^-) = \lim_{ h \rightarrow 0^-} \dfrac{(2(-3+h)-3)-(-9)}{h} = \lim_{ h \rightarrow 0^-} \dfrac{2\cancel{h}}{\cancel{h}}= 2 \\ \displaystyle f'(-3^+)= \lim_{ h \rightarrow 0^+} \dfrac{(4(-3+h)+3)-(-9)}{h} = \lim_{ h \rightarrow 0^+} \dfrac{4\cancel{h}}{\cancel{h}}= 4 \end{array} \right\rbrace \Rightarrow \nexists f'(-3) \]
\pause
La función no es derivable, ya que las derivadas laterales son distintas.



\end{frame}
\section{Función derivada}
\begin{frame}[t]{Función derivada}
\begin{alertblock}{Función derivada}
La función derivada de una función $f(x)$ es una nueva función, $f'(x)$, que asocia a cada punto $x$ la derivada de esa función en ese punto.
\[
 f'(x)=\lim_{h \to 0} \dfrac{f(x+h)-f(x)}{h} 
\]
\end{alertblock}
\end{frame}
\begin{frame}{Tabla de derivadas}

\setlength{\arrayrulewidth}{0.2mm}
\setlength{\tabcolsep}{10pt}
\renewcommand{\arraystretch}{2}
\newcolumntype{s}{>{\columncolor[HTML]{AAACED}} p{1.5cm}}
\arrayrulecolor{miverde}


\begin{table}[H]
\begin{center}
\begin{tabular}{|p{1.2cm}|p{1.2cm}|p{1.2cm}|p{1.2cm}|}

\hline
 \textbf{Función} & \textbf{Derivada} & \textbf{Función} & \textbf{Derivada}\\
 \hline
	 $k$ & $0 $ & $\sin x$ & $ \cos x$ \\
\rowcolor{verdep}	 $x $ & $1 $ & $ \cos x$ & $ -\sin x$\\
	 $x^n $ & $n x^{n-1} $ & $ \tan x$ & $ 1+\tan^2 x $\\
\rowcolor{verdep}	 $ e^x $ & $ e^x$ & $ \tan x$ & $\dfrac{1}{\cos^2 x} $\\
	 $ a^x$ & $ a^x \cdot \ln a$ & $ \arcsin x$ & $ \dfrac{1}{\sqrt{1-x^2}}$ \\
\rowcolor{verdep}	 $ \ln x$ & $\dfrac{1}{x} $ & $ \arccos x$ & $ \dfrac{-1}{\sqrt{1-x^2}}$\\
	 $ \log_ a x$ & $ \dfrac{1}{x \cdot \ln a}$ &  $ \arctan x$ & $ \dfrac{1}{1+x^2}$ \\
\hline
\end{tabular}
\caption{\label{tab:tabla-derivadas}Derivadas de las funciones elementales.}
\end{center}
\end{table}


\end{frame}

\begin{frame}{Álgebra de derivadas}

\setlength{\arrayrulewidth}{0.2mm}
\setlength{\tabcolsep}{10pt}
\renewcommand{\arraystretch}{2}
\newcolumntype{s}{>{\columncolor[HTML]{AAACED}} p{1.5cm}}
\arrayrulecolor{miverde}
\begin{center}
\begin{tabular}{|p{8cm}|}
\hline
\rowcolor{lightgray} {\huge{Álgebra de derivadas}} \\

	 $\left( f(x)+g(x) \right)' =f'(x) + g'(x) $ \\
\rowcolor{verdep}	  $\left( f(x)-g(x) \right)' =f'(x) - g'(x) $ \\
	 $\left( k\cdot f(x) \right)' =k\cdot f'(x) $ \\
\rowcolor{verdep}	 $\left( f(x) \cdot g(x) \right)' =f'(x)\cdot g(x)  + f(x) \cdot g'(x) $ \\
   
	  $\left( \dfrac{f(x)}{g(x)} \right)' =\dfrac {f'(x)\cdot g(x)-f(x)\cdot  g'(x)}{g^2(x)} $ \\
\rowcolor{verdep}	 $ \left[ f\left(g(x) \right)\right]' =f'\left(g(x) \right) \cdot  g'(x) $ \\
\hline
\end{tabular}
\end{center}
\end{frame}

\begin{frame}[t]{Derivadas}
\begin{exampleblock}{Ejemplo}
Calcular las siguientes derivadas:
\begin{tasks}[label=\alph*)](2)
\task $f(x)=5x^4-3x^3+4x-2$
\task $g(x)= 5x^3-\sqrt[4]{x}$
\task $h(x)= x^2 \cdot \sen x$
\task $m(x)=\dfrac{3+2x}{x^2}$
\end{tasks}
\end{exampleblock}
\end{frame}

\begin{frame}[t]{Derivadas}
\begin{exampleblock}{Ejemplo}
Calcular las siguientes derivadas:
\begin{tasks}[label=\alph*)](2)
\task $f(x)= (x^3+3x-2)^7 $
\task $g(x)= e^{x^2+3}$
\task $h(x)= \ln (1-3x)$
\task $m(x)= \cos (x^3+2x-3)$
\end{tasks}
\end{exampleblock}
\end{frame}

\begin{frame}[t]{Derivación logarítmica}
Cuando queremos calcular la derivada de una función elevada a otra función, tenemos que tomar logaritmos en los dos miembros y luego derivar.

\begin{alertblock}{Derivación logarítmica}
$f(x)=g(x)^{h(x)}$

$\ln f(x)= \ln g(x)^{h(x)} = h(x) \cdot \ln g(x)$

$\dfrac{1}{f(x)}\cdot f'(x)= h'(x) \cdot \ln g(x)+ h(x) \cdot \dfrac{1}{g(x)}\cdot g'(x)$
\end{alertblock}
\end{frame}

\begin{frame}[t]{Derivación logarítmica}
\begin{exampleblock}{Ejemplo}
Hallar las derivadas de la siguientes funciones
\begin{tasks}[label=\alph*)](2)
\task $ f(x)= x^x$
\task $f(x)=\left( x-\sen x \right)^{x}$
\end{tasks}
\end{exampleblock}
\end{frame}

\begin{frame}[t]{Ejemplo}
\begin{exampleblock}{Ejemplo}
Calcular el valor de los parámetros para que la función:
\[f(x)=\left\{\begin{array}{cc} {ax+b} & {x<0} \\ {x^{2} -3x+2} & {x\geq 0} \end{array}\right. \] 
Sea continua y derivable.
\end{exampleblock}

\end{frame}
\end{document}