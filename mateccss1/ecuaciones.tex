\documentclass[9pt]{../if-beamer}
%\usetheme{metropolis}
%\usetheme{Warsaw}
%\metroset{titleformat=smallcaps,block=fill}
\usecolortheme{seahorse}
%\usepackage[utf8]{inputenc}
\usepackage[spanish]{babel}
\usepackage{amsmath}
\usepackage{amsfonts}
\usepackage{amssymb}
\usepackage{graphicx}
\usepackage{tikz} 
\usetikzlibrary{matrix}
\usetikzlibrary{calendar,decorations.markings} 
\usetikzlibrary{shapes,positioning}
\usepackage{tcolorbox}
\usepackage{tkz-tab,tkz-euclide,tkz-fct}
\usetkzobj{all} 
\usepackage{polynom}

\usepackage{extarrows}
%Nuevos entornos
\newenvironment{matrizgauss}{\left( \begin{array}{ccc|ccc}}{\end{array} \right)}
\newenvironment{matrizampliada}{\left( \begin{array}{ccc|c}}{\end{array} \right)}
\newenvironment{sistematres}{\left\lbrace \begin{array}{rrrrrrr}}{\end{array} \right.}
\newenvironment{sistemados}{\left\lbrace \begin{array}{rrrr}}{\end{array} \right.}
\newenvironment{adjunto}{\[ \begin{array}{lll}}{\end{array} \]}
\newenvironment{sistema}{\left\lbrace \begin{array}{rrrrrr}}{\end{array} \right. }

%nuevos comandos

\newcommand{\limite}[2]{\displaystyle \lim_{x \rightarrow #1}{#2}}
\newcommand{\limiteserie}[1]{\displaystyle \lim_{n \rightarrow +\infty}{#1}}
\newcommand{\matrizdos}[4]{\begin{pmatrix} #1 & #2 \\  #3 & #4  \end{pmatrix}}
\newcommand{\determinantedos}[4]{\begin{vmatrix} #1 & #2 \\ #3 & #4  \end{vmatrix}}
\newcommand{\matriztres}[9]{\begin{pmatrix} #1 & #2 & #3 \\ #4 & #5 & #6 \\ #7 & #8 & #9 \end{pmatrix}}
\newcommand{\determinantetres}[9]{\begin{vmatrix} #1 & #2 & #3 \\ #4 & #5 & #6 \\ #7 & #8 & #9 \end{vmatrix}}
\newcommand{\filasistematres}[6]{#1 & #2 & #3 &#4 & #5 & = & #6}
\newcommand{\integral}[1]{\displaystyle \int #1}
\newcommand{\intdef}[3]{\displaystyle \int_#1^#2 #3}


%%Comandos para abreviar los entornos%%

\newcommand{\problema}[1]{\begin{ejer} #1  \end{ejer}}
\newcommand{\sol}[1]{\begin{solu} #1 \end{solu}}

\newtcbox{\resultado}[1][center]{#1,colback=red!5!white,
colframe=red!75!black}

\newtcbox{\resaltado}[1][center]{#1,colback=blue!5!white,
colframe=blue!75!black}

\newenvironment{gaussjordandos}{\left( \begin{array}{cc|cc}}{\end{array}\right)}
\newenvironment{gaussjordantres}{\left( \begin{array}{ccc|ccc}}{\end{array}\right)}


\newcommand{\R}{\mathbb{R}}

\author{Departamento de Matemáticas}
\title{Ecuaciones}
%\subtitle{Definición y cálculo}
%\setbeamercovered{transparent} 
%\setbeamertemplate{navigation symbols}{} 
%\logo{\includegraphics[scale=0.05]{../../images/logoa.jpg}} 
%\institute{UHEI - IVED} 
\date{\includegraphics[scale=0.15]{../imagenes/logoa.jpg}}
\begin{document}

\begin{frame}
\titlepage
\end{frame}

\begin{frame}
\tableofcontents
\end{frame}

\section{Ecuaciones de segundo grado}

\begin{frame}{Ecuaciones de segundo grado}



Las ecuaciones de segundo grado tienen la forma 

\[ ax^2+bx+c=0 \]

\pause
Para resolverla utilizamos la siguiente fórmula:
\[x=\dfrac{-b\pm\sqrt{b^2-4ac}}{2a}\]
\pause

Pudiéndose dar tres casos:
\pause

\begin{enumerate}[<+->]
\item $\sqrt{b^2-4ac}>0 \Rightarrow  \mbox{dos soluciones} $
\item $\sqrt{b^2-4ac}=0  \Rightarrow  \mbox{una única solución}$
\item $\sqrt{b^2-4ac}<0  \Rightarrow  \mbox{No tiene solución}$
\end{enumerate}
\end{frame}

\begin{frame}{Ejemplo ecuación segundo grado}
\begin{exampleblock}
Resolver la siguiente ecuación: $x^2-10x+21=0$
\end{exampleblock}

\pause
En este caso:

\pause
$ \begin{cases} a= 1 \\ \pause b= -10 \pause \\ c=21 \end{cases} $

\pause
Aplicando la fórmula:

 \begin{equation*}
 \begin{split}
 x & =  \dfrac{-(-10) \pm \sqrt{(-10)^2 - 4 \cdot 1 \cdot 21 }}{2 \cdot 1}   \\ \pause
 & =  \dfrac{10 \pm \sqrt{100 - 84 }}{2} \\
 & =  \dfrac{10 \pm \sqrt{16}}{2} \\
 & =  \dfrac{10 \pm 4 }{2} \Rightarrow \begin{cases} x_1 = 7 \\ x_2 = 3 \end{cases} \\
 \end{split}
 \end{equation*} 
 
\end{frame}


\end{document}