\documentclass[8pt]{beamer}
%\usetheme{CambridgeUS}
\logo{\includegraphics[scale=0.10]{../imagenes/logoa}}
\usepackage[spanish]{babel}
%\usecolortheme{seahorse}
%\usepackage{beamerthemeblackboard}
%\usepackage{graphics}
%\usecolortheme[RGB={6,138,200}]{structure}
%\usepackage[orientation=landscape, size=custom,
 % width=16, height=9, scale=0.5]{beamerposter}
%\usepackage[utf8]{inputenc}
\usetheme{metropolis}
\metroset{titleformat=smallcaps,block=fill}
\usepackage{booktabs}
\usepackage[scale=2]{ccicons}

\usepackage{amsmath}
\usepackage{amsfonts}
\usepackage{amssymb}
\usepackage{graphicx}
\usepackage{colortbl}
\usepackage{tikz} 
\usetikzlibrary{matrix}
\usetikzlibrary{calendar,decorations.markings} 
\usetikzlibrary{shapes,positioning}

\usepackage{tkz-tab,tkz-euclide,tkz-fct}
\usetkzobj{all}  
\usepackage{tcolorbox} 
%\usepackage{enumitem} 
\usepackage{tasks} 
\usepackage{asymptote}  
\newcommand{\sen}{\mathop{\rm sen}\nolimits}

\newcommand{\tg}{\mathop{\rm tg}\nolimits}
\newcommand{\arcsen}{\mathop{\rm arcsen}\nolimits}
\newcommand{\arctg}{\mathop{\rm arctg}\nolimits}
\newcommand{\g}{{}^\circ}     
        
\newcommand{\R}{\mathbb{R}}
\newcommand{\Z}{\mathbb{Z}}
\newcommand{\N}{\mathbb{N}}
\newcommand{\Q}{\mathbb{Q}}
\newcommand{\I}{\mathbb{I}}
\newcommand{\limite}[2]{\displaystyle \lim_{x \rightarrow #1}{#2}}
\renewcommand{\vector}[1]{\overrightarrow{#1}}

\newtcbox{\resultado}[1][center]{#1,colback=red!5!white,
colframe=red!75!black}

\newtcbox{\resaltado}[1][center]{#1,colback=blue!5!white,
colframe=blue!75!black}
\definecolor{titleColor}{rgb}{0.0, 0.42, 0.24}

\title{Funciones}
\author{Ricardo Mateos}
\institute[UHEI-IVED]{Departamento de Matemáticas \\ UHEI - IVED}
\date{Matemáticas CC.SS. I}
\begin{document}
%\ECFJD
\begin{frame}
\maketitle
\end{frame}



\begin{frame}{Dominio de una función}
\begin{exampleblock}{Ejemplo dominio}
Hallar el dominio de definición de las siguientes funciones:
\begin{tasks}[label=\alph*)](2)
\task $f(x)=x^3+3x^+x-2$
\task $f(x)= \sqrt{x-2}$
\task $f(x)= \dfrac{x^2-4}{x^2-1}$
\task $f(x)=\dfrac{\sqrt{x-2}}{x^2 - 3x-4}$
\end{tasks}
\end{exampleblock}
\end{frame}

\begin{frame}
\tableofcontents
\end{frame}

\begin{frame}
\begin{columns}
\begin{column}{0.5\textwidth}
\begin{tikzpicture}[scale=0.35]
\tkzInit[xmin=-3,xmax=3,ymin=-5,ymax=6]
\tkzDrawX
\tkzDrawY
%\tkzText[draw,color=blue,fill=orange!20](2,3){$f(x)=x^2-4$}
\tkzFct[color=blue,domain=-3:3]{(x**2)-4}
%\tkzFct[color=blue,domain=0:9]{-sqrt(x)}
\end{tikzpicture}
\begin{tikzpicture}[scale=0.45]
\tkzInit[xmin=-5,xmax=2,ymin=-1,ymax=3]
\tkzDrawX
\tkzDrawY
%\tkzText[draw,color=blue,fill=orange!20](2,3){$f(x)=xe^{x}$}
\tkzFct[color=blue,domain=-5:1]{x*exp(x)}
\end{tikzpicture}
\end{column}
\begin{column}{0.5\textwidth}
\begin{tikzpicture}[scale=0.35]
\tkzInit[xmin=-3,xmax=3,ymin=-5,ymax=6]
\tkzDrawX
\tkzDrawY
%\tkzText[draw,color=blue,fill=orange!20](2,3){$f(x)=x^2-4$}
\tkzFct[color=blue,domain=-3:3]{(x**2)-4}
%\tkzFct[color=blue,domain=0:9]{-sqrt(x)}
\end{tikzpicture}
\begin{tikzpicture}[scale=0.35]
\tkzInit[xmin=-0.5,xmax=10,ymin=-5,ymax=5]
\tkzDrawX
\tkzDrawY
%\tkzText[draw,color=blue,fill=orange!20](2,3){$f(x)=\sqrt{x}$}
\tkzFct[color=blue,domain=0:9]{sqrt(x)}
\tkzFct[color=blue,domain=0:9]{-sqrt(x)}
\end{tikzpicture}
\end{column}
\end{columns}

\end{frame}

\begin{frame}

\end{frame}

\begin{frame}
\begin{tikzpicture}
\tkzInit[xmin=-5,xmax=2,ymin=-1,ymax=3]
\tkzDrawX
\tkzDrawY
\tkzText[draw,color=blue,fill=orange!20](2,3){$f(x)=xe^{x}$}
\tkzFct[color=blue,domain=-5:1]{x*exp(x)}
%\tkzFct[color=blue,domain=0:9]{-sqrt(x)}
\end{tikzpicture}
\end{frame}

\end{document}